\section{(Limits of) Empirical Performance Measures}
\label{sec:limits-of-emperical}

In the previous section we used the arrival, service and departure processes to define the arrival, service and departure rates.
We can also use these processes to construct a single-server queueing process, cf.~\cref{sec:constr-gg1-queu}, and, once we have the queueing process, we can compute the waiting time process $\{W_{Q,k}\}$, the sojourn time process $\{W_{Q,k}\}$, and the process $\{L(t)\}$ corresponding to the number of the jobs in the system.
Finally, if the queueing system is rate-stable, we can sensibly define a number of performance measures such as the average waiting time.
This we do here and refer to \cref{fig:constructiongg1} for an overview of the relations between these performance measures.


\opt{solutionfiles}{
\subsection*{Theory and Exercises}
\Opensolutionfile{hint}
\Opensolutionfile{ans}
}


With $\{W_k\}$ and $\{W_{Q,k}$, define the \recall{expected sojourn time} as
\begin{equation}\label{eq:49}
 \E W = \lim_{n\to\infty} \frac 1n\sum_{k=1}^n W_k,
\end{equation}
and the expected waiting time, or time in queue, as
\begin{equation}\label{eq:50}
 \E{W_Q} = \lim_{n\to\infty} \frac 1 n\sum_{k=1}^n W_{Q,k}.
\end{equation}
Note that these performance measures are limits of \emph{empirical} measures and are obtained at statistics \emph{observed by arriving jobs}: the first job has a sojourn time $W_1$ when it arrives, the second a sojourn time $W_2$, and so on.
For this reason, we colloquially say that $\E W$ is the average sojourn time as `seen by arrivals'.
The \emph{distribution of the sojourn times} seen by arrivals can be found by counting:
\begin{equation}\label{eq:48}
 \P{W \leq x} = \lim_{n\to\infty} \frac 1n\sum_{k=1}^n \1{W_k\leq x}.
\end{equation}
The (sample) \recall{average number of jobs} in the system as seen by arrivals is given by
\begin{equation}\label{eq:EQ}
\E L = \lim_{n\to\infty}\frac 1 n \sum_{k=1}^n L(A_k-),
\end{equation}
since $L(A_k-)$ is the number of jobs in the system at the arrival epoch of the $k$th job.
Finally, the \emph{distribution of $\{L(t)\}$} as seen by arrivals, is given by
\begin{equation}\label{eq:Qm}
\P{L\leq m} = \lim_{n\to\infty} \frac 1 n \sum_{k=1}^n \1{L(A_k-) \leq m}.
\end{equation}



A related set of performance measures follows by tracking the system's behavior over time and taking the \emph{time-average}, rather than the average at sampling (observation) moments.
% Thus, if we simulate the queueing system up to time $t$, 
% \begin{equation}\label{eq:11}
% \frac 1 t\int_0^t L(s)\d s = \frac 1 t\int_0^t (A(s)-D(s)) \d s,
% \end{equation}
% where we use that $L(s)=A(s) - D(s) + L(0)$ is the total number of jobs in
% the system at time $s$ and $L(0)=0$, cf.~\cref{fig:atltdt}. Observe from the second equation that $\int_0^t L(s)\d s$ is the area enclosed between the graphs of $\{A(t)\}$
% and $\{D(t)\}$. 
Assuming the limit exists we use~\cref{eq:14} to define the \recall{time-average number of jobs} as
\begin{equation}
 \label{eq:46}
 \E L = \lim_{t\to\infty} \frac 1 t\int_0^t L(s) \d s.
\end{equation}
Observe that, notwithstanding that the symbols are the same, this expectation need not be the same as~\cref{eq:EQ}.
In a loose sense we can say that $\E L$ is the average number in the system as perceived by the \emph{server}.
Next, define the \emph{time-average fraction of time the system contains at most $m$ jobs} as
\begin{equation}
 \label{eq:47}
 \P{L\leq m} =\lim_{t\to\infty} \frac 1 t\int_0^t \1{L(s)\leq m} \d s.
\end{equation}
Again, this probability need not be the same as what customers see upon arrival.


\begin{exercise}\clabel{ex:l-165}
Design a queueing system to show that the average number of jobs in the system as seen by the server can be very different from what the customers see upon arrival.
\begin{hint}
Consider a queueing system with constant service and inter-arrival times.
\end{hint}
\begin{solution}
 Take $X_k = 10$ and $S_k = 10-\epsilon$ for some tiny
 $\epsilon$. Then $L(t) = 1$ nearly all of the time. In fact,
 $\E L = 1-\epsilon/10$. However, $L(A_k-)=0$ for all $k$.
\end{solution}
\end{exercise}


\begin{extra}\clabel{ex:90}
 If $L(t)/t \to 0$ as $t\to\infty$, can it still be true that $\E{L}>0$? 
\begin{solution}
 \begin{equation*}
 \E{L} = \lim_{t\to\infty} \frac 1 t \int_0^t L(s) \d s \neq \lim_{t\to\infty} \frac{L(t)}t.
 \end{equation*}
If $L(t)=1$ for all $t$, $\E{L} =1 $, but $L(t)/t \to 0$. 
\end{solution}
\end{extra}


\begin{exercise}\clabel{ex:l-166}
Consider the discrete-time model of the queueing system specified by~\cref{eq:31}.
In such queueing systems, jobs arrive in batches, for instance, when $a_k=3$, three jobs arrive in slot $k$.
Assuming that $L_{k-1}=5$, what queue length have these 3 arrivals seen?  

Provide one definition similar to~\cref{eq:Qm} for the case in which we say that all arrivals see the same number in the system. Provide a second in which we like to express that the first of a batch of arrivals sees less in the system than the last arrival of a batch. 
\begin{hint}
Realize that when jobs arrive in batches, the definition of loss fraction requires some care; not all definitions need to measure the same.
\end{hint}
\begin{solution} 

Suppose that we don't want to distinguish between jobs in a batch, but simply want to say that if one job sees a long queue, all see a long queue.
In that case,
\begin{equation*}
\frac 1{A(n)}\sum_{k=1}^n a_k \1{L_k > m}.
\end{equation*}


For the second code, observe that, since we deal with a system in discrete time, $L_k$ is the queue length at the end of period~$k$.
Thus, $\sum_{k=1}^n \1{L_k > m}$ counts the number of \emph{periods} that the queue is larger than $m$.
This is of course not the same as the number of \emph{items} that see a queue larger than $m$; only when $a_k>0$ the items in a batch would see a queue $L_k>m$.
Thus,
\begin{equation*}
  \sum_{k=1}^n \1{L_k > m} \1{a_k > 0},
\end{equation*}
counts the number of batches. 

Next, by assumption, $a_k$ items arrive during period $k$.
The first of these items sees a queue length of $L_{k-1} - d_k$, the second $L_{k-1}-d_k + 1$, and so on, until the last item, which sees a queue length of $L_k-1 = L_{k-1} - d_k + a_k -1$.
Thus, of all items, the last item sees the largest queue.
Hence, if $L_k \leq m$, all items of the batch see a queue less than $m$.
If, however, $L_k > m$, then $L_k -m$ customers saw $m$ or more jobs in the system.
Therefore, the fraction of arrivals that see a queue with $m$ or more jobs is equal to
\begin{equation*}
 \frac 1 {A(n)} \sum_{k=1}^n (L_k - m) \1{L_k > m} .
\end{equation*}

Here is the code for the second case. 
\begin{pyconsole}
a = [0, 2, 5, 1, 2]
c = [0, 1, 1, 0, 0]

d = [0] * len(a)
L = [0] * len(a)

for k in range(1, len(a)):
    d[k] = min(L[k - 1], c[k])
    L[k] = L[k - 1] + a[k] - d[k]

print(L)

m = 5

res = 0
for k in range(1, len(a)):
    res += (L[k] - m) * (L[k] - m)

print(res, res / sum(a))
\end{pyconsole}

\end{solution}
\end{exercise}




\opt{solutionfiles}{
\Closesolutionfile{hint}
\Closesolutionfile{ans}
\subsection*{Hints}
\input{hint}
\subsection*{Solutions}
\input{ans}
}


%\clearpage
 



%%% Local Variables:
%%% mode: latex
%%% TeX-master: "../companion"
%%% End:
