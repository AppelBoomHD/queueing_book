\section{(Limits of) Empirical Performance Measures}
\label{sec:limits-of-emperical}

If the queueing system is rate-stable, we can sensibly define a number of long-run average performance measures such as the load, the average waiting time in queue, and so on. 
This we do here and refer to \cref{fig:constructiongg1} for an overview of the relations between these performance measures.


\opt{solutionfiles}{
\subsection*{Theory and Exercises}
\Opensolutionfile{hint}
\Opensolutionfile{ans}
}


Recall that we constructed a single-server queueing process $\{L(t)\}$ in cf.~\cref{sec:constr-gg1-queu} based on the arrival process $\{A(t)\}$ and service process $\{S(t)\}$.
Once we have the queueing process, we can compute the waiting time process $\{W_{Q,k}\}$, the sojourn time process $\{W_{k}\}$.


Define the \recall{expected sojourn time} as \emph{seen by  arrivals} as
\begin{equation}\label{eq:49}
 \E W = \lim_{n\to\infty} \frac 1n\sum_{k=1}^n W_k.
\end{equation}
Note that this performance measure is the limit of an \emph{empirical} performance measure as observed by arriving jobs: the first job experiencesa sojourn time $W_1$ when it arrives, the second a sojourn time $W_2$, and so on.
For this reason, we colloquially say such a performance measure is as `seen by arrivals'.
Similarly, the \recall{expected waiting time in queue} is defined as
\begin{equation}\label{eq:50}
 \E{W_Q} = \lim_{n\to\infty} \frac 1 n\sum_{k=1}^n W_{Q,k}.
\end{equation}
The \emph{distribution of the sojourn times} seen by arrivals can be found by counting:
\begin{equation}\label{eq:48}
 \P{W \leq x} = \lim_{n\to\infty} \frac 1n\sum_{k=1}^n \1{W_k\leq x}.
\end{equation}
The (sample) \recall{average number of jobs} in the system as seen by arrivals is given by
\begin{equation}\label{eq:EQ}
\E L = \lim_{n\to\infty}\frac 1 n \sum_{k=1}^n L(A_k-),
\end{equation}
since $L(A_k-)$ is the number of jobs in the system at the arrival epoch of the $k$th job.
Finally, the \emph{distribution of $\{L(t)\}$} as seen by arrivals, is given by
\begin{equation}\label{eq:Qm}
\P{L\leq m} = \lim_{n\to\infty} \frac 1 n \sum_{k=1}^n \1{L(A_k-) \leq m}.
\end{equation}



A related set of performance measures follows by tracking the system's behavior over time and taking the \emph{time-average}, rather than the average at arrival epochs.
% Thus, if we simulate the queueing system up to time $t$, 
% \begin{equation}\label{eq:11}
% \frac 1 t\int_0^t L(s)\d s = \frac 1 t\int_0^t (A(s)-D(s)) \d s,
% \end{equation}
% where we use that $L(s)=A(s) - D(s) + L(0)$ is the total number of jobs in
% the system at time $s$ and $L(0)=0$, cf.~\cref{fig:atltdt}. Observe from the second equation that $\int_0^t L(s)\d s$ is the area enclosed between the graphs of $\{A(t)\}$
% and $\{D(t)\}$. 
Assuming the limit exists we use~\cref{eq:14} to define the \recall{time-average number of jobs} as
\begin{equation}
 \label{eq:46}
 \E L = \lim_{t\to\infty} \frac 1 t\int_0^t L(s) \d s.
\end{equation}
Observe that, notwithstanding that the symbols are the same, this expectation is not necessarily the same as~\cref{eq:EQ}.
In a loose sense we can say that $\E L$ is the average number in the system as perceived by the \emph{server}.
Next, define the \emph{time-average fraction of time the system contains at most $m$ jobs} as
\begin{equation}
 \label{eq:47}
 \P{L\leq m} =\lim_{t\to\infty} \frac 1 t\int_0^t \1{L(s)\leq m} \d s.
\end{equation}
Again, this probability need not be the same as what customers see upon arrival.


\begin{exercise}\clabel{ex:l-165}
Design a queueing system to show that the average number of jobs in the system as seen by the server can be very different from what customers see upon arrival.
\begin{hint}
Consider a queueing system with constant service and inter-arrival times.
\end{hint}
\begin{solution}
 Take $X_k = 10$ and $S_k = 10-\epsilon$ for some tiny
 $\epsilon$. Then $L(t) = 1$ nearly all of the time. In fact,
 $\E L = 1-\epsilon/10$. However, $L(A_k-)=0$ for all $k$.
\end{solution}
\end{exercise}


\begin{extra}\clabel{ex:90}
 If $L(t)/t \to 0$ as $t\to\infty$, can it still be true that $\E{L}>0$? 
\begin{solution}
 \begin{equation*}
 \E{L} = \lim_{t\to\infty} \frac 1 t \int_0^t L(s) \d s \neq \lim_{t\to\infty} \frac{L(t)}t.
 \end{equation*}
If $L(t)=1$ for all $t$, $\E{L} =1 $, but $L(t)/t \to 0$. 
\end{solution}
\end{extra}

Taking the above limits requires considerable care, and at least two questions are important.
What type of limit do we actually mean here?
And, once such limits are given proper meaning, what is the rate of convergence of, for instance, the random variables $\{L_k\}$ to the limiting random variable~$L$?
Here we sidestep all such fundamental issues, but see~\cref{rem:em-1}.
Assuming that the first of these questions is answered, the limiting random variables are known as the \recall{steady-state} or \recall{stationary} limits, and the related distributions are often called \recall{limiting} or \emph{stationary distributions}.

Related to the second question about the convergence rate, we provide some intuition by means of an example.
We consider the sequence of waiting times $\{W_{Q,k}\}$ to a limiting random variable $W_Q$, where $W_{Q,k}$ is constructed according to the recursion~\cref{eq:56}.
Suppose that $X_k\sim U\{1,2,4\}$ and $S_k\sim U\{1,2,3\}$.
Starting with $W_{Q,0}=5$ we use~\cref{eq:56} to compute the \emph{exact} distribution of $W_{Q,k}$ for $k=1,2,\ldots, 20$, cf., the left panel in~\cref{fig:convergence}.
We see that when $k=5$, the `hump' of $\P{W_{Q,5}=x}$ around $x=5$ is due the starting value of $W_{Q,0}=5$.
However, for $k>10$ the distribution of $W_{Q,k}$ hardly changes, at least not visually.
Apparently, the convergence of the sequence of distributions of $W_{Q,k}$ is rather fast.

In the middle panel we show the results of a set of \emph{simulations} for increasing simulation length, up to $N=1000$ samples, where we use the \emph{empirical distribution} 
\begin{equation*}
\P{W_Q\leq x} = \frac 1n \sum_{k=1}^n \1{W_{Q,k} \leq x}
\end{equation*}
to estimate the waiting time distribution. 
As should be clear from the figure, the simulated distribution also converges quite fast to some limiting function.
Finally, in the right panel we compare the densities as obtained by the exact method and simulation with $n=1000$.
Clearly, for all practical purposes, these densities can be treated as the same.


The combination of the fast convergence to the steady-state situation and the difficulties with the transient analysis validates, to some extent, that most queueing theory is concerned with the analysis of the system in \emph{stationary} or \emph{steady state}.

\begin{figure}
 \centering
% see progs/waiting_time_simulation.py
\input{../progs/waiting_time_1.tex}
\input{../progs/waiting_time_2.tex}
\input{../progs/waiting_time_3.tex}
% \includegraphics{progs/gg1convergence}
 \caption{The density of $W_{Q,k}$ for $k=5, 10, 15, 20$ computed by
 an exact method as compared the density obtained by simulation of
 different run lengths $N=200, 400, \ldots, 1000$. The right panel
 compares the exact density of $W_{Q,20}$ to the density obtained by simulation
 for $N=1000$.}
\label{fig:convergence}
\end{figure}




\begin{exercise}\clabel{ex:l-135}
 Suppose that $X_k\in\{1,3\}$ such that $\P{X_k=1}=\P{X_k=3}$ and
 $S_k\in\{1,2\}$ with $\P{S_k=1}=\P{S_k=2}$. Write a computer program
 to see how fast the distributions of $W_{Q,k}$ converge to a limiting distribution function.
\begin{solution}
 Here is an example with Python.
 I compute the difference, i.e., the Kolmogorov-Smirnov statistic, between the distributions of $W_{Q,k-1}$ and $W_{Q,k}$,
\begin{equation*}
 \max_x\{ |\P{W_{Q,k}\leq x} - \P{W_{Q,k-1}\leq x}|\},
\end{equation*}
for $x$ in the support of $W_{Q,k}$. 

The code can be found in the \pyv{exact} function in the file \pyv{waiting_time_simulation.py} at
my github repo.

If you make a plot, you will see that after some 10 customers the distribution hardly changes any further. 

\end{solution}
\end{exercise}

% \begin{exercise}\clabel{ex:l-136}
%  Validate the results of~\cref{fig:convergence} with simulation.
% \begin{solution}
%  The code is in the file \pyv{waiting_time_simulation.py} at my github repo.
% \end{solution}
% \end{exercise}


Up to now, we considered queueing systems in continuous time.
However, the performance measures for discrete-time queueing models needs some modification, mainly for the reason that multiple jobs can arrive in a single period.
What statistics do these jobs assemble?
The next exercise focuses on this problem.

\begin{exercise}\clabel{ex:l-166} Consider the discrete-time model of the queueing system specified by~\cref{eq:31}.
In such queueing systems, jobs arrive in batches, for instance, when $a_k=3$, three jobs arrive in slot $k$.
Assuming that $L_{k-1}=5$, what queue length have these 3 arrivals seen?  

Provide a definition similar to~\cref{eq:Qm} for the case in which we say that all arrivals see the same number in the system.
Provide another definition to express that the first of a batch of arrivals sees less jobs in the system than the last arrival of a batch. 
\begin{hint}
Realize that when jobs arrive in batches, the definition of loss fraction requires some care; not all definitions need to measure the same.
\end{hint}
\begin{solution} 

Suppose that we don't want to distinguish between jobs in a batch, but simply want to say that if one job sees a long queue, all see a long queue.
In that case,
\begin{equation*}
\frac 1{A(n)}\sum_{k=1}^n a_k \1{L_k > m}.
\end{equation*}


For the second code, observe that, since we deal with a system in discrete time, $L_k$ is the queue length at the end of period~$k$.
Thus, $\sum_{k=1}^n \1{L_k > m}$ counts the number of \emph{periods} that the queue is larger than $m$.
This is of course not the same as the number of \emph{items} that see a queue larger than $m$; only when $a_k>0$ the items in a batch would see a queue $L_k>m$.
Thus,
\begin{equation*}
  \sum_{k=1}^n \1{L_k > m} \1{a_k > 0},
\end{equation*}
counts the number of batches. 

Next, by assumption, $a_k$ items arrive during period $k$.
The first of these items sees a queue length of $L_{k-1} - d_k$, the second $L_{k-1}-d_k + 1$, and so on, until the last item, which sees a queue length of $L_k-1 = L_{k-1} - d_k + a_k -1$.
Thus, of all items, the last item sees the largest queue.
Hence, if $L_k \leq m$, all items of the batch see a queue less than $m$.
If, however, $L_k > m$, then $L_k -m$ customers saw $m$ or more jobs in the system.
Therefore, the fraction of arrivals that see a queue with $m$ or more jobs is equal to
\begin{equation*}
 \frac 1 {A(n)} \sum_{k=1}^n (L_k - m) \1{L_k > m} .
\end{equation*}

Here is the code for the second case. 
\begin{pyconsole}
a = [0, 2, 5, 1, 2]
c = [0, 1, 1, 0, 0]

d = [0] * len(a)
L = [0] * len(a)

for k in range(1, len(a)):
    d[k] = min(L[k - 1], c[k])
    L[k] = L[k - 1] + a[k] - d[k]

print(L)

m = 5

res = 0
for k in range(1, len(a)):
    res += (L[k] - m) * (L[k] - m)

print(res, res / sum(a))
\end{pyconsole}

\end{solution}
\end{exercise}



\begin{remark}\label{rem:em-1}
The \emph{long-run limiting behavior} of a queueing system (i.e., the first question) is a subtle topic.
%The underlying question is what happens if we simulate the system for a long time.
For instance, does there exist a random variable $L$ such that $L_k\to L$ in some sense?
%The answer to this question is in the affirmative, provided some simple stability conditions are satisfied, see~\cref{sec:rate-stability}.
The answer to this question requires a considerable amount of mathematics.
To sketch what has to be done, realize that we first need to define $\{L_k\}$ as a stochastic process (up to now we just considered each $L_k$ as a \emph{number}, i.e., a measurement or simulation of the queue length time of the $k$th period.)
The construction of $L_k$ as a proper random variable is not as simple as the definition of the service times  $\{S_k\}$, for instance.
In the latter case, we just assume these random variables to be i.i.d.
However, in the former case, it is apparent from~\cref{eq:59} that the queue lengths $\{L_k\}$ are \emph{constructed} in terms of recursions, hence they are certainly \emph{not} i.i.d.
Next, based on these recursions, we need to show that the sequence of distribution functions $\{G_k\}$ associated with the random variables $\{L_k\}$ converges to some limiting distribution function~$G$, say.
Finally, it is necessary to show that it is possible to construct a random variable~$L$ that has~$G$ as its distribution function.
In this sense, finally,  we say that~$L_k \to L$. All this can be done, but we refer to the specialized literature for this. 
\end{remark}


\opt{solutionfiles}{
\Closesolutionfile{hint}
\Closesolutionfile{ans}
\subsection*{Hints}
\input{hint}
\subsection*{Solutions}
\input{ans}
}


%\clearpage
 



%%% Local Variables:
%%% mode: latex
%%% TeX-master: t
%%% End:
