% arara: pdflatex
% arara: clean: { extensions: [ aux, blg, idx, ilg, ind, log, out, pytxcode, rel, toc ] }
% !arara: clean: { files: [ ans.tex, hint.tex] }

\documentclass[companion]{subfiles}

\begin{document}

\section{Setups and Batch Processing}
\label{sec:setups-batch-proc}

In some cases machines have to be setup before they can start producing items.
Consider, for instance, a machine that paints red and blue bikes.
When the machine requires a color change, it may be necessary to clean up the machine.
Another example is an oven that needs a temperature change when different item types require different production temperatures.
Service operations form another setting with setup times: when servers (personnel) have to move from one part of a building to another, the time spent moving cannot be spent on serving customers.
In all such cases, the setups consume a significant amount of time; in fact, setup times of an hour or longer are not uncommon.
Clearly, in such situations, it is necessary to produce in batches: a server processes a batch of jobs (or customers) of one type or at one location, then the server changes from type or location, starts serving a batch of another type or at another location.
Once done with one type, the server is setup again, and so on.
Here we focus on the effect of change-over, or setup, times on the average sojourn time of jobs.

First we make a model and provide a list of elements required to compute the expected sojourn time of an item, then we illustrate how to use these elements in a concrete case.

\opt{solutionfiles}{
\subsection*{Theory and Exercises}
\Opensolutionfile{hint}
\Opensolutionfile{ans}
}



Specifically, we analyze the following batch queueing situation.
There are two job families, e.g., red and blue, each served by the same single server.
Jobs arrive at rate $\lambda_r$ and $\lambda_b$, respectively, so that the arrival rate of jobs is $\lambda= \lambda_b+\lambda_r$.
Jobs of both types require an average \recall{net processing time} of $\E{S_0}$, provided the server is already setup for the correct job color.
The setup times $\{R_i\}$ are assumed to form an i.i.d.
sequence with common random variable $R$ and independent of $S_0$.
The sojourn time comprises the following steps.
First, jobs of each color are assembled into batches of size $B$, which we assume to be the same for both colors.
Once a batch is complete, the batch enters a queue (of batches).
After some time the batch reaches the head of the queue.
Then the machine performs a setup, and starts processing each job individually until the batch is complete.
Finally, once a job is finished, it can leave the system; as a consequence, it does not have to wait for other jobs in the same batch to form a new batch.




\begin{exercise}\clabel{ex:48}
  Show that the average time a job has to wait to fill the batch (to which this jobs belongs) is given by
\begin{equation}\label{eq:79}
 \E{W_r} = \frac{B-1}{2\lambda_r}.
\end{equation}
\begin{hint}
 Show that the total time to form a red batch is $(B-1)/\lambda_r$. 
\end{hint}
\begin{solution}
 Suppose a batch is just finished. The first job of a new batch needs to wait, on average, $B-1$ inter-arrival times until the batch is complete, the second $B-2$ inter-arrival times, and so on. The last job does not have to wait at all. Thus, the total time to form a batch is $(B-1)/\lambda_r$. 

An arbitrary job can be anywhere in the batch, hence the average time a job must wait until the batch is complete is half the total time. 
\end{solution}
\end{exercise}

Now that we have a batch of jobs, we need to estimate the average time a batch spends in queue.
For this we can use the $G/G/1$ waiting time formula, but we have to convert the effects of the setup times into job service times.
Define, to this end, the \recall{effective processing time} $\E S$ as the average time the server is occupied with processing a job from a batch including the setup time required to setup the batch.

\begin{exercise}\clabel{ex:488}
  Motivate that the effective processing time of an \emph{item} should be defined as
\begin{equation}\label{eq:81}
 \E{S} = \E{S_0} + \frac{\E{R}} B.
\end{equation}
\begin{hint}
 What fraction of the setup time $\E R$ `belongs' to one job?
\end{hint}
\begin{solution}
  The total service time spent on a batch of size $B$ is $B \E{S_0} + \E R$.
  The effective time per job is then the average, i.e., $(B \E{S_0}+\E R)/B$.
  Simplifying gives the answer.
\end{solution}
\end{exercise}

With the previous exercise, the load becomes
\begin{equation*}
\rho = \lambda \left(\E{S_0} + \frac{\E{R}}B\right).
\end{equation*}
There is another important way to look at the load, see the next exercise.

\begin{exercise}\clabel{ex:489}
Explain that the load can also be written as
\begin{equation*}
\rho = \lambda_B (B \E{S_0} + \E R),
\end{equation*}
where $\lambda_B=\lambda/B$ is the arrival rate of \emph{batches} and $\E{S_B} = B \E{S_0} + \E R$ is the service time an entire batch.
\begin{solution}
It is evident that the rate at which batches arrive is 
\begin{equation*}
 \lambda_B = \frac \lambda B,
\end{equation*}
since both job colors have the same batch size. 
Then the equality has the interpretation of the batch arrival rate times the work per batch.
\end{solution}
\end{exercise}

\begin{exercise}\clabel{ex:l-137}
Show that the requirement $\rho < 1$ leads to the following constraint on the minimal batch size~$B$ 
 \begin{equation*}
 B>\frac{\lambda \E R}{1-\lambda \E{S_0}}.
 \end{equation*}
\begin{solution}
 We require that the load is less than one, in other words, that $\lambda\left(\E{S_0} + {\E{R}}/B\right) < 1$. This is equivalent to $\E{R}/B < 1/\lambda - \E{S_0}$, hence $B > \E{R}/(1/\lambda - \E{S_0})$. Multiplying with $\lambda$ gives the result. 
\end{solution}
\end{exercise}

Observe that we nearly have all elements to apply Sakasegawa's formula to the batch: the service time of a \emph{batch} is
\begin{equation}
  \label{eq:90}
\E{S_B} = \E R + B\E{S_0},
\end{equation}
and the load $\rho$ is given above.
Therefore it remains to find $C_{a,B}^2$ and $C_{s,B}^2$, for the \emph{batches, not the items}.

\begin{exercise}\clabel{ex:490}
Explain that the SCV of the batch inter-arrival times is given by
 \begin{equation}\label{eq:82}
C_{a,B}^2 = \frac{C_{a}^2}B.
\end{equation}
\begin{hint}
Use \cref{eq:95} and \cref{eq:94}.
\end{hint}
\begin{solution}
The variance of the inter-arrival time of batches is $B$ times the variance of job inter-arrival times. The inter-arrival times of batches is also $B$ times the inter-arrival times of jobs. Thus, 
\begin{equation*}
 C_{a,B}^2 = \frac{B \V{X}}{(B \E X)^2} = \frac{\V X}{(\E X)^2} \frac 1 B = \frac{C_a^2}{B}.
\end{equation*}
\end{solution}
\end{exercise}


\begin{exercise}\clabel{ex:491}
Show that the SCV $C_{s,B}^2$ of the service times of the batches takes the form
\begin{equation}\label{eq:84}
C_{s, B}^2 = \frac{B \V{S_0} + \V{R}}{(B \E{S_0} + \E R)^2}.
\end{equation}
\begin{hint}
 What is the variance of a batch service time?
\end{hint}
\begin{solution}
 The variance of a batch is $\V{\sum_{i=1}^B S_{0,i} + R} = B\V{S_0} + \V R$, since the normal service times $S_{0,i}, i=1,\ldots, B$ of the job are independent, and also independent of the setup time $R$ of the batch.
\end{solution}
\end{exercise}

Observe that we now have all elements to fill in Sakasegawa's formula, which becomes
\begin{equation*}
  \E{W_{Q,B}} = \frac{C_{a,B}^2 + C_{s,B}^2}2 \frac \rho{1-\rho} \E{S_B}.
\end{equation*}

It is left to find a rule to determine what happens to an item once the batch to which the item belongs enters service.
If the job has to wait until all jobs in the batch are served, the expected time an item spends at the server is $\E R + B \E{S_0}$.


\begin{exercise}\clabel{ex:492}
Show that, when items can leave right after being served, the time at the server is given by
\begin{equation}\label{eq:85}
\E{R} + \frac{B+1}{2}\E{S_0}.
\end{equation}
\begin{solution}
 First, wait until the setup is finished, then wait (on average) for half of the batch (minus the job itself) to be served, and then the job has to be served itself, that is,
$\E{R} + \frac{B-1}{2}\E{S_0} +\E{S_0}$.
\end{solution}
\end{exercise}


\begin{exercise}\clabel{ex:l-139}
 Jobs arrive at $\lambda=3$ per hour at a machine with $C_a^2=1$; service times are exponential with an average of 15 minutes. Assume $\lambda_r = 0.5$ per hour, hence $\lambda_b = 3-0.5=2.5$ per hour. Between any two batches, the machine requires a cleanup of 2 hours, with a standard deviation of $1$ hour, during which it is unavailable for service.
 What is the smallest batch size that can be allowed?

 What is the average time a red job spends in the system in case $B=30$ jobs?
 Finally, observe that there is $B$ that minimizes the average sojourn time.
\begin{solution}
First check the load.
\begin{pyconsole}
labda = 3 # per hour
ES0 = 15./60 # hour
ES0
ER = 2.
B = 30
ESe = ES0+ ER/B
ESe

rho = labda*ESe
rho
\end{pyconsole}
Evidently, the load is smaller than $1$. 

The minimal batch size is
\begin{pyconsole}
Bmin = labda*ER/(1-labda*ES0)
Bmin
\end{pyconsole}
So, with $B=30$ we are on the safe side. 

The time to form a red batch is 
\begin{pyconsole}
labda_r = 0.5
EW_r = (B-1)/(2*labda_r)
EW_r # in hours
\end{pyconsole}
And the time to form a blue batch is 
\begin{pyconsole}
labda_b = labda-labda_r
EW_b = (B-1)/(2*labda_b)
EW_b # in hours
\end{pyconsole}


Now the time a batch spends in queue
\begin{pyconsole}
Cae = 1.
CaB = Cae/B
CaB
Ce = 1 # SCV of service times
VS0 = Ce*ES0*ES0
VS0
VR = 1*1. # Var setups is sigma squared
VSe = B*VS0 + VR
VSe
ESb = B*ES0+ER
ESb
CeB = VSe/(ESb*ESb)
CeB
EWq = (CaB+CeB)/2 * rho/(1-rho) * ESb
EWq
\end{pyconsole}

The time to unpack the batch, i.e., the time at the server. 
\begin{pyconsole}
ES = ER + (B-1)/2 * ES0 + ES0 
ES
\end{pyconsole}


The overall time red jobs spend in the system.
\begin{pyconsole}
total = EW_r + EWq + ES
total
\end{pyconsole}

\end{solution}
\end{exercise}


\begin{exercise}\clabel{ex:l-138}
  What important insights can you learn from the above about setting proper batch sizes?
\begin{solution}
  We can derive a number of important insights from this model.
  First, the time to assemble and unpack batches increases linearly of $B$.
  Second, observe from~\cref{eq:82} and \cref{eq:84} that the SCVs are $B$ times as small, but in~\cref{eq:90} we see that the batch service time is roughly $B$ times larger.
  Thus, the factor $B$ approximately cancels out in Sakasegawa's formula.
  However, the load decreases as a function of $B$, and this is the most important reason to use batch production.
  Third, since the load decreases as a function of the $B$, the expected time in queue must also decrease.
  Combining this with the linear batch forming times, there must be a batch size that minimizes the total time in the system.
  Fourth, with a it of numerical experimentation will show that the minimum in the waiting time is rather insensitive to $B$.
  However, by making $B$ smaller, the load increases, and we know that queueing times increase very steeply when the load increases.
  Therefore, by making $B$ small, we `gain' time linearly in batch forming times, but we `pay' hyperbolically in $B$.
  Given the randomness in practical situations, it is better, when tuning the batch size to minimize the time in the system, to set $B$ a bit too large than too small.
\end{solution}
\end{exercise}

A much more interesting and realistic problem is to assume that we have many families such that items of family~$i$ arrive at rate $\lambda_i$ and require service time $S_i$.
Typically, the setup time depends on the sequence in which the families are produced; let this be given by $R_{ij}$ when the machine switched from producing family~$i$ to family~$j$.
Often, in practice, $R_{ij} \neq R_{ji}$, for example, a switch in color from white to black takes less cleaning time and cost than from black to white.
Then the problem becomes to determine a good schedule in which to produce the families and the batch size $B_i$, which may be depend on the family.
Here is an exercise to show how to handle a simple case.

\begin{exercise}\clabel{ex:l-254}
 Consider a paint factory that contains a paint mixing machine that serves two classes of jobs, A and B.
 The processing times of jobs of types A and B are constant and require $t_A$ and $t_B$ hours.
 The job arrival rate is $\lambda_A$ for type A and $\lambda_B$ for type $B$ jobs.
 It takes a setup time of $S$ hours to clean the mixing station when changing from paint type A to type B, and there is no time required to change from type B to A.

 To keep the system  stable, it is necessary to produce the jobs in batches, for otherwise the server, i.e., the mixing machine, spends a too large fraction of time on setups.
 Motivate that the following linear program can be used to determine the minimal batch sizes:
\begin{equation*}
 \text{minimize } T
\end{equation*}
such that $ T= k_A t_A + S + k_B t_B$, $\lambda_A T < k_A$ and $\lambda_B T < k_B$.
\begin{hint}
Here are some questions to help you interpret this formulation.
\begin{enumerate}
\item What are the decision variables for this problem? In other words, what are the `things' we can control/change?
\item What are the interpretations of $k_A t_A$, and $S+k_B t_B$?
\item What is the meaning of the first constraint? Realize that $T$
 represents one production cycle. After the completion of one such
 cycle, we start another cycle. Hence, the start of every cycle can
 be seen as a restart of the entire system.
\item What is the meaning of the other two constraints?
\item Why do we minimize the cycle time $T$?
\item Solve for $k_A$ and $k_B$ in terms of $S$, $\lambda_A, \lambda_B$ and $t_A, t_B$. 
\item Generalize this to $m$ job classes and such that the cleaning
 time between jobs of class $i$ and $j$ is given by $S_{i j}$. (Thus,
 the setup times are sequence-dependent.) 
\end{enumerate}
\end{hint}

\begin{solution}
 Realize that the machine works in cycles. A cycle starts with
 processing $k_A$ jobs of type A, then does a setup, and processes
 $k_B$ jobs of type B, and then a new cycle starts again. The time
 it takes to complete one such cycle is $T=k_A t_A + S + k_B t_B$.
 The number of jobs of type A processed during one such cycle is,
 of course, $k_A$. Observe next that the average number of jobs
 that arrive during one cycle is $\lambda_A T$. We of course want
 that $\lambda_A T< k_A$, i.e., fewer jobs of type A arrive on
 average per cycle than what we can process.
\end{solution}
\end{exercise}



\opt{solutionfiles}{
\Closesolutionfile{hint}
\Closesolutionfile{ans}
\subsection*{Hints}
\input{hint}
\subsection*{Solutions}
\input{ans}
}
\end{document}

%\clearpage

%%% Local Variables:
%%% mode: latex
%%% TeX-master: t
%%% End:
