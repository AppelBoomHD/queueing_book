\section{Preliminaries}
\label{sec:preliminaries}


\opt{solutionfiles}{
\subsection*{Theory and Exercises}
\Opensolutionfile{hint}
\Opensolutionfile{ans}
}


Here is an overview of concepts you are supposed to have seen in earlier courses.
We will use these concepts over and over in the rest of the course.


We use the notation:
 \begin{align*}
 [x]^+ &= \max\{x, 0\}, \\
 f(x-) &= \lim_{y\uparrow x} f(y),\\
 f(x+) &= \lim_{y\downarrow x} f(y),\\
 \1{A} &=
 \begin{cases}
 1, & \text{ if $A$ is true}, \\
 0, & \text{ if $A$ is false}.
 \end{cases}, 
\end{align*}
where the last equation defines an \recall{indicator variable}.


The function $f(h)=o(h)$ means that $f$ is such that $f(h)/h \to 0$ as $h\to 0$.
If we write $f(h) = o(h)$ it is implicit that $|h| \ll 1$.
We call this \recall{small $o$ notation}.
\begin{exercise}\clabel{ex:l-104}
 Let $c$ be a constant (in $\R$) and the functions $f$ and $g$ both of $o(h)$. Then show that (1) $f(h) \to 0$ when $h\to 0$, (2) $c\cdot f = o(h)$, (3) $f+g=o(h)$, and (4) $f\cdot g=o(h)$. 
\begin{solution}
 In fact (1) is trivial: $|f(h)| \leq |f(h)/h|$ when $|h| < 1$.
 But it is given that the right-hand side goes to zero.
 For (2) and (3):
\begin{align*}
\lim_{h\to 0} \frac{c f(h)}{h} &= c \lim_{h\to 0} \frac{f(h)}{h} = 0, \; \text{as } f = o(h), \\
\lim_{h\to 0} \frac{f(h) + g(h)} h &= \lim_{h\to 0} \frac{f(h)} h + \lim_{h\to 0} \frac{g(h)} h = 0.
\end{align*}
For (4), use the Algebraic Limit Theorem,
\begin{align*}
\lim_{h\to 0} \frac{f(h)g(h)}{h} &= \lim_{h\to 0} h \frac{f(h)}{h} \frac{g(h)}{h} \\
&= \lim_{h\to 0} h \lim_{h\to 0} \frac{f(h)}{h} \lim_{h\to 0} \frac{g(h)}{h} \\
&= 0 \cdot 0 \cdot 0 = 0.
\end{align*}
\end{solution}
\end{exercise}


You should know that:
\begin{subequations}
 \begin{align}
 (a+b)^n &= \sum_{i=0}^n {n \choose i} a^{n-i} b^i, \label{eq:71}\\
e^x &= \lim_{n\to\infty} (1+x/n)^n, \label{eq:65}\\
 e^x &= 1 + x + \frac{x^2}{2!} + \frac{x^3}{3!} + \cdots = \sum_{k=0}^{\infty} \frac{x^k}{k!}, \label{eq:76}\\
 \sum_{n=0}^N \alpha^n &= \frac{1-\alpha^{N+1}}{1-\alpha}. \label{eq:61}
\end{align}
\end{subequations}

\begin{extra}\clabel{ex:87}
 Why is $e^{x} = 1 +x + o(x)$?
\begin{solution}
 When $|x|\ll 1$, the terms with $n\geq 2$ in~\cref{eq:76} are $x^n = o(x)$. Then applying $x^n + x^m = o(x)$ to the Taylor series gives the result.
\end{solution}
\end{extra}


You should know that for a non-negative, integer-valued random variable $X$ with \recall{probability mass function} $f(k)= \P{X = k} = f(k)$, 
\begin{subequations}
\begin{align}
X&=\sum_{n=0}^\infty X\1{X=n} = \sum_{n=0}^\infty n \1{X=n}, \label{eq:77} \\
\E X &= \sum_{n=0}^\infty n f(n), \\
\E{g(X)} &= \sum_{n=0}^\infty g(n) f(n), \label{eq:66}\\
\E{\1{X\leq x}} &= \P{X\leq x}, \label{eq:74}\\
\V X &= \E{X^2} - (\E X)^2.\label{eq:68}
\end{align}
\end{subequations}

\begin{extra}
 Why is~\cref{eq:77} true?
\begin{solution}
To see~\cref{eq:77}, note first that $X\1{X=n} = n\1{X=n}$ because $X=n$ when $\1{X=n} =1$, and second that $\sum_{n=0}^\infty \1{X=n} =1$, since $X$ takes one of the values in $\N$, and events $\{X=n\}$ and $\{X=m\}$ are non-overlapping when $n\neq m$. 
\end{solution}
\end{extra}

\begin{exercise}\clabel{ex:l-105}
Define the \recall{survivor function} of $X$ as $G(k) = \P{X>k}$. Show that
\begin{equation*}
 G(k) = \sum_{m=0}^\infty \1{m>k} f(m).
\end{equation*}
As you will see below, this idea makes the computation of certain expressions quite a bit easier.
\begin{solution}
 This is just rewriting the definition:
\begin{equation*}
G(k) = \P{X>k} = \sum_{m=k+1}^\infty \P{X=m} = \sum_{m=k+1}^\infty f(m)=\sum_{m=0}^\infty \1{m>k}f(m).
\end{equation*}
\end{solution}
\end{exercise}

\begin{exercise}\clabel{ex:l-106}
 Express the probability mass $f(k)$ and the survivor function $G(k)$ in terms of the \recall{distribution function} $F(k)=\P{X\leq k}$ of $X$.
\begin{hint}
This exercise is just meant to become familiar with the notation.
\end{hint}
\begin{solution}
 \begin{align*}
 f(k) &= \P{X=k} = \P{X\leq k} - \P{X\leq k-1} = F(k)-F(k-1), \\
 G(k) &= \P{X>k} = 1 - \P{X\leq k} = 1-F(k). 
 \end{align*}
\end{solution}
\end{exercise}

\begin{extra}
 Which of the following is true: $G(k) = 1-F(k)$, $G(k) = 1-F(k-1)$, or $G(k) = 1-F(k+1)$?
\begin{solution}
 $G(k) = 1- F(k) = 1-\P{X\leq k} = \P{X>k}$. 
 It is all too easy to make, so called, off-by-one errors, such as
 in the three alternatives above. I nearly always check simple
 cases to prevent such simple mistakes. I advise you to acquire the
 same habit.
\end{solution}
\end{extra}


\begin{extra}\clabel{ex:6}
 Use indicator functions to prove that $\E X = \sum_{k=0}^\infty G(k)$.
\begin{hint}
Write 
$\sum_{k=0}^\infty G(k) = \sum_{k=0}^\infty \sum_{m=k+1}^\infty \P{X=m}$, reverse the summations. Then realize that $\sum_{k=0}^\infty \1{k<m} = m$. 
You should be aware that this sort of problem is just a regular probability
 theory problem, nothing fancy. We use/adapt the tools you learned in
 calculus to carry out 2D integrals (or in this case 2D summations).
\end{hint}
\begin{solution}
Observe first that $\sum_{k=0}^\infty \1{m>k} = m$, since $\1{m>k}=1$ if $k<m$ and $\1{m>k} = 0$ if $k\geq m$. With this, 
\begin{align*}
\sum_{k=0}^\infty G(k) 
&= \sum_{k=0}^\infty \P{X>k} 
= \sum_{k=0}^\infty \sum_{m=k+1}^\infty \P{X=m} \\
& = \sum_{k=0}^\infty \sum_{m=0}^\infty \1{m>k} \P{X=m} 
= \sum_{m=0}^\infty \sum_{k=0}^\infty \1{m>k} \P{X=m} \\
&= \sum_{m=0}^\infty m\P{X=m} = \E X.
\end{align*}
In case you are interested in mathematical justifications: the interchange of the two summations is allowed by Tonelli's theorem because the summands are all positive.
(Interchanging the order of summations or integration is not always allowed because the results can be different when part of the integrand is negative.
Check Fubini's theorem for more on this if you are interested.)
\end{solution}
\end{extra}

\begin{exercise}\clabel{ex:66}
 Use indicator functions to prove that
$\sum_{i=0}^\infty i G(i) = \E{X^2}/2 - \E{X}/2.$
\begin{hint}
$\sum_{i=0}^\infty i G(i) = \sum_{n=0}^\infty \P{X=n} \sum_{i=0}^\infty i \1{n\geq i+1}$,
and reverse the summations.
\end{hint}
\begin{solution}
\begin{align*}
\sum_{i=0}^\infty i G(i)
&= \sum_{i=0}^\infty i \sum_{n=i+1}^\infty \P{X=n} = \sum_{n=0}^\infty \P{X=n} \sum_{i=0}^\infty i \1{n\geq i+1} \\
&= \sum_{n=0}^\infty \P{X=n} \sum_{i=0}^{n-1}i = \sum_{n=0}^\infty \P{X=n} \frac{(n-1)n}{2} \\
&= \sum_{n=0}^\infty \frac{n^2}{2} \P{X=n} - \frac{\E X}{2}
= \frac{\E{X^2}}{2} - \frac{\E X}{2}.
\end{align*}
\end{solution}
\end{exercise}



Let $X$ be a continuous non-negative random variable with distribution function $F$. We write 
\begin{equation*}
 \E{X} = \int_0^\infty x \d F(x)
\end{equation*}
for the expectation of $X$. Here $\d F(x)$ acts as a shorthand for $f(x) \d x$\footnote{For the interested reader, $\int x \d F(x)$ is a Lebesgue-Stieltjes integral with respect to the distribution function $F$.}. Recall that
\begin{align*}
\E{g(X)} &= \int_0^\infty g(x) \d F(x).
\end{align*}



\begin{exercise}\clabel{ex:l-107}
 Use indicator functions to prove that 
$ \E X = \int_0^\infty x \d F(x) = \int_0^\infty G(y) \d y,$
where $G(x) = 1 - F(x)$. 
\begin{hint}
$\E X = \int_0^\infty x \d F(x) = \int_0^\infty \int_0^\infty \1{y\leq x} \d y \d F(x)$.
\end{hint}
\begin{solution}
 \begin{align*}
 \E{X} &= \int_0^\infty x \d F(x) = \int_0^\infty \int_0^x \d y \d F(x) \\
 & = \int_0^\infty \int_0^\infty \1{y\leq x} \d y \d F(x) = \int_0^\infty \int_0^\infty \1{y\leq x} \d F(x) \d y\\
 & = \int_0^\infty \int_y^\infty \d F(x) \d y = \int_0^\infty G(y) \d y.
 \end{align*}
\end{solution}
\end{exercise}

You should be able to use indicator functions and integration by parts to show that $\E{X^2} = 2\int_0^\infty y G(y) \d y$, where $G(x) = 1- F(x)$, provided the second moment exists.

\begin{extra}
 Use indicator functions to prove that for a continuous non-negative random variable~$X$ with distribution function $F$, $ \E{X^2} = \int_0^\infty x^2 \d F(x) = 2 \int_0^\infty y G(y) \d y,$ where $G(x) = 1 - F(x)$.
\begin{hint}
$\int_0^\infty y G(y) \d y = \int_0^\infty y \int_0^\infty \1{y\leq x}f(x)\, \d x \d y$.
\end{hint}
\begin{solution}
 \begin{align*}
\int_0^\infty y G(y) \d y 
&= \int_0^\infty y \int_y^\infty f(x)\, \d x \d y = \int_0^\infty y \int_0^\infty \1{y\leq x}f(x)\, \d x \d y\\
&= \int_0^\infty f(x) \int_0^\infty y \1{y \leq x}\, \d y \d x
= \int_0^\infty f(x) \int_0^x y\, \d y \d x\\
&= \int_0^\infty f(x) \frac{x^2}2 \d x =\frac{\E{X^2}}2.
 \end{align*}
\end{solution}
\end{extra}

\begin{exercise}\clabel{ex:l-108}
 Show that $\E{X^2}/2 = \int_0^\infty y G(y) \d y$ for a continuous non-negative random variable $X$ with survivor function $G$.
 \begin{solution}
 \begin{hint}
 Use integration by parts. 
 \end{hint}
 \begin{equation}
 \int_0^\infty y G(y) \d y 
= \frac{y^2}2 G(y) \bigg|_0^\infty - \int_0^\infty \frac{y^2}2 g(y)\d y = \int_0^\infty \frac{y^2}2 f(y)\d y = \frac{\E{X^2}}2,
 \end{equation}
 since $g(y) = G'(y) = - F'(y) = - f(y)$. Note that we used $\frac{y^2}2 G(y) \bigg|_0^\infty = 0 - 0 = 0$, which follows from our assumption that $\E{X^2}$ exists, implying that $\lim_{y \to \infty} y^2G(y) = 0$.
\end{solution}
\end{exercise}


You should know that for the \recall{moment-generating function} $M_X(s)$ of a random variable~$X$ and~$s\in \mathbb{R}$ sufficiently small is defined as: 
\begin{subequations}
\begin{align}
 M_X(s) &= \E{e^{s X}}, \\
 M_X(s) & \text{ uniquely characterizes the distribution of $X$}, \label{eq:75}\\
 \E{X} &= M_{X}'(0) = \left.\frac{\d M_{X}(s)}{d s}\right|_{s=0},\label{eq:69}\\
\E{X^2} &= M_{X}''(0), \label{eq:64}\\
M_{X+Y}(s) &= M_X(s)\cdot M_Y(s), \quad \text{ if $X$ and $Y$ are independent}. \label{eq:73}
\end{align}
\end{subequations}

\begin{extra}
 What is the value of $M_X(0)?$
\begin{solution}
 $M_X(0) = \E{e^{0 X}} = \E{e^0} = \E{1} = 1.$
\end{solution}
\end{extra}

To help you recall the concept of \recall{conditional probability} consider the following question.
\begin{exercise}\clabel{ex:l-109}
 We have one gift to give to one out of three children. As we cannot
 divide the gift into parts, we decide to let `fate decide'. That
 is, we choose a random number in the set $\{1, 2, 3\}$. The first
 child that guesses this number wins the gift. Show that the
 probability of winning the gift is the same for each child.
\begin{hint}
 For the second child, condition on the event that the first does not choose the right number.
 Use the definition of conditional probability:
 $\P{A|B} = \P{AB}/\P{B}$ provided $\P{B}>0$.
\end{hint}
\begin{solution}
 The probability that the first child to guess also wins is
 $1/3$. What is the probability for child number two? Well, for
 him/her to win, it is necessary that child one does not win and
 that child two guesses the right number of the remaining
 numbers. Assume, without loss of generality that child 1 chooses
 $3$ and that this is not the right number. Then 
 \begin{equation*}
 \begin{split}
&\P{\text{Child 2 wins}} \\
&= \P{\text{Child 2 guesses the right number and child 1 does not win}} \\
&= \P{\text{Child 2 guesses the right number} \given \text{ child 1 does not win}}
\cdot \P{\text{Child 1 does not win}} \\
&= \P{\text{Child 2 makes the right guess in the set $\{1,2\}$}}\cdot \frac 23 \\
&= \frac 1 2\cdot \frac 23 = \frac 1 3.
 \end{split}
 \end{equation*}
 Similar conditional reasoning gives that child 3 wins with probability $1/3$. 
\end{solution}
\end{exercise}

You should know that:
\begin{subequations}
\begin{align}
\P{A\given B} & = \frac{\P{AB}}{\P{B}}, \quad\text{ if } \P{B}>0, \\
 \P{A} &= \sum_{i=1}^n \P{A B_i} = \sum_{i=1}^n \P{A\given B_i} \P{B_i}, \quad\text{ if $A=\bigcup_{i=1}^n B_i$ and $\P{B_i>0}$ for all $i$}. \label{eq:70}
\end{align}
\end{subequations}


\opt{solutionfiles}{
\Closesolutionfile{hint}
\Closesolutionfile{ans}
\subsection*{Hints}
\input{hint}
\subsection*{Solutions}
\input{ans}
}


%%% Local Variables:
%%% mode: latex
%%% TeX-master: "../companion"
%%% End:
