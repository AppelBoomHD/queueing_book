
\section{Gordon-Newell Networks}
\label{sec:gordonNewell}


\subsection*{Theory and Exercises}

\Opensolutionfile{hint}
\Opensolutionfile{ans}


The formula with the visit ratios should be like this:
\begin{equation*}
  V_k = \sum_{j=0}^M V_j P_{jk}, 
\end{equation*}
i.e., the sum should start at index 0. This is to include the
load/unload station.

Also, assume that the load/unload station has just one server.

% It is perhaps the easiest to start with studying the MVA algorithm,
% then the MDA algorithm and finish with the convolution algorithm.

You should realize that the algorithms discussed in this section are
meant to be carried out by computers. Thus the results will be
numerical, not in terms of formulas.

Mind the order of $V$ and $P$ in the computation of the visit ratios:
do not mix up $VP=V$ with $P V=V$, as in general, $VP \neq P V$.  We use
$VP=V$.


\begin{exercise}\label{ex:mva}
  Compute the visit ratios for a network with three stations such that all jobs from station 0 move to station 1, from station 1 all move to station 2, and from station 2 half of the jobs move to station 0 and the other half to station 1. 
  \begin{solution}
    The routing matrix $P$ is
    \begin{equation*}
      P = 
      \begin{pmatrix}
        0 & 1 & 0 \\
0& 0 & 1 \\
1/2 & 1/2 & 0
      \end{pmatrix}.
    \end{equation*}
    Solving $V=VP$ leads to
    $(V_0, V_1, V_2) = (V_2/2, V_0 + V_2/2, V_1)$. Thus, from the last
    item, $V_2 = V_1$, and from the first $V_0 = V_2/2$. Since
    $V_0=1$, it follows that $V_2 = 2V_0$ and $V_1=V_2=2 V_0$.
  \end{solution}
\end{exercise}

\begin{comment}
\begin{exercise}
  What if $P$ has an eigenvalue $\alpha$ whose modulus $|\alpha|>$? What if for all eigenvalues $\alpha_i$ we have that $|\alpha_i|<1$?
  \begin{solution}
    TBD.
  \end{solution}
\end{exercise}
\end{comment}

\begin{exercise}
Zijm.Ex.3.1.1 
\begin{solution}
If a part would need refitting or repositioning at the load/unload station, the part would visit the load/unload station more than once during its stay in the network. The visit ratio of the load/unload station can then no longer be set to 1.
\end{solution}
\end{exercise}

\begin{exercise}
Zijm.Ex.3.1.2
\begin{solution}
Let's number the states from 1 to 4. If station 1 feeds into station 2, and so on, and station 4 into station 1, then 
\begin{equation*}
  P = 
  \begin{pmatrix}
    0 & 1 & 0 & 0\\
    0 & 0 & 1 & 0\\
    0 & 0 & 0 & 1\\
    1 & 0 & 0 & 0\\
  \end{pmatrix}.
\end{equation*}
The visit ratios are, of course, $V_1= V_2 = V_3 = V_4$. 
\end{solution}
\end{exercise}

\begin{exercise}
Zijm.Ex.3.1.3
\begin{solution}
We number  station a as 1, and so on.
\begin{equation*}
  P = 
  \begin{pmatrix}
     0 & 1 & 0 & 0 & 0 &0 \\
     1/5 & 0 & 4/5 & 0 & 0 & 0\\
     0& 0 & 0 & 1/3 & 2/3 & 0 \\
     0 & 0 & 0 & 0 & 0 & 1\\
     0 & 0 & 0 & 0 & 0 & 1\\
     1 & 0 & 0 & 0 & 0 & 0\\
  \end{pmatrix}.
\end{equation*}
We now solve for $V$ in the equation $V = VP$. Since, by
definition, $V_1=1$, it suffices to express the other visit ratios in
terms of $V_1$. From elementary linear algebra, the second column
implies that $V_2 = V_1$, and the third that $V_3 = 4/5 V_2$, so that
$V_3 = 4/5 V_1$. From the fourth column $V_4 = 1/3 V_3$, thus,
$V_4 = 1/3 \cdot 4/5 V_1 = 4/15 V_1$.  From the fifth column
$V_5 = 2/3 V_4$, hence $V_5 = 8/45 V_1$. Finally, from the sixth
column, $V_6 = V_4+V_5$, hence $V_6 = (4/15 + 8/45)V_1$.

For later courses on Markov chains, it is important to note the
following. Write the visit ratio equation $V = VP$ as $V(1-P)=0$ where
$1$ is the indicator matrix. Clearly, $V$ is must be a left
eigenvector of the matrix $1-P$ with eigenvalue 0 (recall that
$V(1-P) = 0 = 0\cdot V$). Thus, at least one row or column of the matrix $1-P$ is superfluous to solve for $V$. 
\end{solution}
\end{exercise}

\begin{exercise}
  Relate Zijm.Eq.3.3 to the form of the steady-state distribution of
  the number of jobs in an $M/M/c$ queue.
  \begin{solution}
    Except for the normalization constant, the expressions are the
    same as equations 1.36 and 1.37 of Zijm's book.
  \end{solution}
\end{exercise}

\begin{exercise}
Zijm.Ex.3.1.4
\begin{solution}
Consider network one with routing $1\to3\to4\to2\to1$, each transition occurring with probability one, i.e.,
\begin{equation*}
  P =
  \begin{pmatrix}
    0 & 0 & 1 & 0  \\
    1 & 0 & 0 & 0  \\
    0 & 0 & 0 & 1  \\
    0 & 1 & 0 & 0  \\
  \end{pmatrix}.
\end{equation*}
All nodes are is visited equally often. Another network could correspond to the routing $1\to2\to3\to4\to1$.  Yet another would be this:
\begin{equation*}
  P =
  \begin{pmatrix}
    1/2 & 0   & 1/2 & 0  \\
    1/2 & 1/2 & 0 & 0  \\
    0   & 1/2 & 0 & 1/2  \\
    0   & 0   & 1/2 & 1/2  \\
  \end{pmatrix}.
\end{equation*}
Observe that I choose the rate such that each node receives the same
fraction of traffic.

For the interested: There must be many such `equivalent' networks, but
a general method to classify all equivalent networks seems hard. The
question comes down to finding all stochastic matrices $P$ that have
at least one (left) eigenvector $V$ in common. Observe that the visit
ratio equation $VP=P$ implies that $V$ is a left eigenvector with
eigenvalue 1.
\end{solution}
\end{exercise}

\begin{exercise}
Zijm.Ex.3.1.5
\begin{hint}
 First write down all  different states, and then use Zijm.Eq.3.2 and 3.3.
\end{hint}
\begin{solution}
  Since $M=2$, we have three stations: Stations 0, 1, and 2. We also
  have $N=2$ jobs. Thus, the states are $(2,0,0)$ (meaning that the
  load/unload station has 2 jobs, and stations 1 and 2 no jobs),
  $(0,2,0)$, $(0,0,2)$, $(1,1,0)$, $(1,0,1)$, and $(0,1,1)$. Note that
  the routing matrix $P$ is not given, so that we cannot compute the
  visit ratios. Hence I leave them unspecified. Now, realize that for
  a fixed $\vec n=(n_0, n_1, n_2)$,
  $\Pi_i f_i(n_i) = f_0(n_0)f_1(n_1)f_2(n_2)$, so that if 
  \begin{align*}
    \vec n &= (2,0,0) & f_0(2)f_1(0)f_2(0) &= \left(\frac{V_0}{\mu_0}\right)^2, \\
    \vec n &= (0,0,2) & f_0(0)f_1(0)f_2(2) &= \left(\frac{V_2}{\mu_2}\right)^2, \\
    \vec n &= (0,2,0) & f_0(0)f_1(2)f_2(0) &= \frac12\left(\frac{V_1}{\mu_1}\right)^2.
  \end{align*}
  Note that in this last result we use that Station 1 has two servers. For the other combinations, 
  \begin{align*}
    \vec n &= (1,1,0) & f_0(1)f_1(1)f_2(0) &= \frac{V_0}{\mu_0}\frac{V_1}{\mu_1}, \\
    \vec n &= (1,0,1) & f_0(1)f_1(0)f_2(1) &= \frac{V_0}{\mu_0}\frac{V_2}{\mu_2}, \\
    \vec n &= (0,1,1) & f_0(0)f_1(1)f_2(1) &= \frac{V_1}{\mu_1}\frac{V_2}{\mu_2}.
  \end{align*}
Finally, add up all the above numbers to make $G(M,N)$. 
\end{solution}
\end{exercise}



\Closesolutionfile{hint}
\Closesolutionfile{ans}

\opt{solutionfiles}{
\subsection*{Hints}
\input{hint}
\subsection*{Solutions}
\input{ans}
}

%\clearpage
