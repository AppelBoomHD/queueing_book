
\section{N-policies for the $M/M/1$ queue}
\label{sec:n-policies}


Let us consider the $M/M/1$ queue in which the server switches off as soon as it becomes idle and it costs~$K$ to switch on.
Supposing that each job charges $h$ Euros per unit time while in the system, it makes sense to build up a queue of jobs while the server is idle, and after some time switch on the server to process jobs until the system is empty again.
In particular, we analyze the influence of the $N$-policy on the long-run average cost.
For this we make a cost model in several steps and at the end we discuss how to minimize the cost as a function of the threshold~$N$.
In passing, we obtain a third way to compute the time-average number $\E L$ of jobs in the system; the first resulted from the analysis of the $M/M/1$ queue in~\cref{sec:mm1}, the second from Little's law, cf.~\cref{ex:l-215}

\opt{solutionfiles}{
\subsection*{Theory and Exercises}
\Opensolutionfile{hint}
\Opensolutionfile{ans}
}



Suppose the server is on, and there are $q$ jobs in the system.
Let us write $T(q)$ for the expected time to clear the system.
Now one of two events happens first.
Either a new job enters the system, or the job in service leaves.
It follows from~\cref{ex:3} that $\alpha=\lambda/(\lambda+\mu)$ is the probability the first event occurs and $\beta=\mu/(\lambda+\mu)$ is the probability the second occurs.
Moreover, from~\cref{ex:10} we see that the expected time to either an arrival or a departure, whichever is first, is $1/(\mu+\lambda)$.
Therefore, $T(q)$, must satisfy the following recursion:
\begin{equation}
  \label{eq:92}
  T(q) = \alpha T(q+1) + \beta T(q-1) + \frac{1}{\lambda+\mu}. 
\end{equation}
The reader should note that this type of recursion is a difference equation.
Moreover, the ideas behind its derivation are very similar to the ideas used in dynamic programming.

\begin{exercise}\clabel{ex:aab}
Provide an intuitive explanation for this formula, and show that $T(q)=q/(\mu-\lambda)$ solves~\cref{eq:92}. 
\begin{hint}
We just guess the form $T(q) = aq + b$, i.e., linear in $q$. It is clear that $T(0)=0$: if the system is empty, the time to clear it is zero. Thus, $b=0$, and it remains to solve for $a$. 
\end{hint}

\begin{solution}
  Jobs come in at rate $\lambda$ and are served at rate $\mu$.
  Therefore the systems drains at rate $\mu-\lambda$.
  When there are $q$ jobs in the system, it must take $q/(\mu-\lambda)$ units of time to serve these jobs.

  We substitute the guess $T(q) = aq$ in ~\cref{eq:92}, and solve for $a$. Specifically,
  \begin{equation*}
    a q = \alpha (aq + a) + \beta (a q - a) + 1/(\lambda+\mu). 
  \end{equation*}
Noticing that $\alpha + \beta = 1$, this reduces to $0 = a(\alpha - \beta) + 1/(\lambda + \mu)$. The rest is trivial.
\end{solution}
\end{exercise}


With the same line of reasoning we can compute the expected cost $V(q)$ to clear the system.
Noting that the queueing cost is $hq$ per unit time when there are $q$ jobs in the system, $V(q)$ must satisfy the relation
\begin{equation}\label{eq:93}
  V(q) = \alpha V(q+1) + \beta V(q-1) + h\frac{q}{\lambda + \mu}.
\end{equation}
%because $1/(\lambda + \mu)$ is the expected time we have to wait until either an arrival or a departure occurs.


To solve this, observe that in~\cref{eq:92} the last term is a constant, and that $T(q)$ is a linear function in~$q$.
In the recursion for $V(q)$ we see that the last term is linear in $q$ which leads us to the guess $a q^2 + b q + c$ for $V(q)$.
Thus, we substitute this into the above expression and then try to solve for $a,b$ and $c$.\footnote{The reader might wonder about the uniqueness of the solution.
  Noting that $V(q+1)$ follows directly from $V(q)$ and $V(q-1)$, there can be just one solution.}
As $V(0)=0$, it follows already that $c=0$.
Thus, it remains to find $a$ and $b$.

\begin{exercise}\clabel{ex:nmm-2}
  Use the ideas of~\cref{ex:aab} to show that
  \begin{equation*}
    V(q) = a q^2 + b q = \frac h 2 \frac 1 {\mu -\lambda} q^2 + \frac h 2 \frac{\lambda + \mu}{(\mu - \lambda)^2}q.
  \end{equation*}

\begin{hint}
  Fill in our guess $V(q) = aq^2 + bq$ in the recursion~\cref{eq:93}.
  Then, match the coefficients of $q^2$, $q$ and the constants, and use that $\alpha + \beta = 1$.
\end{hint}

\begin{solution}
 Filling our guess $V(q) = aq^2 + bq$ in the recursion~\cref{eq:93}  gives
  \begin{equation*}
    aq^2 + b q = \alpha a (q^2 + 2q + 1) + \alpha b (q+1) + \beta a (q^2 - 2q + 1) + \beta b (q - 1) + hq/(\lambda + \mu). 
  \end{equation*}
  Matching the coefficients of $q^2$, $q$ and the constants, and using that $\alpha + \beta = 1$, we obtain
  \begin{align*}
    a &= \alpha a + \beta a &\implies a &= a, \\
    b &= \alpha a 2 + \alpha b - \beta a 2 + \beta b + h/(\lambda + \mu), & \implies a &= \frac h 2 \frac 1 {\mu -\lambda},\\
    0 &= \alpha a + \alpha b + \beta a - \beta b, & \implies b &= \frac a {\beta - \alpha}. 
  \end{align*}
By combining the last two equations we get the answer. 
\end{solution}
\end{exercise}


Interestingly, the above turns out to be immediately useful.
Suppose we switch on the server when $N=1$, that is, directly at the arrival of the first job after the system became empty.
Write $C(1)$ for the duration of a cycle that starts when the system becomes idle and stops when the system becomes idle again (and after at least one job has arrived).
In other words, the cycle consists of an idle and a busy period.

\begin{exercise}\clabel{ex:nmm-3}
Explain that $C(1)=1/\lambda + T(1)$ for the $M/M/1$ queue.
\begin{hint}
  What is the expected  time to a job just after the server becomes idle? (Use the memoryless property.)
\end{hint}
\begin{solution}
  After clearing the system, we have to wait for a new job.
  The expected time for this job arrival is $1/\lambda$.
  (How do we use the memoryless property here?)
  Once the job arrived, the busy period starts with this one job, and the expected busy time is $T(1)$.
\end{solution}
\end{exercise}

\begin{exercise}\clabel{ex:nmm5}
Use the renewal-reward theorem to explain the  relation
\begin{equation*}
  \frac{V(1)}{C(1)} = \frac{V(1)}{1/\lambda + T(1)} = h \E L,
\end{equation*}
where $\E L$ is given by~\cref{eq:el}. Then use the above expressions for $V(1)$ and $C(1)$ to verify this.

With this result, we have found yet another way to compute the expected number of jobs in the system.  
\begin{hint}
  What is the cost of one cycle? What is the duration of one cycle? 
\end{hint}
\begin{solution}
  First the explanation.
  On the one hand, the cost of the jobs in the system during one cycle must be $V(1)$.
  The duration of one cycle is $C(1)$.
  By the renewal-reward theorem, the time-average cost must then by $V(1)/C(1)$.
  On the other hand, if the time-average number of jobs in the system is $\E L$, and each job pays $h$ per unit time, the time-average cost must be $h \E L$.

  Now the check; it's plain algebra:
\begin{align*}
  \frac{V(1)}{1/\lambda + T(1)}
  &= \frac{a+b}{1/\lambda + 1/(\mu-\lambda)} \\
  &= \left(\frac h 2 \frac 1 {\mu -\lambda} + \frac h 2 \frac{\lambda + \mu}{(\mu - \lambda)^2}\right)\frac{\lambda(\mu-\lambda)}{\mu}\\
&=\frac{h}2 \rho \left(1 + \frac{\lambda+\mu}{\mu-\lambda}\right) = \frac{h}{2} \rho \frac{2\mu}{\mu-\lambda}  \\
&= h \frac{\rho}{1-\rho} = h \E L. \\
\end{align*}
\end{solution}
\end{exercise}

It remains to generalize to a general threshold $N$; just above we already covered the case with $N=1$.
As we already have expressions for the cost and time for the time the server is on, we only have to consider the cost and time while the server is off.
Note that right after the server switches off, we need $N$ independent inter-arrival times to reach level $N$.
By~\cref{ex:54}, the expected time  to switch on must be equal to $N/\lambda$.

For the cost while building up the queue, we use again a recursive procedure.
Write $W(q)$ for the accumulated queueing cost from the moment the server becomes idle up to the arrival time of the $q$th job (the job that sees $q-1$ jobs in the system).
Then, by the next exercise,
\begin{equation}\label{eq:99}
  W(q) = W(q-1) + h\frac{q-1}{\lambda} = h \frac{q(q-1)}{2\lambda}.
\end{equation}

\begin{exercise}\clabel{ex:nmm-4}
Explain the above recursion, and derive the right-hand side.
\begin{hint}
  Use~\cref{eq:61}.
\end{hint}
\begin{solution}
  The cost up to the $q$th job is the cost up to the arrival of job $q-1$ plus the cost while there are $q-1$ jobs in the system.
  The time between the arrival of job $q-1$ and $q$ is $1/\lambda$.

  It is clear that $W(0)=0$. Therefore, $W(q) = h\sum_{i=1}^q (i-1) = hq(q-1)/2\lambda$. 
\end{solution}
\end{exercise}

It remains to assemble all results. Let us assume that the switching cost is $K$. Then, by the renewal-reward theorem, the time-average cost of the $N$-policy is equal to
\begin{equation*}
  \frac{W(N)+K+V(N)}{C(N)},
\end{equation*}
where $C(N) = N/\lambda + T(N)$.

Finding the optimal $N$ is easy (from a practical point).
Observe that $V(N)$ and $W(N)$ are quadratic in $N$, while $C(N)$ is linear in $N$.
Hence, the average cost is a convex function of $N$, and locating the minimizer of a one-dimensional convex function is simple.



\opt{solutionfiles}{
\Closesolutionfile{hint}
\Closesolutionfile{ans}
\subsection*{Hints}
\input{hint}
\subsection*{Solutions}
\input{ans}
}

%\clearpage 


%%% Local Variables:
%%% mode: latex
%%% TeX-master: "../companion"
%%% End:
