\section{Non-preemptive Interruptions, Server Adjustments}
\label{sec:non-preempt-interr}

\subsection*{Theory and Exercises}

\Opensolutionfile{hint}
\Opensolutionfile{ans}

%See section 1.11 of Zijm's book for the theory.

Besides setups, other types of interruptions can occur, for example, small adjustments required by a machine after serving a few jobs.  Interruptions, or outages, occurring \emph{between} two jobs are called \emph{non-preemptive} as they do not preempt the processing of a job.  In this section we develop a simple model to incorporate the effects of such non-preemptive outages in the service times of jobs, so that we can use the $G/G/1$ waiting time formula. 


\begin{exercise}
  What makes outages different from service time lost due to setup times between batches, as discussed in Section~\ref{sec:setups-batch-proc}?  
  \begin{solution}
Setups are planned, in the model of Section~\ref{sec:setups-batch-proc}, between $B$ jobs. Thus, the setups do not occur between any two jobs.  Random interruptions, on the other hand, can potentially happen anywhere and between any two jobs. 
  \end{solution}
\end{exercise}


Let us assume that adjustments occur geometrically distributed, i.e., have the memoryless property in discrete time, between any two jobs with a mean of $B$ jobs between any two repairs.  Thus, the probability of a repair between two jobs is $p=1/B$. 

\begin{exercise}
  Show that the average \recall{effective processing time} $S$, i.e., the time the service is occupied with processing jobs including adjustments, is 
  \begin{equation*}
    \E{S} = \E{S_0} + \frac{\E T}B.
  \end{equation*}
\begin{hint}
If there is no failure, the service time is $S_0$. If there is a failure, the service of a job is $T + S_0$, since we add the outage time to the service time of the job. 
\end{hint}
  \begin{solution}
    \begin{equation*}
      \E{S} = (1-p)\E{S_0} + p (\E{T} + \E{S_0}) = \E{S_0} \frac{B-1}B + (\E{S_0} + \E T) \frac 1B,
    \end{equation*}
since $p=1/B$. 
  \end{solution}
\end{exercise}

Thus the effective server load including downtimes is $\rho = \lambda \E{S}$. 

%The next step is to compute $\E{S^2}$. 
\begin{exercise}
  Show that
  \begin{equation*}
    \E{S^2} = \E{S_0^2} + 2\frac{\E{S_0}\E{T}} B + \frac{\E{T^2}}B.
  \end{equation*}
  \begin{solution}
  \begin{equation*}
    \begin{split}
    \E{S^2} 
&= \E{S_0^2}\frac{B-1}B + \E{(S_0+T)^2}\frac 1B \\
&= \E{S_0^2}\frac{B-1}B + \E{S_0^2} \frac 1 B + 2 \E{S_0} \E T\frac 1B + \E{T^2}\frac 1B.
    \end{split}
  \end{equation*}
  \end{solution}
\end{exercise}

\begin{exercise}
  Use the above to find that
  \begin{equation*}
    \V{S} = \V{S_0} + \frac{\V{T}} B + (B-1)\left(\frac{\E T}{B}\right)^2.
  \end{equation*}
  \begin{solution}
    \begin{equation*}
      \begin{split}
\V{S} 
&=\E{S^2} - (E{S})^2 \\
&= \E{S_0^2} + 2 \E{S_0} \E T\frac 1B + \E{T^2}\frac 1B  \\
&\quad - (\E{S_0})^2 - 2\E{S_0}\E T \frac 1 B - (\E T)^2\frac1{B^2}\\
&=  \V{S_0} + ((\E{T^2} - (\E T)^2)\frac 1 B + (\E T)^2\left(\frac 1B - \frac 1{B^2}\right).
      \end{split}
    \end{equation*}
  \end{solution}
\end{exercise}

With the above we can compute $C_s^2=\V{S}/(\E{S}^2)$ of the effective job processing times. We have now all elements to fill in the $G/G/1$ waiting time formula!

\begin{exercise}
A machine requires an adjustment with average $5$ hours and standard deviation of $2$ hours. Jobs arrive as a Poisson process with rate $\lambda=9$ per working day. The machine works two $8$ hour shifts a day. Work not processed on a day is carried over to the next day. Job service times are 1.5 hours, on average, with standard deviation of $0.5$ hour. Interruptions occur on average after $30$ jobs.

Compute the average waiting time in queue.
\begin{hint}
  Get the units right. First compute the load, and then compute the rest.
\end{hint}
\begin{solution}
  First we determine the load. 
  \begin{pyconsole}
B=30
ES0 = 1.5
labda = 9./(2*8) # arrival rate per hour
ET=5
ESe=ES0+ET/B
ESe
rho = labda*ESe
rho
  \end{pyconsole}
So, at least the system is stable.

\begin{pyconsole}
VS0 = 0.5*0.5
VT = 2*2
VSe = VS0 + VT/B + (B-1)*(ET/B)**2
VSe
Ce2 = VSe/(ESe*ESe)
Ce2
\end{pyconsole}

And now we can fill in the waiting time formula
\begin{pyconsole}
Ca2=1 # Poisson arrivals
EW = (Ca2+Ce2)/2 * rho/(1-rho) * ESe
EW  
\end{pyconsole}
\end{solution}
\end{exercise}

Observe that with these formulas we can obtain quantitative insights into the effects of reducing adjustment times, or the variability of these adjustments times. In particular, sensitivity analysis is very relevant for practical cases to find out the dominating parameters. 




\Closesolutionfile{hint}
\Closesolutionfile{ans}

\opt{solutionfiles}{
\subsection*{Hints}
\input{hint}
\subsection*{Solutions}
\input{ans}
}

%\clearpage

%%% Local Variables:
%%% mode: latex
%%% TeX-master: "../queueing_book"
%%% End:
