\section{Non-preemptive Interruptions, Server Adjustments}
\label{sec:non-preempt-interr}

\subsection*{Theory and Exercises}

\Opensolutionfile{hint}
\Opensolutionfile{ans}

%See section 1.11 of Zijm's book for the theory.


In Section~\ref{sec:setups-batch-proc} we studied the effect of setup times between batches of jobs.
In this model we assumed that the number of jobs between two setups is fixed to the batch size $B$.
In other words, there are no setups between any two jobs, setups are \emph{planned} between $B$ jobs.
However, other types of interruptions can occur, such as a machine requiring a small adjustment after just a few jobs.
In fact, such random interruptions can happen between any two jobs.
As such outages do not interrupt the processing of a job in service,  we call this \recall{non-preemptive outages}.
In this section we develop a simple model to understand the impact of such outages on the mean time jobs spend in the system.

Let us assume that adjustments occur geometrically distributed between any two jobs with a mean of $B$ jobs between any two repairs.
Consequently, the probability of an outage between any two jobs is $p=1/B$.
(Observe that geometrically distributed random variables satisfy the memoryless property in discrete time.)
The adjustments form an i.i.d.
sequence of random variables distributed as the common random variable ~$R$ and independent of $S_0$.
The main idea is to incorporate the effects of these outages in the mean and SCV of job service times, so that we can use the $G/G/1$ waiting time formula.

Define the \recall{effective processing time} $S$ as the time the server is occupied with processing a job including a potential adjustment, and write $S_0$ for the net service time of a job, i.e., the service time required to serve just the job.

\begin{exercise}[\faCalculator]
  Show that the average effective processing time satisfies $ \E{S} = \E{S_0} + \E R/B$. Conclude that the effective server load including down-times is $\rho = \lambda \E{S}$. 
\begin{hint}
If there is no failure, the service time is $S_0$. If there is a failure, the service of a job is $R + S_0$, since we add the outage time to the service time of the job. 
\end{hint}
  \begin{solution}
    \begin{equation*}
      \E{S} = (1-p)\E{S_0} + p (\E{R} + \E{S_0}) = \E{S_0} \frac{B-1}B + (\E{S_0} + \E R) \frac 1B,
    \end{equation*}
since $p=1/B$. 
  \end{solution}
\end{exercise}

The next step is to find an expression for $\E{S^2}$ from which $\V{S}$ will follow easily.

%The next step is to compute $\E{S^2}$. 
\begin{exercise}[\faCalculator]
  Show that
  \begin{equation*}
    \E{S^2} = \E{S_0^2} + 2\frac{\E{S_0}\E{R}} B + \frac{\E{R^2}}B.
  \end{equation*}
  \begin{solution}
  \begin{align*}
    \E{S^2} 
&= (1-p) \E{S_0^2} + p \E{(S_0+R)^2}\\
&= (1-p) \E{S_0^2} + p \E{S_0^2}  + 2 p \E{S_0} \E R + p \E{R^2}. 
  \end{align*}
  Simplify and substitute $p=1/B$. 
  \end{solution}
\end{exercise}

\begin{exercise}[\faCalculator]
  Use the above to find that
  \begin{equation*}
    \V{S} = \V{S_0} + \frac{\V{R}} B + (B-1)\left(\frac{\E R}{B}\right)^2.
  \end{equation*}
  \begin{solution}
    \begin{equation*}
      \begin{split}
\V{S} 
&=\E{S^2} - (\E{S})^2 \\
&= \E{S_0^2} + 2 \E{S_0} \E R\frac 1B + \E{R^2}\frac 1B  \\
&\quad - (\E{S_0})^2 - 2\E{S_0}\E R \frac 1 B - (\E R)^2\frac1{B^2}\\
&=  \V{S_0} + ((\E{R^2} - (\E R)^2)\frac 1 B + (\E R)^2\left(\frac 1B - \frac 1{B^2}\right).
      \end{split}
    \end{equation*}
  \end{solution}
\end{exercise}

With the above we can compute $C_s^2=\V{S}/(\E{S}^2)$ of the effective job processing times. We have now all elements to fill in the $G/G/1$ waiting time formula!

\begin{exercise}[\faPhoto]
  A machine requires an adjustment with average $5$ hours and standard deviation of $2$ hours.
  Jobs arrive as a Poisson process with rate $\lambda=9$ per working day.
  The machine works two $8$ hour shifts a day.
  Work not processed on a day is carried over to the next day.
  Job service times are 1.5 hours, on average, with standard deviation of $0.5$ hour.
  Interruptions occur on average between $30$ jobs.
  Compute the average waiting time in queue.
\begin{hint}
  Get the units right. First compute the load, and then compute the rest.
\end{hint}
\begin{solution}
  First we determine the load. 
  \begin{pyconsole}
B=30
ES0 = 1.5
labda = 9./(2*8) # arrival rate per hour
ER=5.
ESe=ES0+ER/B
ESe
rho = labda*ESe
rho
  \end{pyconsole}
So, at least the system is stable.

\begin{pyconsole}
VS0 = 0.5*0.5
VR = 2.*2.
VSe = VS0 + VR/B + (B-1)*(ER/B)**2
VSe
Ce2 = VSe/(ESe*ESe)
Ce2
\end{pyconsole}

And now we can fill in the waiting time formula
\begin{pyconsole}
Ca2=1 # Poisson arrivals
EW = (Ca2+Ce2)/2 * rho/(1-rho) * ESe
EW  
\end{pyconsole}
\end{solution}
\end{exercise}

Observe that with these formulas we can obtain quantitative insights into the effects of reducing adjustment times, or the variability of these adjustments times.
For instance, we might decide to do less adjustments, so that $p$ decreases, but the average outage time (or its variance) may increase as a function of this decision.
We now have tools to analyze the consequences of such decisions without needing to actually do the experiments in real life to see the effects.



\Closesolutionfile{hint}
\Closesolutionfile{ans}

\opt{solutionfiles}{
\subsection*{Hints}
\input{hint}
\subsection*{Solutions}
\input{ans}
}

%\clearpage

%%% Local Variables:
%%% mode: latex
%%% TeX-master: "../queueing_book"
%%% End:
