\section{Non-preemptive Interruptions, Server Adjustments}
\label{sec:non-preempt-interr}


In~\cref{sec:setups-batch-proc} we studied the effect of setup times between job batches with a fixed size~$B$.
%In other words, there are no setups between any two jobs, setups are \emph{planned} between $B$ jobs.
However, other types of interruptions can occur, such as a machine requiring random adjustments that can occur between any two jobs.
This type of outages is \recall{non-preemptive} as the outages do not interrupt the processing of a job in service.
In this section we develop a simple model to understand the impact of such outages on job sojourn times; we use the same notation as in~\cref{sec:setups-batch-proc} and follow the same line of reasoning.
With this model, we can analyze the effects of reducing adjustment times or the variability of these adjustments times.
For instance, we might decide to do fewer adjustments, but the average outage times become longer.

\opt{solutionfiles}{
\subsection*{Theory and Exercises}
\Opensolutionfile{hint}
\Opensolutionfile{ans}
}

We assume that adjustments $\{R_i\}$ occur geometrically distributed between any two jobs with a mean of $B$ jobs between any two adjustments.
Consequently, the probability of an outage between any two jobs is $p=1/B$.
Observe that geometrically distributed random variables satisfy the memoryless property in discrete time.
Hence, our assumption implies that the occurrence of an adjustment between jobs $i$ and $i+1$ has no effect on the probability that an adjustment is necessary between jobs $i+1$ and $i+2$.


Contrary to the batch processing case of~\cref{sec:setups-batch-proc}, we now only need to find the mean and variance of the effective processing times.
%From~\cref{ex:l-256} and \cref{,ex:l-257} it follows that
\begin{exercise}\clabel{ex:l-256}
Show that the average effective processing time is given by
\begin{equation} \label{eq:88}
 \E{S} = \E{S_0} + \frac{\E R}B
\end{equation}

\begin{hint}
  If there is no interruption, the service time is $S_0$.
  However, if there is an interruption, the service of a job is $R + S_0$, since we add the outage time to the service time of the job.
  Relate this insight to~\cref{ex:49}.
  In that exercise, take $p=(B-1)/B$ as the probability of a short job of size $\E{S_0}$ and $q=1/B$ as the probability of a long job of $\E{S_0} + \E{R}$.
\end{hint}
\begin{solution}
 \begin{equation*}
 \E{S} = (1-p)\E{S_0} + p (\E{R} + \E{S_0}) = \E{S_0} \frac{B-1}B + (\E{S_0} + \E R) \frac 1B,
 \end{equation*}
since $p=1/B$. 
\end{solution}
\end{exercise}

Clearly, the effective server load including down-times is $\rho = \lambda \E{S}$.
Since the adjustments do not affect the job arrival process, we only have to find an expression how it affects~$C_s^2$.

%The next step is to compute $\E{S^2}$. 
\begin{extra}\clabel{ex:78}
 Show that
 \begin{equation*}
 \E{S^2} = \E{S_0^2} + 2\frac{\E{S_0}\E{R}} B + \frac{\E{R^2}}B.
 \end{equation*}
\begin{hint}
  Again, use~\cref{ex:49}.
\end{hint}
\begin{solution}
 \begin{align*}
 \E{S^2} 
&= (1-p) \E{S_0^2} + p \E{(S_0+R)^2}\\
&= (1-p) \E{S_0^2} + p \E{S_0^2} + 2 p \E{S_0} \E R + p \E{R^2}. 
 \end{align*}
 Simplify and substitute $p=1/B$. 
\end{solution}
\end{extra}

\begin{exercise}\clabel{ex:l-257}
Derive 
\begin{equation}\label{eq:89} 
 \V{S} = \V{S_0} + \frac{\V{R}} B + (B-1)\left(\frac{\E R}{B}\right)^2.
\end{equation}
\begin{hint}
 First compute $\E{S^2}$. See~\cref{ex:78}.
\end{hint}
\begin{solution}
 \begin{equation*}
 \begin{split}
\V{S} 
&=\E{S^2} - (\E{S})^2 \\
&= \E{S_0^2} + 2 \E{S_0} \E R\frac 1B + \E{R^2}\frac 1B \\
&\quad - (\E{S_0})^2 - 2\E{S_0}\E R \frac 1 B - (\E R)^2\frac1{B^2}\\
&= \V{S_0} + ((\E{R^2} - (\E R)^2)\frac 1 B + (\E R)^2\left(\frac 1B - \frac 1{B^2}\right).
 \end{split}
 \end{equation*}
\end{solution}
\end{exercise}

With this, we have all elements to fill in the $G/G/1$ waiting time formula!


\begin{exercise}\clabel{ex:l-255}
 Jobs arrive as a Poisson process with rate $\lambda=9$ per working day.
 The machine works two $8$ hour shifts a day.
 Work not processed on a day is carried over to the next day.
 Job service times are 1.5 hours, on average, with standard deviation of $0.5$ hours.
 Outages occur on average between $30$ jobs.
The average duration of an outage  is of $5$ hours and has a standard deviation of $2$ hours.

 Compute the average waiting time in queue.
\begin{hint}
 Get the units right. First compute the load, and then compute the rest.
\end{hint}
\begin{solution}
 First we determine the load. 
 \begin{pyconsole}
B=30
ES0 = 1.5
labda = 9./(2*8) # arrival rate per hour
ER=5.
ESe=ES0+ER/B
ESe
rho = labda*ESe
rho
 \end{pyconsole}
So, at least the system is stable.

\begin{pyconsole}
VS0 = 0.5*0.5
VR = 2.*2.
VSe = VS0 + VR/B + (B-1)*(ER/B)**2
VSe
Ce2 = VSe/(ESe*ESe)
Ce2
\end{pyconsole}

And now we can fill in the waiting time formula.
\begin{pyconsole}
Ca2=1 # Poisson arrivals
EW = (Ca2+Ce2)/2 * rho/(1-rho) * ESe
EW 
\end{pyconsole}
\end{solution}
\end{exercise}


\begin{remark}
  Let us compare the results of this section to those of~\cref{sec:non-preempt-interr}.
  In both cases the expected service time is the same: an amount $\E R/ B$ gets added to the net processing time $\E{S_0}$.
  However, the impact on the SCV is different.
  Unexpected interruptions must have a larger effect on variability than expected interruptions.

  In more general terms, we can use part of the model of~\cref{sec:setups-batch-proc} to analyze the effect of planned maintenance---do an adjustment after every $B$ jobs---while with the model of this section we consider adjustments that arrive completely unexpectedly.
  With these two models we can compare the influence on waiting times of doing planned adjustments more often (take $B$ a bit small), but remove the unexpected interruptions.
  Hence, we have another set of tools to compare different scenarios to organize production systems.
\end{remark}

\opt{solutionfiles}{
\Closesolutionfile{hint}
\Closesolutionfile{ans}
\subsection*{Hints}
\input{hint}
\subsection*{Solutions}
\input{ans}
}

%\clearpage

%%% Local Variables:
%%% mode: latex
%%% TeX-master: "../companion"
%%% End:
