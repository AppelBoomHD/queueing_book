\section{Exponential Distribution}
\label{sec:expon-distr}


\subsection*{Theory and Exercises}

\opt{solutionfiles}{
\Opensolutionfile{hint}
\Opensolutionfile{ans}
}

In Section~\ref{sec:poisson-distribution} we introduced the Poisson process as a natural model of the (random) number of jobs arriving during intervals of time.
As we will see in the sections to come, we can model single-server queueing system in continuous time if we specify the (probability) distribution of the inter-arrival times, i.e., the time between consecutive arrival epochs of jobs.
A particularly fruitful model for the distribution of the inter-arrival times is the exponential distribution because, as it turns out, it is intimately related to the Poisson distribution.
Besides explaining this relation, we derive many useful properties of the exponential distribution, in particular that it is \emph{memoryless}.

We say that $X$ is an \recall{exponentially distributed} random variable with mean $1/\mu$ if 
\begin{equation*}
  \P{X \leq t} = 1- e^{-\lambda t},
\end{equation*}
and then  we write $X\sim \Exp(\lambda)$.


The Poisson process $N$ and exponentially distributed inter-arrival times are intimately related: A counting process $\{N(t)\}$ is a \emph{Poisson process} with rate $\lambda$ if and only if  the inter-arrival times $\{X_i\}$ are \recall{i.i.d.} (\recall{independent and identically distributed}) and exponentially distributed with mean $1/\lambda$,  in short, 
\begin{equation*}
X_i\sim \Exp(\lambda) \Leftrightarrow N(t) \sim P(\lambda t).
\end{equation*}


We next provide further relations between the Poisson distribution and the exponential distribution. 


\begin{exercise}[\faFlask]
In the above expression for $\P{X\leq t}$ we require that $\lambda>0$. What would happen if you would allow $\lambda$ to be zero or negative?
\begin{hint}
  The interpretation of the function $1-e^{-\lambda t}$ is a probability. What are the consequences of this? What happens if $\lambda=0$?
\end{hint}
\begin{solution}
  If $\lambda<0$, then $1-e^{-\lambda t}$ grows to $-\infty$ if $t\to \infty$. Just by itself this is not a problem. However, in our case $1-e^{-\lambda t}$ has the interpretation of a distribution function. Now, recall that a distribution function is bounded to values in the interval $[0,1]$.

Suppose $\lambda=0$. Then $\P{X\leq t} = 1-e^{0} = 0$. In words, this would mean that the probability of a finite inter-arrival time between any two customers is zero. So, no customers can arrive in this case. 
\end{solution}
\end{exercise}


\begin{exercise}[\faCalculator] \label{exer:lambda}
  If the random variable $X\sim\Exp(\lambda)$, show that its mean $\E X = \frac{1}\lambda$. 
  \begin{hint}
 \begin{equation*}
    \E X = \int_0^\infty t f(t)\, \d t =
    \int_0^\infty t \lambda e^{-\lambda t}\, \d t,
  \end{equation*}
  where~$f(t)=\lambda e^{-\lambda t}$ is the density function of the distribution function $F$ of $X$. Now solve the integral.
  \end{hint}
  \begin{solution}
    \begin{align*}
\E{X} 
&= \int_0^\infty t \lambda e^{-\lambda t} \d t, \quad\text{density is } \lambda e^{-\lambda t} \\
&=   \lambda^{-1} \int_0^\infty u e^{-u}\, \d u, \quad \text{ by  change of variable $u=\lambda t$},   \\
&=  -\lambda^{-1}\left. t e^{- t}\right|_0^\infty + \lambda^{-1} \int_0^\infty e^{- t} \d t\\
&=  - \lambda^{-1} \left. e^{- t} \right|_0^\infty =  \frac1\lambda.
    \end{align*}
  \end{solution}
\end{exercise}

\begin{exercise}[\faCalculator]\label{ex:15} 
  If $X\sim\Exp(\lambda)$, show that its second moment $\E{X^2} =  \frac{2}{\lambda^2}$.
  \begin{hint}
  \begin{equation*}
  \E{X^2}= \int_0^\infty t^2 \lambda e^{-\lambda t}\, \d t =  \frac{2}{\lambda^2}.
  \end{equation*}
  \end{hint}
  \begin{solution}
    \begin{align*}
\E{X^2} 
&= \int_0^\infty t^2 \lambda e^{-\lambda t} \d t \\
&=   \lambda^{-2} \int_0^\infty u^2 e^{-u}\, \d u, \quad \text{ by  change of variable $u=\lambda t$},   \\
&= -\lambda^{-2}\left. t^2 e^{- t}\right|_0^\infty + 2\lambda^{-2}\int_0^\infty t e^{- t} \d t \\
&=  -2\lambda^{-2}\left. t e^{- t}\right|_0^\infty + 2\lambda^{-2} \int_0^\infty e^{- t} \d t\\
&=  - 2\lambda^{-2} \left. e^{- t} \right|_0^\infty \\
&=  2/\lambda^2.
    \end{align*}
  \end{solution}
\end{exercise}


\begin{exercise}[\faCalculator]
  If $X\sim\Exp(\lambda)$, show that the \recall{variance}
$\V X = \lambda^{-2}$.
\begin{hint} Use~\eqref{eq:68}. 
  % , and memorize, the very practical formula
  % \begin{equation*}
  % \V X = \E{X^2} - (\E X)^2.
  % \end{equation*}
  \end{hint}
  \begin{solution}
    By the previous problems, $\E{X^2}=2/\lambda^2$ and $\E X = 1/\lambda$. Hence the variance $\V X = \E{X^2} - (\E X)^2 = \frac{2}{\lambda^2} - (\frac{1}{\lambda})^2 = \lambda^{-2}.$
  \end{solution}
\end{exercise}

The  above exercises can also be easily solved with the moment-generating function of $X$:
\begin{equation}\label{eq:60}
  M_X(t) = \E{e^{t X}} = \int_0^\infty e^{t x} \lambda e^{-\lambda x} \d x.
\end{equation}

\begin{exercise}[\faBalanceScale]\label{ex:11}
  Suppose $X\sim\Exp(\lambda)$.
  Only when $t < \lambda$ the moment-generating function $M_X(t)$ is finite.
\begin{solution} The claim is true. 
\end{solution}
\end{exercise}

\begin{exercise}[\faCalculator]\label{ex:33}
 If $X\sim\Exp(\lambda)$, show that its moment-generating function
    \begin{equation*}
    M_X(t) = \frac{\lambda}{\lambda-t}.
    \end{equation*}
   \begin{hint}
    \begin{equation*}
      M_X(t) = \E{\exp(t X)} =\int_0^\infty e^{t x} f(x) \,\d x.
\end{equation*}
\end{hint}
    \begin{solution}
    \begin{align*}
      M_X(t) &= \E{\exp(t X)}  
=\int_0^\infty e^{tx} f(x) \,\d x 
=\int_0^\infty e^{tx} \lambda e^{-\lambda x} \,\d x  \\
&=\lambda \int_0^\infty e^{(t-\lambda)x} \,\d x 
=\frac{\lambda}{\lambda -t}.
    \end{align*}
    \end{solution}
  \end{exercise}

\begin{exercise}[\faCalculator]\label{ex:21}
Use the moment-generating function  of $X\sim\Exp(\lambda)$  to show that
    \begin{align*}
      \E{X} &=\frac1\lambda, & 
      \E{X^2} &=\frac2{\lambda^2}.
    \end{align*}
\begin{hint}
Use~\eqref{eq:69} and \eqref{eq:64}.
\end{hint}
\begin{solution} By Ex.~\ref{ex:33}, 
  $ M_X'(t)=\lambda/(\lambda-t)^2$. Hence, $M_X'(0)=1/\lambda$. Next, $M_X''(t)=2\lambda/(\lambda-t)^3$, hence $\E{X^2}=M_X''(0)=2\lambda/\lambda^3=2\lambda^{-2}$. 
\end{solution}
  \end{exercise}

\begin{exercise}[\faBalanceScale]\label{ex:29}
When  $X\sim\Exp(\lambda)$ the SCV is larger than $1$.
  \begin{hint}
  Use~\eqref{eq:62} 
  \end{hint}
  \begin{solution}
  By the previous problems, $\V X = 1/\lambda^2$ and $\E X=1/\lambda$, hence
  \begin{equation*}
    C^2= \frac{\V X}{(\E X)^2} = \frac{1/\lambda^2}{1/\lambda^2} = 1.
  \end{equation*}
\end{solution}
\end{exercise}

We now provide a number of relations between the Poisson distribution and the exponential distribution  to conclude that a process $N_\lambda$ is a Poisson process with rate $\lambda$ iff the inter-arrival times $\{X_i\}$ between individual jobs are i.i.d. $\sim\Exp(\lambda)$. 


\begin{exercise}[\faPhoto]
  If $N_\lambda$ is a Poisson process with rate $\lambda$, show that the time $X_1$ to the first arriving job is $\Exp(\lambda$).
  \begin{hint}
    What is the meaning of $\P{N(t)=0}$?
  \end{hint}
  \begin{solution}
If there are no arrivals in some interval $[0,t]$, then it
must be that $N(t) = 0$. Hence, for the first inter-arrival time $X_1$: 
\begin{equation*}
 \P{X_1> t} = \P{N(t) = 0} = e^{-\lambda t} \frac{(\lambda t)^0}{0!}= e^{-\lambda t}.
\end{equation*}
  \end{solution}
\end{exercise}

  \begin{exercise}[\faCalculator]
    Assume that inter-arrival times $\{X_i\}$ are i.i.d. and $\sim\Exp(\lambda)$. Let
the arrival time of the $i$th job be  $A_i=\sum_{k=1}^i X_k$. Show that %We do not need the MGF here.
 \begin{equation*}
\E{A_i} = \frac i\lambda.
 \end{equation*}
 \begin{hint} What is $\E{X+Y}$ for some general random variables (each with finite mean)?
 \end{hint}
  \begin{solution}
The simplest way to obtain the answer is to use that the expectation is a linear operator, i.e., $\E{X+Y}= \E X + \E Y$ for any r.v. $X$ and $Y$. Then,
\begin{equation*}
\E{A_i} = \E{\sum_{k=1}^i X_k} = i \E{X} = \frac i \lambda.
\end{equation*}
(Just as a reminder, $\E{X Y} \neq \E X \E Y$ in general. Only when $X$ and $Y$ are uncorrelated (which is implied by independence), the product of the expectations is the expectation of the products.)

  \end{solution}
\end{exercise}

  \begin{exercise}[\faCalculator]
 Let $A_i$ be the arrival time of customer $i$ and set $A_0=0$.
    Assume that the inter-arrival times $\{X_i\}$ are i.i.d.  with
    exponential distribution with mean $1/\lambda$ for some
    $\lambda>0$.  Prove that $A_i$ has density
\begin{equation*}
f_{A_i}(t) = \lambda e^{-\lambda t} \frac{(\lambda t)^{i-1}}{(i-1)!}.
\end{equation*}
\begin{hint}
 Check the result for $i=1$ by filling in $i=1$ (just to be
     sure that you have read the formula right), and compare the result
     to the exponential density. Then write $A_i =\sum_{k=1}^i X_k$, and compute the moment
     generating function for $A_i$ and use that the inter-arrival times
     $X_i$ are independent. Use the moment-generating function  of $X_i$.
\end{hint}
\begin{solution}
 One way to find the distribution of $A_i$ is by using the
    moment-generating function $M_{A_i}(t) = \E{e^{t A_i}}$ of
    $A_i$. Let $X_i$ be the inter-arrival time between customers $i$
    and $i-1$, and $M_X(t)$ the associated moment-generating
    function. Using the i.i.d. property of the $\{X_i\}$,
\begin{align*}
  M_{A_i}(t) &= \E{e^{t A_i}} = \E{\exp\left(t\sum_{k=1}^{i} X_k\right)} \\
& = \prod_{k=1}^{i} \E{e^{tX_k}} = 
\prod_{k=1}^{i} M_{X_k}(t) = 
\prod_{k=1}^{i} \frac{\lambda}{\lambda -t }
 = \left(\frac{\lambda}{\lambda -t }\right)^i.
\end{align*}
From a table of moment-generating functions it follows immediately that
$A_i \sim \Gamma(i,\lambda)$, i.e., $A_i$ is Gamma distributed.
\end{solution}
\end{exercise}

\begin{exercise}[\faCalculator]
  Use the density $f_{A_i}$ of the previous exercise to show that $\E{A_i}=i/\lambda$. 
  \begin{hint}
Use the standard integral 
    $\int_0^\infty x^n e^{-\alpha x} \d x =
    \alpha^{-n-1}n!$. 
    Another way would be to use that, once you have the moment
    generating function of some random variable $X$,
    $\E X = \frac{\d}{\d t} M_X(t) |_{t=0}$. 
  \end{hint}
\begin{solution}
  \begin{equation*}
\E{A_i} = \int_0^\infty t f_{A_i} (t) \, \d t  = 
\int_0^\infty t  \lambda e^{-\lambda t} \frac{(\lambda t)^{i-1}}{(i-1)!}\, \d t.
  \end{equation*}
Thus, 
  \begin{equation*}
\E{A_i} = \frac{1}{(i-1)!} \int_0^\infty   e^{-\lambda t} (\lambda t)^i\,\d t = \frac{i!}{(i-1)!\lambda}=\frac{i}\lambda,
  \end{equation*}
  where we used the hint.

What if we would use the moment-generating function, as derived by the previous exercise?
\begin{align*}
    \E{A_i} 
&= \left.\frac{\d}{\d t} M_{A_i}(t)\right|_{t=0} \\
&= \left.\frac{\d}{\d t} \left(\frac{\lambda}{\lambda-t}\right)^i\right|_{t=0} \\
&= i \left.\left(\frac{\lambda}{\lambda-t}\right)^{i-1}\frac{\lambda}{(\lambda-t)^2}\right|_{t=0} 
= \frac i\lambda.
\end{align*}


\end{solution}
\end{exercise}

\begin{exercise}[\faCalculator]
  If the inter-arrival times $\{X_i\}$ are i.i.d.
  $\sim \Exp(\lambda)$, prove that the number $N(t)$ of arrivals during interval $[0,t]$ is Poisson distributed.
  \begin{hint}
  Realize that
    $\P{N(t)=k} = \P{A_k \leq t} - \P{A_{k+1} \leq t}$.
  \end{hint}
    \begin{solution}
      We want to show that
    \begin{equation*}
      \P{N(t)=k} = e^{-\lambda t}\frac{(\lambda t)^k}{k!}.
    \end{equation*}
    Now observe that
    $\P{N(t)=k} = \P{A_k \leq t} - \P{A_{k+1} \leq t}$.  Using the
    density of $A_{k+1}$ as obtained previously and applying partial
    integration leads to
\begin{align*}
\P{A_{k+1} \leq t} 
&= \lambda \int_0^t \frac{(\lambda s)^{k}}{k!}e^{-\lambda s}\, \d s \\
&= \lambda \frac{(\lambda s)^{k}}{k!}\frac{e^{-\lambda s}}{-\lambda} \Big|_{0}^t + \lambda \int_0^t \frac{(\lambda s)^{k-1}}{(k-1)!}e^{-\lambda s}\, \d s \\
&= - \frac{(\lambda t)^{k}}{k!} e^{-\lambda t} + \P{A_k \leq t}.
\end{align*}
We are done.
    \end{solution}
\end{exercise}

We now introduce another fundamental concept.
A random variable $X$ is called \recall{memoryless} when it satisfies
\begin{equation*}
  \P{X > t+h | X>t} = \P{X>h}.
\end{equation*}
In words, the probability that $X$ is larger than some time $t+h$, conditional on it being larger than a time~$t$, is equal to the probability that $X$ is larger than $h$.

\begin{exercise}[\faCalculator]
  Show that $X\sim \Exp(\lambda)$ is  memoryless.  

  \begin{hint}
Condition on the event ${X>t}$. 
  \end{hint}
  \begin{solution}
By  the definition of conditional probability
\begin{equation*}
  \P{X>t+h|X>t} = \frac{\P{X>t+h, X>t}}{\P{X>t}} = \frac{\P{X>t+h}}{\P{X>t}} = \frac{e^{-\lambda(t+h)}}{e^{-\lambda t}} = e^{-\lambda h} = \P{X>h}.
\end{equation*}

Thus, no matter how long we have been waiting for the next arrival to occur, the probability that it will occur in the next $h$ seconds remains the same.


This property
seems to be vindicated also in practice: suppose that a patient with a
broken arm just arrived at the emergency room of a hospital, what does
that tell us about the time the next patient will be brought in? Not
much, as most of us will agree.


As an aside,  it can be shown that only exponential random variables have the
memoryless property. The proof of this fact requires quite some work;
we refer the reader to the literature if s/he wants to check this, see
e.g. \citet[Appendix 3]{yushkevich69:_markov_proces}.
  \end{solution}
\end{exercise}


\begin{exercise}[\faCalculator]\label{ex:10}
  If $X\sim\Exp(\lambda)$ and $S\sim\Exp(\mu)$, and $X$ and $S$ are
  independent, show that 
  \begin{equation*}
Z=\min\{X,S\}\sim\Exp(\lambda+\mu),
  \end{equation*}
hence $\E Z = (\lambda+\mu)^{-1}$.
\begin{hint}
Use that if $Z=\min\{X, S\}>x$,  it then must be that $X>x$
  and $S>x$. Then use independence of $X$ and $S$.
\end{hint}
  \begin{solution}
Use that $X$ and $S$ are independent to get
    \begin{equation*}
      \begin{split}
      \P{Z>x} 
&= \P{\min\{X,S\}>x} = \P{X>x\text{ and } S>x} = \P{X>x}\P{S>x} \\
&= e^{-\lambda x} e^{-\mu x} = e^{-(\lambda+\mu)x}.
      \end{split}
    \end{equation*}
  \end{solution}
\end{exercise}

\begin{exercise}[\faCalculator]\label{ex:3}
   If $X\sim \Exp(\lambda)$, $S\sim\Exp(\mu)$, and $X$ and $S$ are independent, show that 
    \begin{equation*}
      \P{X\leq S} = \frac{\lambda}{\lambda+\mu}.
    \end{equation*}
    \begin{hint}
     Define the joint distribution of $X$ and $S$ and carry out
      the computations, or use conditioning, or use the result of the
      previous exercise.
    \end{hint}
\begin{solution}
There is more than one way to show that $\P{X\leq S} = \lambda/(\lambda+\mu)$.  

Observe first that $X$ and $S$, being
exponentially distributed, have a density. Moreover, as they are
independent,  the joint density takes the form
\begin{equation*}
f_{X,S}(x,y) = f_X(x)f_S(y) = \lambda \mu e^{-\lambda x} e^{-\mu
  y}.
\end{equation*}
With this,
\begin{align*}
    \P{X\leq S} 
&= \E{\1{X\leq S}} \\
&= \int_0^\infty \int_0^\infty \1{x\leq y} f_{X,S}(x,y)\, \d y\,\d x\\
&= \lambda \mu \int_0^\infty \int_0^\infty \1{x\leq y} e^{-\lambda x} e^{-\mu y} \, \d y\,\d x\\
&= \lambda \mu \int_0^\infty e^{-\mu y} \int_0^y e^{-\lambda x}\, \d x \, \d y \\
&= \mu \int_0^\infty e^{-\mu y} (1-e^{-\lambda y})\,\d y\\
&= \mu \int_0^\infty (e^{-\mu y} - e^{-(\lambda +\mu)y} ) \,\d y\\
&= 1 - \frac{\mu}{\lambda + \mu} \\
&= \frac{\lambda}{\lambda + \mu}.
\end{align*}

\end{solution}
\end{exercise}



\begin{exercise}[\faCalculator]\label{ex:49}
  A machine serves two types of jobs. The processing time of jobs of
  type $i$, $i=1,2$, is exponentially distributed with parameter
  $\mu_i$. The type $T$ of a job is random and independent of anything
  else, and such that $\P{T=1} = p = 1-q = 1-\P{T=2}$. (An example
  is a desk serving men and women, both requiring different average
  service times, and $p$ is the probability that the customer in
  service is a man.)  Show that  the expected processing time  and  variance are given by
\begin{align*}
  \E X &= p \E{X_1}  + q \E{X_2} \\
\V X &= p \V{X_1} + q \V{X_2} + pq(\E{X_1} - \E{X_2})^2.
  \end{align*}
Interestingly, we see that even if $\V{X_1} = \V{X_2} = 0$, $\V X > 0$
if $\E{X_1} \neq \E{X_2}$. Bear this in mind; we will use these ideas
later when we discuss the effects of failures on the variance of
service times of jobs.
\begin{hint}
    Let $X$ be the processing (or service) time at the server, and
    $X_i$ the service time of a type $i$ job. Then, 
    \begin{equation*}
      X = \1{T=1} X_1 + \1{T=2} X_2,
    \end{equation*}
    where $\1{\cdot}$ is the indicator function, that is, $\1{A}=1$ if the
    event $A$ is true, and $\1{A}=0$ if $A$ is not true.   
\end{hint}
  \begin{solution}
With the hint, 
\begin{align*}
  \E X 
&= \E{\1{T=1} X_1} + \E{\1{T=2} X_2} \\
&= \E{\1{T=1}} \E{ X_1} + \E{\1{T=2}} \E{X_2}, \text{ by the independence of $T$}, \\
&= \P{T=1} /\mu_1 + \P{T=2}/ \mu_2 \\
&= p /\mu_1 + q/ \mu_2 \\
&= p \E{X_1}  + q \E{X_2}.
\end{align*}
(The next derivation may seem a bit long, but the algebra is
standard. I include all steps so that you don't have to use pen and
paper yourself if you want to check the result.) Next, using that
\begin{equation*}
\1{T=1}\1{T=2} = 0 \text{ and } \1{T=1}^2 = \1{T=1},
\end{equation*}
we get
\begin{align*}
  \V X 
&= \E{X^2} - (\E X)^2 \\
&= \E{\left(\1{T=1} X_1 + \1{T=2} X_2\right)^2} - \left(\frac{p}{\mu_1}+\frac{q}{\mu_2}\right)^2 \\
&= \E{\1{T=1} X_1^2 + \1{T=2} X_2^2} - \left(\frac{p}{\mu_1}+\frac{q}{\mu_2}\right)^2 \\ 
&= p \E{X_1^2} + q \E{X_2^2} - \left(\frac{p}{\mu_1}+\frac{q}{\mu_2}\right)^2 \\ 
&= p \V{X_1} + p (\E{X_1})^2 + q \V{X_2} + q(\E{ X_2})^2 - \left(\frac{p}{\mu_1}+\frac{q}{\mu_2}\right)^2 \\ 
&= p \V{X_1} + \frac{p}{\mu_1^2} + q \V{X_2} + \frac{q}{\mu_2^2} - \left(\frac{p}{\mu_1}+\frac{q}{\mu_2}\right)^2 \\ 
&= p \V{X_1} + q \V{X_2}
+ \frac{p}{\mu_1^2} + \frac{q}{\mu_2^2}
- \frac{p^2}{\mu_1^2}-\frac{q^2}{\mu_2^2}  -\frac{2pq}{\mu_1\mu_2}\\ 
&= p \V{X_1} + q \V{X_2}
+ \frac{p(1-p)}{\mu_1^2} + \frac{q(1-q)}{\mu_2^2}
-\frac{2pq}{\mu_1\mu_2}\\ 
&= p \V{X_1} + q \V{X_2}
+ \frac{pq}{\mu_1^2} + \frac{qp}{\mu_2^2}
-\frac{2pq}{\mu_1\mu_2}\\ 
&= p \V{X_1} + q \V{X_2}
+ pq(\E{X_1} - \E{X_2})^2.
\end{align*}
\end{solution}
\end{exercise}




\opt{solutionfiles}{
\Closesolutionfile{hint}
\Closesolutionfile{ans}
\subsection*{Hints}
\input{hint}
\subsection*{Solutions}
\input{ans}
}

%\clearpage  


%%% Local Variables:
%%% mode: latex
%%% TeX-master: "../queueing_book"
%%% End:
