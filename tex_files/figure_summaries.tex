\section{Graphical Summaries}

We finish this chapter with providing two summaries in graphical form to clarify how all concepts developed in this chapter relate.

\begin{figure}[hp]
  \centering
  \begin{tikzpicture}[node distance = 2.5cm]
\tikzset{
    %Define standard arrow tip
    >=stealth',
    %Define style for boxes
    % Define arrow style
    pil/.style={
           ->,
           thick,
           shorten <=2pt,
           shorten >=2pt,}
}

% Define block styles
\tikzstyle{block} = [rectangle, draw,text centered, rounded corners, minimum height=3em]

    % nodes
    \node [block, fill=red!50] (X_k) {$\{X_k\}$};
    \node [block, right=2.5cm of  X_k,fill=red!50] (A_k) {$\{A_k\}$}
    edge[pil,bend left=45] node[below] {$X_k := A_k - A_{k-1}$} (X_k)
    edge[pil,<-, bend right=45] node[above] {$A_k := A_{k-1} + X_{k}$} (X_k); 
    \node [block, right=2.5cm of  A_k ] (A_t) {$\{A(t)\}$}
    edge[pil,bend left=45] node[below] {$A_k := \inf\{t: A(t)\geq k\}$} (A_k)
    edge[pil,<-, bend right=45] node[above] {$A(t) := \max\{k: A_k\leq t\}$} (A_k); 
    \node [block, below=2cm of  X_k ] (EX) {$\frac 1n \sum_{k=1}^n X_k \to \E X$}
    edge[pil, <-] (X_k);
    \node [block, below=2cm of  A_t ] (lambda) {$\frac{A(t)}t  \to \lambda$}
    edge[pil, <-] (A_t);
    \node [block, below=2cm of  A_k ] {$\E X = \lambda^{-1}$}
    edge[pil, <-] (EX)
    edge[pil, <-] (lambda);

    \node[below=1cm of lambda] (dummy) {}; 

    \node [block, below=1cm of EX, fill=red!50] (S_k) {$\{S_k\}$};
    \node [block, right=2.5cm of  S_k ] (U_k) {$\{U_k\}$}
    edge[pil,<->] (S_k);
    \node [block, right=2.5cm of  U_k ] (U_t) {$\{U(t)\}$}
    edge[pil,<->]  (U_k);
    \node [block, below=1cm of  S_k ] (ES) {$\frac 1n \sum_{k=1}^n S_k \to \E S$}
    edge[pil, <-] (S_k);
    \node [block, below=1cm of  U_t ] (mu) {$\frac{U(t)}t  \to \mu$}
    edge[pil, <-] (U_t);
    \node [block, below=1cm of  U_k ] {$\E S = \mu^{-1}$}
    edge[pil, <-] (ES)
    edge[pil, <-] (mu);

    \node [block, right=of dummy] {Stability:\newline $\lambda < \mu$}
edge[pil,bend right=25, <-] (lambda.east)
edge[pil,bend left=25, <-] (mu.east);

    \node[block, below=1cm of  ES, fill=red!50 ] (W_k) {$W_{k}=\max\{W_{k-1} - X_k,0\} +S_k$};
    %edge[pil, bend left=45,<-]  (X_k)
    %edge[pil, bend right=55,<-]  (S_k.west);
    \draw[->] (S_k.west) [out=180, in=110] to  (W_k.north west);
    \draw[->] (X_k.west) [out=230, in=110] to  (W_k.north west);
    \node[block, right=1cm of  W_k, fill=red!50 ] (D_k) {$D_{k}=A_k + W_{k}$} 
    edge[pil,bend right=25, <-]  (A_k)
    edge[pil,<-]  (W_k);
    \node[block, right=1cm of  D_k ] (D_t) {$D(t)=\max\{k: D_k\leq t\}$} 
    edge[pil, <-]  (D_k);

    \node[block, below=1cm of  W_k, fill=blue!40] (W) {$\frac1n\sum_{k=1}^n W_k \to \E W$} 
    edge[pil, <-]  (W_k);
    \node[block, below=1cm of  D_t] (Q_t) {$L(t) := A(t) - D(t)$} 
    edge[pil, <-]  (D_t);
    \draw[->] (A_t.east) [out=20, in = 40] to  (Q_t);
    %edge[pil, bend right=95, <-]  (A_t.north);

    \node[block, right=1cm of  D_t] (delta) {$\frac{D(t)}t \to \delta$} 
    edge[pil, <-]  (D_t);


    \node[block, below=1cm of  D_k, fill=blue!40] (L) {$\frac 1 t \int_0^t L(s)\,\d s \to \E L$} 
    edge[pil, <-]  (Q_t);
    % \node[block, below=1cm of  L] (Little) {$\E L = \lambda \E W$} 
    % edge[pil, <-]  (L)
    % edge[pil, <-]  (W);

    \node[block, right=1cm of  Q_t] (hoi) {$\delta \leq \lambda$} 
    edge[pil, <-]  (delta)
    edge[pil, bend right=10, <-]  (lambda)
    edge[pil, bend left = 30, <-]  node[below] {$L(t)>0$} (Q_t);

    \node[block, below=1cm of W, fill=blue!40] (PW) {$\frac 1n \sum_{k=1}^n \1{W_k \leq w} \to \P{W\leq w}$}
    edge[pil, <-] (W);

    \node[block, below=1cm of Q_t, fill=blue!40] (PL) {$\frac 1t \int_{0}^t \1{L(s) \leq l} \to \P{L\leq l}$}
    edge[pil, <-] (Q_t);

% \node[block, below=1cm of PW, fill=gray!40] (PM) {Performance measures};
% \node[block, below=1cm of PM, fill=gray!40] {$G/G/1$ Construction \& simulation};

    % \node[block, below=1cm of L, text width=2cm, fill=gray!40] (perf) {Performance measures}
    % edge[pil, ->] (L)
    % edge[pil, ->] (PL)
    % edge[pil, ->] (PW)
    % edge[pil, ->] (W);

  \end{tikzpicture}  

  \caption{Here we sketch the relations between the construction of
    the $G/G/1$ queue from the primary data, i.e., the inter-arrival
    times $\{X_k; k\geq 0\}$ and the service times $\{S_k; k\geq 0\}$,
    and different performance measures. 
% The performance measures are
%     shown in \protect\tikz \protect\node[fill=blue!40] {blue};, the
%     essential components for the construction of the $G/G/1$ are shown
%     in \protect\tikz \protect\node[fill=red!40] {red}.
} 
    \label{fig:constructiongg1}

\end{figure}





\begin{figure}[p]
  \centering

  \begin{tikzpicture}[node distance = 2.5cm]

\tikzset{
    %Define standard arrow tip
    >=stealth',
    %Define style for boxes
    % Define arrow style
    pil/.style={
           ->,
           thick,
           shorten <=2pt,
           shorten >=2pt,}
}
\tikzstyle{block} = [rectangle, draw,text centered, rounded corners, minimum height=3em]

    % nodes
    \node [block, text width=5.5cm, align=center] (level) {LEVEL CROSSING: Counting up- and down-crossings};
\node[block, below=1cm of level] (An) {$|A(n,t)-D(n,t)|\leq 1$}
edge[pil,<-] (level); 

\node[block, left=1.5cm of An] (A) {$A(t)-D(t)=L(t)$} 
edge[pil,<-] (level); 

\node[block, right=1.5cm of An] (Anm) {$|A(m,n,t)-D(n,t)|\leq 1$}
edge[pil,<-] (level); 

\node[block, below=1cm of A] (At) {$\frac{A(t)}t \approx \frac{D(t)}t$ if $\frac{L(t)}t \to 0$} 
edge[pil,<-] (A); 

\node[block, below=1.5cm of At] (lambda) {$\lambda=\delta$} 
edge[pil,<-] node[fill=white] {$t\to\infty$} (At); 

\node[block, below=1cm of An] (AnDn) {$\frac{A(n,t)}t\approx\frac{D(n,t)}t$}
edge[pil,<-] (An); 

\node[block, below=1.5cm of AnDn] (AnDn2) {$\frac{A(n,t)}{Y(n,t)}\frac{Y(n,t)}t\approx\frac{D(n,t)}{Y(n+1)}\frac{Y(n+1)}t$}
edge[pil,<-] (AnDn); 

\node[block, below=1.5cm of AnDn2, text width=4cm] (lp) {Level Crossing: \\
$\lambda(n)p(n) = \mu(n+1)p(n+1)$}
edge[pil,<-] node[fill=white] {$t\to\infty$} (AnDn2); 

\node[block, below=1cm of lp, text width=3cm] (poisson) {Poisson: \\
$\lambda=\lambda(n)$, \\
$\mu=\mu(n)$}
edge[pil,<-] (lp);

\node[block, below=1cm of poisson, text width=4cm] (mm1) {$M/M/1$, $M/M/c$, $M/M/c/k$, \ldots} edge[pil,<-] (poisson);
;

\node[block, right=0.6cm of lp, text width=5cm, align=center] (batch) {Recursion: \\ $\lambda\sum_{m=0}^nG(n-m)p(m) = \mu(n+1)p(n+1)$}
edge[pil,<-] node[fill=white] {$t\to\infty$} (Anm); 

\node[block, right=2.3cm of mm1] (batch2) {$M^X/M/1$}
edge[pil,<-]  (poisson)
edge[pil,<-]  (batch); 


\node[block, below=1cm of mm1, text width=4.5cm] (perf) {Performance
  measures:
$\E L= \sum_{n=0}^\infty n p(n)$, $\P{L\geq m}$, \ldots} 
edge[pil,<-] (mm1)
edge[pil,<-] (batch2);

\node[block, below=1.5cm of lambda] (pasta1) {$\frac{A(t)}t\frac{A(n,t)}{A(t)} = \frac{A(n,t)}{Y(n,t)}\frac{Y(n,t)}t$} 
edge[pil,<-] (AnDn2)
edge[pil,<-,bend left=20] (At.south west)
;

\node[block, below=1.5cm of pasta1] (pasta2) {$\lambda \pi(n) = \lambda(n)p(n)$} 
edge[pil,<-] node[fill=white] {$t\to\infty$} (pasta1);

\node[block, below=1.5cm of pasta2, text width=3cm] (pasta3) {PASTA: $\pi(n) = p(n)$} 
edge[pil,<-] (poisson)
edge[pil,<-] (pasta2)
edge[pil,->] (perf);

\end{tikzpicture}
  \caption{With level crossing arguments we can derive a number of
    useful relations. This figure presents an overview of these
    relations that we derive in this and the next sections.}
\label{fig:summaries}
\end{figure}


%%% Local Variables:
%%% mode: latex
%%% TeX-master: "../companion"
%%% End:
