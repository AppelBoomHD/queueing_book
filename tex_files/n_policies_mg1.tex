% arara: pdflatex
% arara: clean: { extensions: [ aux, blg, idx, ilg, ind, log, out, pytxcode, rel, toc ] }
% !arara: clean: { files: [ ans.tex, hint.tex] }

\documentclass[companion]{subfiles}

\begin{document}

\section{N-policies for the $M/G/1$ queue}
\label{sec:n-policies-mg1}

Interestingly, we can extend the analysis of~\cref{sec:distr-queue-length} and~\cref{sec:n-policies} to compute the average cost of the $M/G/1$ queue under an $N$-policy.
Thus, rather than exponentially distributed service times, we now consider general service times.
For the rest, the model and the problem is the same as in~\cref{sec:n-policies}. 

\opt{solutionfiles}
{\subsection*{Theory and Exercises}
\Opensolutionfile{hint}
\Opensolutionfile{ans}
}


Similar to~\cref{sec:distr-queue-length}, we consider the $M/G/1$ queueing process at departure moments.
It is easy to find an expression for the time to clear the queue right after the departure of a job that leaves $q$ jobs behind.
Analogous to~\cref{eq:92} and using the definition of $f(k)=\P{Y=k}$ as the probability that $Y=k$ jobs arrive during an arbitrary service time, we see that $T(q)$ must satisfy the relation
\begin{equation*}
  T(q) = \sum_{k=0}^\infty f(k) T(q+k-1) + \E S.
\end{equation*}
To see this, observe that when $k=0$ the next level is $q-1$ so that it takes $T(q-1)$ time to hit level~$0$.
When $k=1$, the level returns to $q$ after the departure of the job currently in service; consequently, we have not made any progress, hence we still need $T(q)$ units of time,  and so on.
Similar to the reasoning in~\cref{sec:n-policies}, we guess that $T(q) = a q + b$. Since $T(0)=0$, we already have $b=0$. 

\begin{exercise}\clabel{ex:nmg3}
Substitute $T(q) = a q$ in the above recursion for $T(q)$, and solve for $a$ to obtain
\begin{equation*}
  T(q) = \frac{\E S}{1- \E Y}q = \frac{\E S}{1-\lambda \E S}q.
\end{equation*}
\begin{hint}
  Use that $\sum_{k=0}^\infty k f(k) = \E Y$. Then use~\cref{ex:mg1-3}. Realize also that $\sum_k f(k) = 1$. 
\end{hint}
\begin{solution}
  \begin{align*}
    a q = \sum_{k=0}^\infty a (q+k-1) f(k) + \E S \implies a q = a q + a \E Y - a + \E S.
  \end{align*}
The terms linear in $q$ cancel. And thus, $0=a\E Y - a + \E S$. The result follows right away.
\end{solution}
\end{exercise}


Let $V(q)$ be the cost to clear the system right after a departure that leaves $q$ jobs behind.
Thus, $V(0)=0$.
Again, analogous~\cref{sec:distr-queue-length} and the derivation above for $T(q)$, we see that $V(q)$ must satisfy the relation
\begin{equation}
  \label{eq:98}
  V(q) = \sum_{k=0}^\infty f(k) V(q+k-1) + H(q), 
\end{equation}
where $H(q)$ is the queueing, or holding, cost, which we will determine next.

The queueing cost $H(q)$ consists of two components.
The first is the cost to keep $q$ jobs in the system while there is a job in service.
Clearly, the expected cost of this is $h q \E S$.
The second component is the cost of new jobs that arrive during the service.
While this is slightly harder to determine, we can combine the ideas underlying the derivation in~\cref{eq:99} and \cref{ex:p-35}  and \cref{ex:3}.
Specifically, assume it is given that the service time is~$s$.
Then the expected total queueing cost $U(s)$ of the new arrivals must satisfy for some $0<\delta\ll 1$
\begin{equation*}
  U(s) = U(s-\delta) + (1-\lambda \delta)\cdot 0 + \lambda \delta h s + o(\delta).
\end{equation*}
This follows because during the interval $[0,\delta]$ a job arrives with probability $\lambda \delta$ and stays $s$ time units in the system, while with probability $1-\lambda \delta$ no job arrives, and the cost of this is~$0$.
By the next problem, 
\begin{equation*}
  U(s) = \lambda h \frac{s^2}2. 
\end{equation*}

\begin{exercise}\clabel{ex:nmg4}
Assuming that $U(\cdot)$ is differentiable, we can use that $(U(s) - U(s-\delta))/\delta \approx U'(s)$.
Use this in the above to arrive at the differential equation $U'(s) = \lambda h s$ that $U$ must satisfy with condition $U(0)=0$ (why is this so?).
Solve this DE to conclude that $U(s)$ is as given above. 
\begin{solution}
Is it plain algebra:
\begin{align*}
  U(s)&=  U(s-\delta) + (1-\lambda \delta)\cdot 0 + \lambda \delta h s + o(\delta) &\implies\\
  (U(s) -U(s-\delta))/\delta &= \lambda h s + o(\delta) & \implies \\
  U'(s)&= \lambda h s, \quad\text{as } \delta \to 0.
\end{align*}
Integrating  both sides with respect to $s$ gives $U(s) - U(0) = \lambda h s^2/2$. As $U(0)=0$, we obtain the final result.
\end{solution}
\end{exercise}

Since $U(s)$ is the expected total cost given that the service time is $s$, it follows from~\cref{ex:mg1-3} that $\E{U(S)} =  \lambda h \E{S^2}/2$ is the expected queueing cost of the new jobs that arrive during the service.
By combining the first and second component of $H(q)$, we obtain
\begin{equation*}
  H(q) = h q \E S + \frac 12 \lambda h \E{S^2}.
\end{equation*}
To solve~\cref{eq:98}, note that as in~\cref{eq:93}, the cost $H(q)$ has a term linear in~$q$ and a constant term.
Thus, once again, we guess that $V(q)$ is a quadratic function in $q$.
Then we substitute this form into~\cref{eq:98}, assemble terms with the same power in $q$, and try to solve for the coefficients in $aq^2 + bq+c$.
If we succeed, the solution we found must be correct, as a difference equation of the type~\cref{eq:98} can have just one solution once the boundary conditions are given.
Again, $V(q)=0 \implies c = 0$.
In fact, observe that we have all elements to compute the costs; except for (quite a bit of) algebra, we are done!

\begin{extra}
Assemble all terms of $V(q) = \sum_k V(q+k-1) + H(q)$ by their power in $q$ to obtain:
\begin{align*}
  a q^2 &= a q^2, \\
  b q &= 2a q \E Y - 2a q + b q + h q \E S,\\
  0 &= a \E{Y^2} - 2a \E Y + a + b \E Y - b + \frac 12 \lambda h \E{S^2}.
\end{align*}
\begin{hint}
  Just expand the left- and right-hand sides, and equate the terms with the same power in $q$.
\end{hint}
\begin{solution}
  Just expand
  \begin{equation*}
    \sum_{k=0}^\infty f(k)[ a(q+k-1)^2 + b(q+k-1)] + H(q),
  \end{equation*}
  and use that $\sum_k f(k) k = \E Y$ and $\sum_k f(k) k^2 = \E{Y^2}$. 
\end{solution}
\end{extra}

\begin{extra}
Simplify the above to obtain
\begin{align*}
  a &= \frac{h \E S}{2(1-\E Y)}, \\
  b(1-\E Y) &= a(\E{Y^2} - 2 \E Y + 1) + \frac 12 h \lambda \E{S^2}.
\end{align*}
\begin{solution}
  The first equation in the previous equation is superfluous.
  In the second we can cancel $bq$ at both sides.
  Simplifying gives us the expression for $a$.
  Similar for the third equation.
\end{solution}
\end{extra}

\begin{extra}
  Show that $\E{Y^2} = \lambda^2 \E{S^2} + \lambda \E S$. 
\begin{solution}
This follows directly from~\cref{ex:mg1-3}.
\end{solution}
\end{extra}

\begin{exercise}\clabel{ex:nm-2}
Show that 
\begin{align*}
  a &= \frac{h \E S}{2(1-\rho)}, &
  b &= \frac{h \E S}{2(1-\rho)^2} \left( 1+ \rho C_s^2\right).
\end{align*}
(In a sense, this is trivial, as it is just algebra, but it is hard to get the details right.)
\begin{solution}
  We already derived $a$, except that we replaced $1-\E Y= 1-\lambda \E S = 1-\rho$. 
For $b$, and using the expressions for $\E Y$ and $\E{Y^2}$, 
\begin{align*}
b(1-\E Y) &= a(\E{Y^2} - 2 \E Y + 1) + \frac 12 h \lambda \E{S^2} \\
&= \frac{h \E S}{2(1-\E Y)} (\E{Y^2} - 2 \E Y + 1) + \frac 12 h \lambda \E{S^2} \\
&= \frac{h \E S}{2(1-\lambda \E S)} \left(\lambda^2 \E{S^2} + \lambda \E S - 2 \lambda \E S + 1 +  \lambda \E{S^2}\frac{1-\lambda \E S}{\E S}\right) \\
&= \frac{h \E S}{2(1-\lambda \E S)} \left(\lambda^2 \E{S^2} - \lambda \E S + 1 +  \frac{\lambda \E{S^2}}{\E S} - \lambda^2 \E{S^2}\right) \\
&= \frac{h \E S}{2(1-\lambda \E S)} \left(1+ \frac{\lambda \E{S^2}}{\E S }  - \lambda \E S \right) \\
&= \frac{h \E S}{2(1-\lambda \E S)} \left( 1+ \frac{\lambda \E{S^2}}{\E S }  - \lambda \frac{(\E S)^2}{\E S}\right) \\
&= \frac{h \E S}{2(1-\lambda \E S)} \left( 1+ \lambda \frac{\V{S}}{\E S }\right)\\
&= \frac{h \E S}{2(1-\lambda \E S)} \left( 1+ \lambda \frac{\V{S}}{(\E S)^2 } \E S\right)\\
&= \frac{h \E S}{2(1-\lambda \E S)} \left( 1+ \rho C_s^2\right).
\end{align*}
Divide now both sides by $1-\E Y$. 
\end{solution}
\end{exercise}

\begin{extra}
We need to check our work.  For this reason, show that the results of~\cref{ex:nm-2} reduce to those of~\cref{ex:nmm-2} for the $M/M/1$ queue. 
\begin{solution}
  For $a$, multiply the numerator and denominator by $\mu=1/ \E S$.
  For $b$, multiply by $\mu^2 = 1/(\E S)^2$, use that $C_s^1=1$ because the service times are exponentially distributed, and note that
  \begin{equation*}
    \frac{1+\rho}{1-\rho} = \frac{\mu + \lambda}{\mu-\lambda}.
  \end{equation*}
\end{solution}
\end{extra}

\begin{exercise}\clabel{ex:nmg5}
As in~\cref{sec:n-policies}, show that we can obtain the Pollaczek-Khinchine equation~\cref{eq:710} with the results we have obtained up to now. 
\begin{hint}
  What is the cycle time $C(1)$ when the service starts with one job in the system? 
\end{hint}
\begin{solution}
  Right after the arrival of the first job that sees an empty system, the service of this job starts.
  Thus, $C(1) = 1/\lambda + T(1)$ is the expected duration of a cycle, and $V(1)$ the expected queueing cost of once cycle.
  The rest is algebra:
  \begin{align*}
    \frac{V(1)}{1/\lambda + T(1)}
    &= \frac{a + b}{1/\lambda + \E S/(1-\lambda \E S)} \\
    &= \frac{h} 2 \frac{\E S}{1-\rho} \frac{1 + (1+\rho C_s^2 )/(1-\rho)}{1/\lambda + \E S /(1-\rho)} \\
    &= \frac{h} 2 \frac{\E S}{1-\rho} \frac{1-\rho + 1 +\rho C_s^2}{(1-\rho)/\lambda + \E S} \\
    &= \frac{h} 2 \frac{\E S}{1-\rho} \frac{1-\rho + 1 +\rho C_s^2}{1/\lambda } \\
    &= \frac{h} 2 \frac{\rho}{1-\rho} (1-\rho + 1 +\rho C_s^2) \\
    &= \frac{h} 2 \frac{\rho}{1-\rho} (2-2\rho + \rho +\rho C_s^2) \\
    &= h \rho + \frac h 2 \frac{\rho^2}{1-\rho} (1+ C_s^2).
  \end{align*}
It checks!  
\end{solution}
\end{exercise}

For the $M/G/1$ queue it is evident that the expected queueing cost while the server is idle is also given by~\cref{eq:99}.
With this, and noting that there is a cost~$K$ to switch on the server, it follows from~\cref{ex:n-mg3} that the long-run time-average average costs are equal to
\begin{equation}
  \label{eq:100}
    \frac{V(N) + K + W(N)}{C(N)}
    = \frac h 2 \frac{\rho^2}{1-\rho} (1+ C_s^2) + h \rho + h \frac{N-1}2 + K \frac{\lambda(1-\rho)}N.
\end{equation}
Note that, if $N=1$ and $K=0$, this reduces to $h \E L$, as it should by our work above.
Finally, minimizing over $N$ gives that
\begin{equation*}
  N^* \approx \sqrt{\frac{2\lambda(1-\rho)K}{h}}.
\end{equation*}

\begin{remark}
  The expression for $N^*$ is a famous result in inventory theory: it is the optimal order size for a machine that produces items with holding cost $h$ per item per unit time, a cost $K$ to switch on the machine, demand arrives at rate $\lambda$, and the machine produces at rate $\mu=1/\E S$.
  $N^*$ is known as the \recall{Economic Production Quantity (EPQ)}.
  Taking $\mu\to\infty$, it reduces to the \recall{Economic Order Quantity (EPQ)}.
\end{remark}

\begin{exercise}\clabel{ex:n-mg3}
Derive~\cref{eq:100}.
\begin{solution}
  Note first that $C(N) = N(1/\lambda + \E S / (1-\rho)) = N/(\lambda(1-\rho))$. Then,
  \begin{align*}
    \frac{V(N) + K + W(N)}{C(N)}
    &= \left(aN^2 + bN + K + h N(N-1)/2 \lambda\right) \frac{\lambda(1-\rho)}N \\
    &= \frac h 2 \rho N  + \frac h 2 \frac \rho{1-\rho} (1+\rho C_s^2) + \frac h 2 (N-1)(1-\rho) + K \frac{\lambda(1-\rho)}N \\
    &= \frac h 2 \frac \rho{1-\rho} (1+\rho C_s^2) + \frac h 2 (N-1 + \rho) + K \frac{\lambda(1-\rho)}N \\
    &= \frac h 2 \frac \rho{1-\rho} (\rho + \rho C_s^2 + 1 - \rho) + \frac h 2 (N-1 + \rho) + K \frac{\lambda(1-\rho)}N \\
    &= \frac h 2 \frac{\rho^2}{1-\rho} (1+ C_s^2) +\frac h 2 \rho + \frac h 2 (N-1 + \rho) + K \frac{\lambda(1-\rho)}N \\
    &= \frac h 2 \frac{\rho^2}{1-\rho} (1+ C_s^2) + h \rho + h \frac{N-1}2 + K \frac{\lambda(1-\rho)}N.
  \end{align*}
%With the above expressions for $a$ and $b$ the result follows immediately. 
\end{solution}
\end{exercise}


\begin{extra}
  Why does the load $\rho$ not depend on $N$?
\begin{solution}
  The total number $A(t)$ of job that arrive during $[0,t]$ does not depend on $N$.
  Since the system does not explode, we must have that $\delta = \lambda$, hence the number of jobs served per unit time is, on the long run, what arrives.
  And thus, by~\cref{eq:86}, $\rho$ does not depend on $N$.
  But, why is this system rate-stable?
  Here we provide an intuitive explanation.
  When we switch on the server, the queue `drains' on average at rate $\mu-\lambda$.
  Consequently, no matter how large $N$, the expected time of empty the system is finite.
  And, whenever the system is empty, the stochastic process restarts.
  As such cycles start over and over again, the queue length can never `escape to infinity'.
\end{solution}
\end{extra}

\opt{solutionfiles}
{\Closesolutionfile{hint}
\Closesolutionfile{ans}
\subsection*{Hints}
\input{hint}
\subsection*{Solutions}
\input{ans}
}


\end{document}


%%% Local Variables:
%%% mode: latex
%%% TeX-master: t
%%% End:
