\section{Applications and Useful Identities}
\label{sec:some-usef-ident}


\subsection*{Theory and Exercises}

\Opensolutionfile{hint}
\Opensolutionfile{ans}
With the PASTA property and Little's law we can derive a number of
useful and simple results for the $M/G/1$ queue by which we can analyze a large number of practical queueing situations, see Exercise~\ref{exer: Hall} and below. In particular, we derive expressions for  the following performance measures: server utilization, i.e., $\rho$, average queue length and waiting times. We assume that jobs arrive in accordance to a Poisson process so that we can use the PASTA property.


The fraction of time the server is empty is $1-\rho = p(0)$. By PASTA,
$\pi(0)=p(0)$, hence the fraction of customers that enter an empty
system is also $1-\rho$. Similarly, the fraction of customers that find the server occupied at arrival is equal to the utilization $\rho$. 



Suppose that at $A_k$, i.e., the arrival epoch of the $k$th job, the
server is busy.  The \recall{remaining service time} $S_{r,k}$ as seen by job
$k$ is the time between $A_k$ and the departure epoch of the job in
service. If the server is free at $A_k$, we set $S_{r,k}=0$.  Define
the average \recall{remaining service time} as seen at arrival epochs as
\begin{equation*}
  \E{S_r} = \lim_{n\to\infty} \frac1n \sum_{k=1}^n S_{r,k},
\end{equation*}
provided this limit exists. 

\begin{exercise}
  It is an easy mistake to think that $\E{S_r} = \E S$ when service
  times are exponential. Why is this wrong?
  \begin{hint}
Realize again that $\E{S_r}$ includes the jobs that arrive at an empty system.
  \end{hint}
  \begin{solution}
    $\E{S_r \given S_r>0} = \E S$ for the $M/M/1$ queue, and
    $\E{S_r} = \rho \E{S_r \given S_r>0}$ for the $M/G/1$ queue, it
    follows that
  \begin{equation*}
 \E{S_r} = \rho \E{S_r\given S_r>0} = \rho \E S.
  \end{equation*}
  \end{solution}
\end{exercise}

\begin{exercise}($M/G/1$)
Use  the PASTA property to show that 
\begin{equation}\label{eq:37}
\E{S_r} =  \rho \E{S_r\given S_r>0}.
\end{equation}
\begin{hint}
Use Exercise~\ref{ex:28}.
a free server.  
\end{hint}
\begin{solution}
By the PASTA property we know that $\rho$ is  the probability to find the server busy upon arrival; hence $1- \rho$ is the probability that the server is idle upon arrival. Then,
\begin{equation*}
\E{S_r} =   \rho \E{S_r\given S_r >0} + (1-\rho) \E{S_r\given S_r = 0} = \rho \E{S_r\given S_r>0},
\end{equation*}
since, evidently, $\E{S_r\given S_r=0}=0$.  
\end{solution}
\end{exercise}

\begin{exercise}
  What would you guess for $\E{S_r\given S_r>0}$ for the $M/D/1$ queue? 
  \begin{solution}
    Since the service times are deterministic (and constant), I would
    guess that on average half of the service time remains at the
    moment a job arrives. If the service time is $D$ always, then
    $\E{S_r\given S_r>0} = D/2$.  
  \end{solution}
\end{exercise}

\begin{exercise}
  Try to derive relation~(\ref{eq:37}) with sample path
  arguments. 
  \begin{hint}
This requires some extra definitions, similar to
  $A(n,t)$ as defined in~\eqref{eq:19}.
  \end{hint}
  \begin{solution}
    This is, admittedly, not simple, so let us work in stages. Define
    $A(+,t)$ as the number of arrivals that see a job in service. As
    the server is only busy at time $t$ when $L(t)>0$, we define
    \begin{equation*}
      A(+,t) = \sum_{k=1}^\infty \1{A_k\leq t}\1{L(A_k-) > 0}.
    \end{equation*}
    Next, define $\tilde D_k = \min\{D_i: D_i \geq A_k\}$ as the first
    departure after the arrival time $A_k$ of job $k$. We need to
    consider two cases. If $\tilde D_k = D_k$ then, obviously, the
    first departure after $A_k$ coincides with the departure time of
    job $k$. This is only possible if $L(A_k-) = 0$. But then, it must
    be that the remaining service time of the job in service at time
    $A_k-$ is 0, as there is no job in service. Otherwise, if
    $L(A_k-)>0$ it must be that $\tilde D_k < D_k$, and, as a
    consequence, the remaining service time of the job in service at
    time $A_k-$ is equal to $\tilde D_k - A_k$. All in all, the total
    amount of remaining service times added up for all arrivals up to
    time $t$ can be written as
    \begin{equation*}
      S_r(t) = \sum_{k=1}^\infty (\tilde D_k - A_k) \1{A_k \leq t, L(A_k-) > 0}.
    \end{equation*}
    The remaining service time averaged over all arrivals up to
    time $t$ is therefore $S_r(t)/A(t)$. We can now rewrite this to
\begin{equation*}
  \frac{S_r(t)}{A(t)} = 
  \frac{A(+,t)}{A(t)} \frac{S_r(t)}{A(+,t)}, 
\end{equation*}
and interpret these fractions.  First, consider
\begin{equation*}
  \frac{A(+,t)}{A(t)} = 
\frac{\sum_{k=1}^\infty \1{A_k\leq t, L(A_k-) > 0}}{\sum_{k=1}^\infty \1{A_k\leq t}}.
\end{equation*}
This is the number of jobs up to time $t$ that see at least one job in
the system divided by the total number of arrivals up to time $t$. Thus, as $t\to\infty$, 
\begin{equation*}
  \frac{A(+,t)}{A(t)} \to \sum_{n=1}^\infty \pi(n) = 1-\pi(0).
\end{equation*}
With PASTA we can conclude that $1-\pi(0) = 1-p(0)= \rho$. Second,
\begin{equation*}
\frac{S_r(t)}{A(+,t)} 
= \frac{\sum_{k=1}^\infty (\tilde D_k - A_k) \1{A_k\leq t, L(A_k-) > 0}}{\sum_{k=1}^\infty \1{A_k\leq t, L(A_k-)>0}},
\end{equation*}
is the total amount of remaining service time up to time $t$ divided by
the number of arrivals up to time $t$ that see at least one job in the system. Thus, 
if the limit exists, we can \emph{define}
\begin{equation*}
\E{S_r \given S_r>0}  =\lim_{t\to\infty} \frac{S_r(t)}{A(+,t)}.
\end{equation*}
  \end{solution}
\end{exercise}

Next, consider the waiting time in queue.  
\begin{exercise}($M/G/1$)
Show  that the expected time in queue is
\begin{equation}\label{eq:24}
  \E{W_Q} = \E{S_r} + \E{L_Q} \E S.
\end{equation}
\begin{hint}
  What happens when you enter a queue? First you have to wait until the job in service (if there is any) completes, and then you have to wait for the queue to clear.
\end{hint}
\begin{solution}
It is evident that the expected waiting time for an arriving customer is the expected
remaining service time plus the expected time in queue. The expected
time in queue  must be equal to the expected number of
customers in queue at an arrival epoch times the expected service time
per customer, assuming that service times are i.i.d. If the arrival
process is Poisson, it follows from PASTA that the average number of
jobs in queue perceived by arriving customers is also the
\emph{time-average} number of jobs in queue~$\E{L_Q}$.  
\end{solution}
\end{exercise}
 
\begin{exercise}($M/G/1$)
  Use the previous problem and Little's law to derive that 
\begin{equation}\label{eq:35}
  \E{W_Q} = \frac{\E{S_r}}{1-\rho} = \frac{\rho}{1-\rho} \E{S_r\given S_r>0}.
\end{equation}
\begin{hint}
Use Little's law to see that $\E{L_Q} = \lambda \E{W_Q}$. Substitute this in~(\ref{eq:24}) and simplify.
\end{hint}
\begin{solution}
With  Little's law $\E{L_Q} = \lambda \E{W_Q}$. Using this,
\begin{equation*}
  \E{W_Q} = \E{S_r} + \lambda \E{W_Q} \E S  =\E{S_r} + \rho \E{W_Q},
\end{equation*}
since $\rho=\lambda \E S$. But this gives for the $M/G/1$ queue that
\begin{equation*}
  \E{W_Q} = \frac{\E{S_r}}{1-\rho} = \frac{\rho}{1-\rho} \E{S_r\given S_r>0},
\end{equation*}
where we use~(\ref{eq:37}) in the last equation.
\end{solution}
\end{exercise}

The average waiting time $\E W$ in the entire system, i.e., in queue
plus in service, becomes
\begin{equation*}
  \E W = \E{W_Q}+ \E S = \frac{\E{S_r}}{1-\rho} + \E S.
\end{equation*}

The situation can be significantly simplified for the $M/M/1$ queue as
then the service times are also exponential, hence memoryless,
implying that $\E{S_r\given S_r>0} = \E S$. 

\begin{exercise}($M/M/1$) Derive that
\begin{equation}\label{eq:ew}
  \E W = \frac{\E S}{1-\rho}.
\end{equation}
\begin{solution}
Using~(\ref{eq:37}),
\begin{equation*}
  \begin{split}
  \E W 
&= \E{W_Q}+ \E S = 
\frac{\E{S_r}}{1-\rho} + \E S \\
&= \frac{\rho \E S}{1-\rho} + \E S = \frac{\E S}{1-\rho}.
  \end{split}
\end{equation*}
\end{solution}
\end{exercise}

\begin{exercise}($M/M/1$) Use the PASTA property to conclude that
\begin{equation}\label{eq:61}
  \E W = \E L  \E S + \E S,
\end{equation}
and apply Little's law to find~(\ref{eq:ew}). 
\begin{hint}
Another way to derive the above result is to conclude from PASTA that
the expected number of jobs in the system at an arrival is $\E{L}$.
Since all these jobs require an expected service time $\E S$,
including the job in service by the memory-less property, the time in
queue is $\E L \E S$. The time in the system is then the waiting time
plus the service time. Next, use Little's law $\E L = \lambda \E W$.
\end{hint}
\begin{solution}
From the hint
\begin{equation}
  \E W = \E L  \E S + \E S = \lambda \E W \E S + \E S = \rho \E W  + \E S.
\end{equation}
Substituting Little's law $\E L = \lambda \E W$ and simplifying,
\begin{equation*}
\E W = \frac{\E S}{1-\rho}.
\end{equation*}
\end{solution}
\end{exercise}


\begin{exercise}
  Why is Eq.~(\ref{eq:61}) not true in general for the $M/G/1$ queue?
  \begin{hint}
Think about the consequences of memoryless service times.
  \end{hint}
  \begin{solution}
    Because the remaining service time of the job in service, provided
    there is a job in the system upon arrival, is not exponentially
    distributed in general. Only for the $M/M/1$ queue the service
    times are exponentially distributed, hence memoryless. And only
    when the service time is memoryless, the service time after an
    interruption is still exponential.
  \end{solution}
\end{exercise}

\begin{exercise}($M/M/1$)
Show that 
\begin{equation}\label{eq:wqes}
  \E{W_Q} = \E L \E S = \frac{\rho^2}{1-\rho} \frac 1 \lambda.
\end{equation}
  \begin{solution}
With Little's law:
\begin{equation}
  \E{W_Q} = \E L \E S = \lambda \E W \E S= \rho \E W = \frac{\rho}{1-\rho} \E S = \frac{\rho^2}{1-\rho} \frac 1 \lambda.
\end{equation}
  \end{solution}
\end{exercise}

Finally, we find expressions for  the average lengths.
\begin{exercise}($M/M/1$)
  Show that
  \begin{align*}
\E L &= \frac\rho{1-\rho}, & \E{L_Q} &= \frac{\rho^2}{1-\rho}, & \E{L_s} &= \rho.
  \end{align*}
  \begin{hint}
    Apply Little's law to the $\E{L_Q}$ and so on, and use the earlier expressions for $\E{W_Q}$ and so on.
  \end{hint}
  \begin{solution}
With Little's law, 
\begin{equation*}
  \begin{split}
\E L &= \lambda \E W = \frac{\lambda \E S}{1-\rho} = \frac\rho{1-\rho}, \\
  \E{L_Q} &= \lambda \E{W_Q} = \frac{\rho^2}{1-\rho}, \\
  \E{L_s} &= \E L - \E{L_Q} = \frac{\rho}{1-\rho} - \frac{\rho^2}{1-\rho} = \rho, 
      \end{split}
\end{equation*}
since the expected number of jobs in service $\E{L_s}$ is
equal to the expected number of busy servers.
  \end{solution}
\end{exercise}

\begin{exercise}($M/M/1$) Explain that
$\E{L_Q} = \sum_{n=1}^\infty (n-1)\pi(n)$, and  use this to find that $\E{L_Q}=\rho^2/(1-\rho)$.
  \begin{solution}
    The fraction of time the system contains $n$ jobs is $\pi(n)$ (by
    PASTA). When the system contains $n>0$ jobs, the number in queue
    is one less, i.e., $n-1$.

    \begin{equation*}
      \begin{split}
\E{L_Q} 
&= \sum_{n=1}^\infty (n-1)\pi(n) 
= (1-\rho)\sum_{n=1}^\infty (n-1) \rho^n\\
&= \rho (1-\rho)\sum_{n=1}^\infty (n-1) \rho^{n-1}
= \rho \sum_{n=1}^\infty (n-1) \pi(n-1)\\
&= \rho \sum_{n=0}^\infty n \pi(n)
= \rho \frac{\rho}{1-\rho}.
      \end{split}
    \end{equation*}

Another way to get the same result is by splitting: 
\begin{equation*}
  \begin{split}
\E{L_Q} 
&= \sum_{n=1}^\infty (n-1)\pi(n) 
=\sum_{n=1}^\infty n\pi(n) -\sum_{n=1}^\infty \pi(n)\\
&= \E L - (1-\pi(0)) = \E L - \rho.
  \end{split}
\end{equation*}
  \end{solution}
\end{exercise}

\begin{comment}
\begin{exercise}[use=false]
  As a challenge you can try to derive \eqref{eq:24} by means of sample path arguments.
  \begin{solution}
    \TBD
  \end{solution}
\end{exercise}
\end{comment}




\begin{exercise}($M/M/1$)
  Use the PASTA property to see that the expected waiting time in the
  system must also be equal to
  $\E W = \sum_{n=0}^\infty \E{W_Q \given N=n} \pi(n) + \E S$, 
where $\E{W_Q \given N=n}$ is the waiting time in queue given that an
arrival sees $N=n$ customers upon arrival. Then, 
 motivate that $\E{W_Q \given N=1} = \E S$. Finally, combine the above to see that 
$\E W = \E S \sum_{n=0}^\infty n \pi(n) + \E S = \E S \E L + \E S$.
  \begin{solution}
 The time in the system $\E W$ is the sum of the time in
      queue plus the service time of the job itself.  Conditioning on
      the number of jobs $N$ in the system at arrival moments, i.e.,
      conditioning on $\pi(n)$, gives the result.  

 $\E{W_Q\given N=0}=0$, of course. If $N=1$, there must be a job
      in service.  Because of the memoryless property of the service
      times, the remaining service time of the customer in service is
      still exponentially distributed with mean $\E S$. Thus,
      $\E{W_Q\given N=1}=\E S$. 

 The expected service time for a customer still in queue is
      $\E S$.  Therefore, for each $n$ we have that, when service
      times are exponential, $\E{W_Q\given N=n}= n \E S$. The probability
      to see $n$ jobs in the system by an arrival is $\pi(n)$ which,
      by PASTA, is equal to $p(n)$. Since $\E L = \sum_n n p(n)$, the
      result follows.
  \end{solution}
\end{exercise}

\begin{comment}
  
\begin{exercise}[use=false] Note: the question is unclear, and needs more explanation. 

  There is distinction between $\E{L_Q}$, i.e., the time-average queue
  length, and the expected number of jobs in the system given that the
  server is busy, i.e., $\E{L\given B}$.  is $\E{L_Q} \neq \E{L\given B}$?
\begin{solution}
  $\E{L_Q}$ is the time average of the queue length process (in steady
  state). Hence, $\E{L_Q}$ contains also the time the queue is
  empty. $\E{L\given B}$ the other hand, is the time average of the
  queue length process, provided the server is busy. Even though the
  queue may be empty while the server is busy, the fraction of time
  the queue length is zero given that the server is busy, is smaller
  than the fraction of time the queue length is zero.  Hence,
  $\E{L\given B} > \E{L_Q}$.
\end{solution}
\end{exercise}
\end{comment}

\begin{exercise}($M/M/1$)
Show that the variance of the number of jobs in the system $L$ is
$\V L = \rho/(1-\rho)^2.$
  \begin{hint}
    $\V{L} = \E{L^2} - (\E L)^2$. We already know $\E L$ so it remains
    to compute $\E{L^2}$. Then, $\E{L^2 }= \sum_{n=0}^\infty n^2 \pi(n)$. Use that $\pi(n)=\rho^n(1-\rho)$ is the probability to see $n$ jobs in the system. Now simplify.    

You might use the moment-generating trick of Exercise~\ref{ex:12}.
  \end{hint}
  \begin{solution}
The starting point is  $\V{L} = \E{L^2} - (\E L)^2$. 

With Wolfram Alpha we get
    \begin{equation*}
      \begin{split}
      \E{L^2 }
&= \sum_{n=0}^\infty n^2 \pi(n) \\
&= \rho \frac{1+\rho}{(1-\rho)^2}.
      \end{split}
    \end{equation*}
    Thus,
\begin{equation*}
\V L = \frac{\rho(1+\rho)}{(1-\rho)^2}-\frac{\rho^2}{(1-\rho)^2} = \frac{\rho}{(1-\rho)^2}.
\end{equation*}

Here are three methods to get the result for $\E{L^2}$.  

One way is to use the standard formula for a geometric series with $\rho < 1$:
\begin{equation*}
\dfrac{1}{1-\rho} = \sum_{n=0}^{\infty}\rho^n.
\end{equation*}
If we differentiate the left and right hand side with respect to
$\rho$  and then  multiply with $\rho$ we obtain
\begin{equation*}
\dfrac{\rho}{(1-\rho)^2}=\sum_{n=0}^{\infty}n\rho^n.
\end{equation*}
Again, differentiating and multiplying with $\rho$ yields, 
\begin{equation*}
  \begin{split}
\rho \frac{(1-\rho)^2 + \rho2(1-\rho)}{(1-\rho)^4} 
&= \rho \frac{1-2\rho+\rho^2 + 2\rho-2\rho^2}{(1-\rho)^4} \\
&= \rho \frac{(1-\rho)^2}{(1-\rho)^4} \\
&=\rho \dfrac{1+\rho}{(1-\rho)^3}\\
&=\sum_{n=0}^{\infty}n^2\rho^n,
  \end{split}
\end{equation*}
and hence
\begin{equation*}
(1-\rho)\sum_{n=0}^{\infty}n^2\rho^n = \rho\dfrac{1+\rho}{(1-\rho)^2}.
\end{equation*}
% Write
% $\partial_\rho$ as a shorthand for the derivative with respect to
% $\rho$. Then observe that
% $(\rho \partial_\rho) \rho^n = \rho n \rho^{n-1} = n\rho^n$, hence
% $(\rho \partial_\rho)^2 \rho^n =
% (\rho \partial_\rho)(\rho\partial_\rho) \rho^n = (\rho \partial
% \rho) n \rho^n = n^2\rho^n$. With this,
% \begin{equation*}
%   \begin{split}
% (1-\rho)  \sum_{n=0}^\infty n^2 \rho^n 
% &=  (1-\rho)(\rho \partial_\rho)^2   \sum_{n=0}^\infty \rho^n\\
% &=  (1-\rho)(\rho \partial_\rho)^2   \frac1{1-\rho}\\
% &=  (1-\rho)\rho \partial_\rho   \frac\rho{(1-\rho)^2}\\
% &=  (1-\rho) \rho  \frac{1+\rho}{(1-\rho)^3} \\
% &=  \rho  \frac{1+\rho}{(1-\rho)^2}.
%   \end{split}
% \end{equation*}
Observe that here we just compute the second moment of a geometric
random variable.


Another way to derive the result is by noting that
$\sum_{i=1}^n i= n(n+1)/2$ from which we get that
$n^2 = -n + 2\sum_{i=1}^n i$. Substituting this relation into
$\sum_n n^2 \rho^n$ leads to
\begin{equation*}
  \begin{split}
    \sum_{n=0}^\infty n^2 \rho^n 
&=    \sum_{n=0}^\infty \left(\sum_{i=1}^\infty 2i \1{i\leq n}  - n\right)\rho^n 
=    \sum_{n=0}^\infty \sum_{i=0}^\infty 2i\1{i\leq n}\rho^n  - \sum_{n=0}^\infty n\rho^n \\
&=    \sum_{i=0}^\infty 2i \sum_{n=i}^\infty \rho^n  - \frac{\E L}{1-\rho} 
=    \sum_{i=0}^\infty 2i \rho^i \sum_{n=0}^\infty \rho^n  - \frac{\E L}{1-\rho} \\
&=    \frac2{1-\rho} \sum_{i=0}^\infty i \rho^i   - \frac{\E L}{1-\rho} 
=    \frac2{(1-\rho)^2} \E L - \frac{\E L}{1-\rho} \\
&=    \frac{\E L}{1-\rho}  \left(\frac2{1-\rho}  - 1\right) 
=    \frac{\E L}{1-\rho}  \frac{1+\rho}{1-\rho} \\
&=    \frac{\rho}{1-\rho}  \frac{1+\rho}{(1-\rho)^2}.
\end{split}
\end{equation*}

A third method is based on using the moment-generating functions: 
\begin{equation*}
  M(s) = \E{e^{sL}} = \sum_{n=0}^\infty e^{sn} p(n) = (1-\rho) \sum_{n=0}^\infty (e^s \rho)^n = \frac{1-\rho}{1-\rho e^s};
\end{equation*}
we assume that $s$ such that the summation converges.
Then 
\begin{equation*}
  \E L = \left.\frac d {ds} M(s) \right|_{s=0} = \left.\frac{1-\rho}{(1-\rho e^s)^2} \rho e^s \right|_{s=0} = \frac{\rho}{1-\rho}, 
\end{equation*}
and 
\begin{equation*}
  \E{L^2}= \left.M(s)''\right|_{s=0} = \frac{2\rho^2}{(1-\rho)^2} + \frac{\rho}{1-\rho}.
\end{equation*}
With a a bit of algebra  the previous results follows.
  \end{solution}
\end{exercise}


\begin{exercise}$(M/M/1$) 
To see how large the variance of $L$ is, relative to the mean number of jobs
in the system, we typically consider the square coefficient of
variation (SCV). Show that the square coefficient of variation of $L$ is $1/\rho$. What do you conclude from this?
\begin{hint}
  SCV  $=\V X/(\E X)^2$. 
\end{hint}
  \begin{solution}
To see how large this variance is, relative to the mean number of jobs
in the system, we typically consider the square coefficient of
variation (SCV). As $\E L = \rho/(1-\rho)$,
\begin{equation*}
  \frac{\V L}{(\E{L})^2} = \frac 1 \rho.
\end{equation*}
Thus, the SCV becomes smaller as $\rho$ increases, but does not become
lower than $1$. So, realizing that the SCV of the exponential
distribution is 1, the distribution of the number of jobs in the
system has larger relative variability than the exponential
distribution.
  \end{solution}
\end{exercise}

When a problem is mainly of a computational type, I coded the solutions and show you all the steps in between so that you can check each step in your computations. As the code is typically nearly identical to the mathematical formulas, you should not have any difficulty
understanding the code. (In the computations below I typically use the simplest, but often
not the most efficient, code.)

\begin{exercise}{\faPhoto}(Hall 5.2) \label{exer: Hall} After observing a single-server queue for
  several days, the following steady-state probabilities have been
  determined: $p(0)=0.4$, $p(1) = 0.3$, $p(2)=0.2$, $p(3)=0.05$ and
  $p(4)=0.05$. The arrival rate is 10 customers per hour. 
  \begin{enumerate}
  \item Determine $\E L$ and  $\E{L_Q}$. 
  \item Using Little's formula, determine $\E W$ and $\E{W_Q}$. 
\item Determine $\V{L}$ and $\V{L_Q}$.
  \item Determine the service time and the utilization.
  \end{enumerate}
    \begin{solution} First find $\E L$


\begin{pyconsole}
P = [0.4, 0.3, 0.2, 0.05, 0.05]
EL = sum(n*P[n] for n in range(len(P)))
EL
\end{pyconsole}

There can only be a queue when a job is in service. Since there is
$m=1$ server, we subtract $m$ from the amount of jobs in the system.
Before we do this, we need to ensure that $n-m$ does not become
negative. Thus, $\E{L_Q} = \sum_n \max\{n-m, 0\} p(n)$.

\begin{pyconsole}
m = 1
ELq = sum(max(n-m,0)*P[n] for n in range(len(P)))
ELq
\end{pyconsole}


\begin{pyconsole}
labda = 10./60
Wq = ELq/labda # in minutes
Wq
Wq/60 # in hours

W = EL/labda # in minutes
W
W/60 # in hours
\end{pyconsole}

Let's use the standard definition of the variance, i.e., $\V X = \sum_{i} (x_i-\E X)^2 \P{X=x_i}$, for once.

\begin{pyconsole}
from math import sqrt
var_L = sum((n-EL)**2*P[n] for n in range(len(P)))
var_L
sqrt(var_L)
\end{pyconsole}


\begin{pyconsole}
var_Lq = sum((max(n-m,0)-ELq)**2*P[n] for n in range(len(P)))
var_Lq
sqrt(var_Lq)
\end{pyconsole}


\begin{pyconsole}
mu = 1./(W-Wq)
1./mu # in minutes

rho = labda/mu
rho
\end{pyconsole}

\begin{pyconsole}
rho = EL-ELq
rho
\end{pyconsole}
This checks with  the previous line.

The utilization must also by equal to the fraction of time the server is busy. 
\begin{pyconsole}
u = 1 - P[0]
u
\end{pyconsole}

Yet another way: Suppose we have $m$ servers. If the system is empty,
all $m$ servers are idle. If the system contains one customer, $m-1$
servers are idle. Therefore, in general, the average fraction of time
the server is idle is
\begin{equation*}
1- u = \sum_{n=0}^\infty \max\{n-m, 0\}  p_n,
\end{equation*}
as in the case there are more than $m$ customers in the system, the
number of idle servers is $0$.


\begin{pyconsole}
idle = sum( max(m-n,0)*P[n] for n in range(len(P)))
idle
\end{pyconsole}

   \end{solution}
 
\end{exercise}

\begin{exercise}
  (Hall 5.5) An $M/M/1$ queue has an arrival rate of 100 per hour and a service rate of 140 per hour.
  What is $p(n)$?
  What are $\E{L_Q}$ and $\E L$?
\begin{solution}
First, $p(n) = (1-\rho)\rho^n$. Now,

\begin{pyconsole}
labda = 100. # per hour
mu = 140. # per hour
ES = 1./mu
rho = labda/mu
rho 
1-rho

L = rho/(1.-rho)
L
Lq = rho**2/(1.-rho)
Lq

W = 1./(1.-rho) * ES
W
Wq = rho/(1.-rho) * ES
Wq
\end{pyconsole}

  \end{solution}
\end{exercise}

\begin{exercise}(Hall, 5.6)
  An $M/M/1$ queue has been found to have an average waiting time in queue of 1 minute. The arrival rate is known to be 5 customers per minute.
 What are the service rate and utilization? Calculate $\E{L_Q}$,  $\E L$ and $\E W$. Finally, 
 the queue operator would like to provide chairs for waiting customers. He would like to have a sufficient number so that all customers can sit down at least 90 percent of the time. How many chairs should he provide?

 \begin{hint}
$\E{L_Q}$ follows right away from an application of Little's law. For the other quantities we need to find $\E S$.  One can use that 
\begin{equation*}
\E{W_Q}= \E L \E S = (\E{L_Q} + \E{L_S}) \E S = (\E{L_Q} + \lambda \E{S})\E{S}.
\end{equation*}


Now $\lambda$ and $\E{W_Q}=1$ are given, and $\E{L_Q}$ has just been computed. Hence, $\E{S}$ (which is the unknown here) can computed with the quadratic formula.

Another way is to realize that, for the $M/M/1$-queue,  $\E{W_Q} = \frac{1}{\lambda}\frac{\rho^2}{1-\rho}$. Then solve for $\rho$, and since $\lambda$ is known, $\E S$ follows. 
 \end{hint}
    \begin{solution}


 $\E{W_Q} = \frac{1}{\lambda}\frac{\rho^2}{1-\rho}$. Since
        $\E{W_Q}$ and $\lambda$ is given we can use this formula to
        solve for $\rho$ with the quadratic formula (and using that
        $\rho > 0$):

\begin{pyconsole}
labda = 5. # per minute
Wq = 1.
a = 1.
b = labda*Wq
c = -labda*Wq
rho = (-b + sqrt(b*b-4*a*c))/(2*a)
rho 

ES = rho/labda
ES
\end{pyconsole} 


\begin{pyconsole}
Lq = labda*Wq
Lq

W = Wq + ES
W

L = labda*W
L
\end{pyconsole} 


The next problem is to find $n$ such that
      $\sum_{j=0}^n p_j > 0.9$.

\begin{pyconsole} 
total = 0. 
j = 0
while total <= 0.9:
   total += (1-rho)*rho**j
   j += 1

total
j
n = j- 1 # the number of chairs 
\end{pyconsole} 

Observe that $j$ is one too high once the condition is satisfied, thus subtract one.

As a check, I use that $(1-\rho) \sum_{j=0}^n \rho^j = 1-\rho^{n+1}$.

\begin{pyconsole}
1-rho**(n) #  this must be too small.
1-rho**(n+1) # this must be OK.
\end{pyconsole} 

And indeed, we found the right $n$.

    \end{solution}
\end{exercise}

\begin{exercise}
  (Hall 5.7). A single-server queueing system is known to have Poisson
  arrivals and exponential service times. However, the arrival rate
  and service time are state dependent. As the queue becomes longer,
  servers work faster, and the arrival rate declines, yielding the
  following functions (all in units of number per hour):
  $\lambda(0) = 5$, $\lambda(1)=3$, $\lambda(2)=2$,
  $\lambda(n)=0, n\geq 3$, $\mu(0) = 0$, $\mu(1)=2$, $\mu(2)=3$, $\mu(n)=4, n\geq 3$. 
Calculate the state probabilities, i.e., $p(n)$ for $n=0,\ldots$. 
\begin{hint}
Use the level-crossing equations of the $M(n)/M(n)/1$ queue.
\end{hint}
    \begin{solution}
      Follows right away from the hint.
    \end{solution}
\end{exercise}

\begin{exercise}
  (Hall 5.14) An airline phone reservation line has one server and a buffer for two customers.
  The arrival rate is 6 customers per hour, and a service rate of just 5 customers per hour.
  Arrivals are Poisson and service times are exponential.
  Estimate $\E{L_Q}$ and the average number of customers served per hour.
  Then, estimate $\E{L_Q}$ for a buffer of size~5.
  What is the impact of the increased buffer size on the number of customers served per hour?
  \begin{hint}
This is a queueing system with loss, in particular the $M/M/1/1+2$ queue.
  \end{hint}
    \begin{solution}
First compute $\E{L_Q}$ for the case with a buffer for $2$ customers.

\begin{pyconsole}
labda = 6.
mu = 5.
rho = labda/mu
c = 1
b = 2
\end{pyconsole} 

Set $p(n) = \rho^n$ initially, and normalize later. Use the
expressions for the $M(n)/M(n)/1$ queue.  Observe that $\rho>1$. Since
the size of the system is $c+b+1$ is finite, all formulas work for
this case too.


There are 4 states in total: $0,1,2,3$. (The reason to import \pyv{numpy} here and convert the lists to arrays is to fix the output precision to 3, otherwise we get long floats in the output.)

\begin{pyconsole}
import numpy as np
np.set_printoptions(precision=3)

P = np.array([rho**n for n in range(c+b+1)])
P

G = sum(P)
G

P /= G # normalize
P
\end{pyconsole} 

\begin{pyconsole}
L = sum(n*P[n] for n in range(len(P)))
L

Lq = sum((n-c)*P[n] for n in range(c,len(P)))
Lq
\end{pyconsole} 


The number of jobs served per hour must be equal to the number of jobs
accepted, i.e., not lost. The fraction of customers lost is equal to
the fraction of customers that sees a full system.

\begin{pyconsole}
lost = labda*P[-1] # the last element of P
lost

accepted = labda*(1.-P[-1]) # rate at which jobs are accepted
accepted
\end{pyconsole}  

Now increase the buffer $b$ to 5.

\begin{pyconsole}
b = 5
P = np.array([rho**n for n in range(c+b+1)])
P
G = sum(P)
G

P /= G # normalize
P

L = sum(n*P[n] for n in range(len(P)))
L

accepted = labda*(1.-P[-1])
accepted
\end{pyconsole}      
    \end{solution}
\end{exercise}

\begin{comment}
\begin{exercise}[use=false]
  The code is in \pyv{progs/mm1\_waiting\_time_distribution.py}. In
this program we compute the waiting time
  distribution. At this point in the text this distribution has not
  yet been derived.  This problem should be moved to another section.

Consider an $M/M/1$ queue with $\mu=1.2$ per hour. Make plots of the
waiting time distribution $\P{W_Q\leq t}$ for various values of
$\lambda$ to show the dependency on the arrival rate. Compare this to
the dependence of average waiting time $\E{W_Q}$ on $\lambda$.

The idea of making such plots is to analyze
  whether the service capacity suffices for a situation in which one
  doesn't know the arrival rate very accurately, but one can control
  the service rate, e.g., by hiring personnel.
  \begin{solution}
    Let's just plot the waiting time distribution
for various values of $\lambda$. 


\begin{figure}[ht]
  \centering
\includegraphics{progs/mm1_waiting_time.tex}
  \caption{The waiting time distribution for various values of the arrival rate.}
  \label{fig:waitingtime}
\end{figure}


\end{solution}

\end{exercise}
\end{comment}


\begin{exercise}(Hall 5.3) After observing a queue with two servers
  for several days, the following steady-state probabilities have been
  determined: $p(0)=0.4$, $p(1) = 0.3$, $p(2)=0.2$, $p(3)=0.05$ and
  $p(4)=0.05$. The arrival rate is 10 customers per hour.
  \begin{enumerate}
  \item Determine $\E L$ and  $\E{L_Q}$. 
  \item Using Little's formula, determine $\E W$ and $\E{W_Q}$. 
  \item Determine $\V L$ and $\V{L_Q}$.
  \item Determine the service time and the utilization.
  \end{enumerate}
  \begin{solution}
 Determine $\E L$ and  $\E{L_Q}$. 

\begin{pyconsole}
P = [0.4, 0.3, 0.2, 0.05, 0.05]

c = 2
Lq = sum((n-c)*P[n] for n in range(c,len(P)))
Lq

L= sum(n*P[n] for n in range(len(P)))
L
\end{pyconsole}

 Using Little's formula, determine $\E W$ and $\E{W_Q}$. 
\begin{pyconsole}
labda = 10./60
Wq = Lq/labda # in minutes
Wq
Wq/60 # in hours

W = L/labda
W
\end{pyconsole} 

 Determine $\V L$ and $\V{L_Q}$.
\begin{pyconsole}
from math import sqrt
var_L = sum((n-L)**2*P[n] for n in range(len(P)))
var_L
sqrt(var_L)

\begin{pyconsole}
var_Lq = sum((max(n-c,0)-Lq)**2*P[n] for n in range(len(P)))
var_Lq
\end{pyconsole}

Determine the service time and the utilization.
\begin{pyconsole}
mu = 1./(W-Wq)
1./mu # in minutes

rho = labda/mu
rho
\end{pyconsole}

\begin{pyconsole}
rho = EL-ELq
rho
\end{pyconsole}
This checks the previous line.

The utilization must also by equal to the fraction of time the server is busy. 
\begin{pyconsole}
u = 1 - P[0]
u
\end{pyconsole}
    \end{solution}
\end{exercise}  

\begin{exercise}
  (Hall 5.8) The queueing system at a fast-food stand behaves in a peculiar fashion.
  When there is no one in the queue, people are reluctant to use the stand, fearing that the food is unsavory.
  People are also reluctant to use the stand when the queue is long.
  This yields the following arrival rates (in numbers per hour): $\lambda(0) = 10$, $\lambda(1)=15$, $\lambda(2)=15$, $\lambda(3)=10$, $\lambda(4)=5$, $\lambda(n)=0, n\geq 5$.
  The stand has two servers, each of which can operate at 5 per hour.
  Service times are exponential, and the arrival process is Poisson.
  Calculate the steady state probabilities.
  Next, what is the average arrival rate?
  Finally, determine $\E L$, $\E{L_Q}$, $\E W$ and $\E{W_Q}$.
  \begin{solution}
First the service rates.
\begin{pyconsole}
import numpy as np
labda = [10., 15., 15., 10., 5.]
c = 2
mn = 2*np.ones(len(labda)+1, dtype=int)  # number of active servers
mn[0] = 0  # no service if system is empty
mn[1] = 1  # one busy server if just one job present
mu = 5*mn # service rate is 5 times no of active servers
mu
\end{pyconsole}

Since there can be arrivals in states $0,\ldots, 4$,  the system can contain $0$ to $5$ customers, i.e., $p(0),\ldots, p(5)$.

Use the level crossing result for the $M(n)/M(n)/1$ queue:

\begin{pyconsole}
P = [1]*(len(labda)+1)
for i in range(1,len(P)):
    P[i] = labda[i-1]/mu[i]*P[i-1]

P = np.array(P) # unnormalized probabilities
P
\end{pyconsole}

\begin{pyconsole}
G = sum(P) # normalization constant
G
P /= G # normalize
P 
\end{pyconsole} 

$\lambda = \sum_{n}\lambda(n) p(n)$.

\begin{pyconsole}
labdaBar = sum(labda[n]*P[n] for n in range(len(labda)))
labdaBar
\end{pyconsole}


The average number in the system is: 

\begin{pyconsole}
Ls = sum(n*P[n] for n in range(len(P)))
Ls
\end{pyconsole}


The average number in queue: 
\begin{pyconsole}
c = 2
Lq = sum((n-c)*P[n] for n in range(c,len(P)))
Lq
\end{pyconsole} 

And now the waiting times:

\begin{pyconsole}
Ws = Ls/labdaBar
Ws # time in the system

Wq = Lq/labdaBar
Wq # time in queue
\end{pyconsole} 

    \end{solution}
\end{exercise}

\begin{exercise}
  (Hall 5.10) A repair/maintenance facility would like to determine
  how many employees should be working in its tool crib. The service
  time is exponential, with mean 4 minutes, and customers arrive by a
  Poisson process with rate 28 per hour. The customers are actually
  maintenance workers at the facility, and are compensated at the same
  rate as the tool crib employees.
 What is $\E W$ for $c=1, 2, 3$, or $4$ servers? How many employees should work in the tool crib?
  \begin{hint}
Realize that we have to control the number of servers. Hence,
    we are dealing with a multi-server queue, i.e., the $M/M/c$
    queue. Use the formulas of Eq.~\eqref{eq:9}.

The remark that maintenance workers are compensated at the same rate
as the tool crib workers confused me a bit at first.  Some thought
revealed that the consequence of this remark is that is it just as
expensive to let the tool crib workers wait (to help maintenance
workers) as to let the maintenance workers wait for tools. (Recall, in
queueing systems always somebody has to wait, either the customer in queue or
the server being idle. If it is very expensive to let customers wait, the number
of servers must be high, whereas if servers are relatively expensive, customers have to do the waiting.)
  \end{hint}
  \begin{solution}

      Would one server/person do? 
\begin{pyconsole}
labda = 28./60 # arrivals per minute
ES = 4.
labda*ES
\end{pyconsole} 

If $c=1$, the load $\rho=\lambda \E S/c >1$ is clearly undesirable for one server.  We need at
least two servers.

It is not relevant to focus on the time in the system, as time
  in service needs to be spent anyway. Hence, we focus on the waiting
  time in queue.


I just convert the formulas of~\eqref{eq:9} to Python code. This saves
me time during the computations.

\begin{pyconsole}
def WQ(c, labda, ES):
    rho = labda*ES/c
    G = sum([(c*rho)**n/factorial(n) for n in range(c)])
    G += (c*rho)**c/(1.-rho)/factorial(c)
    Lq = (c*rho)**c/(factorial(c)*G) * rho/(1.-rho)**2
    return Lq/labda # Wq, Little's law

\end{pyconsole} 

Considering the scenario with one server is superfluous as $\rho>1$ in
that case.

What is the waiting time for $c=2$ servers?

\begin{pyconsole}
WQ(2, 28./60, 4) # in minutes
WQ(2, 28./60, 4)/60. # in hours
\end{pyconsole}

What is the waiting time for $c=3$ servers?

\begin{pyconsole}
WQ(3, 28./60, 4)   # in minutes
WQ(3, 28./60, 4)/60. # in hours
\end{pyconsole}


What is the waiting time for $c=4$ servers?

\begin{pyconsole}
WQ(4, 28./60, 4) # in minutes
WQ(4, 28./60, 4)/60. # in hours
\end{pyconsole} 

In the next  part of the question we will interpret these numbers.

Since both types of workers cost the same amount of money per unit
time, it is best to divide the amount of waiting/idleness equally over
both types of workers.  I am inclined to reason as follows. The
average amount of waiting time done by the maintenance workers per
hour is $\lambda \E{W_Q}$. To see this, note that maintenance workers arrive at rate $\lambda$, and each worker waits on average $\E{W_Q}$ minutes. Thus, worker time is wasted at rate $\lambda \E{W_Q}$. Interestingly, with Little's law, $\E{L_Q}=\lambda \E{W_Q}$, i.e., the rate at which workers waste capacity (i.e.  waiting in queue) is $\E{L_Q}$. On the other hand, the  rate of work capacity wasted by the tool crib employees being idle is $c-\lambda \E{S}$, as $\lambda \E S$ is the average number of servers busy, while $c$ crib servers are available.

As both types of
employees are equally expensive, we need to choose $c$ such that
the number of maintenance workers waiting (i.e., being idle because they are waiting in queue), is  equal to the number of crib workers being idle. In other words, we search for a $c$ such that $\E{L_Q} \approx c- \lambda \E S$ (where, of course, $\E{L_Q}$ depends on $c$).


\begin{pyconsole}
labda = 28./60
ES = 4.
c = 2
ELQ = labda*WQ(c, labda, ES)
ELQ
c-labda*ES
\end{pyconsole} 
Now the maintenance employees wait more than the tool crib employees.

\begin{pyconsole}
c = 3
ELQ = labda*WQ(c, labda, ES)
ELQ
c-labda*ES
\end{pyconsole} 

\begin{pyconsole}
c = 4
ELQ = labda*WQ(c, labda, ES)
ELQ
c-labda*ES
\end{pyconsole} 

Clearly, $c=3$ should do.
    \end{solution}
\end{exercise}



\Closesolutionfile{hint}
\Closesolutionfile{ans}

\opt{solutionfiles}{
\subsection*{Hints}
\input{hint}
\subsection*{Solutions}
\input{ans}
}

%\clearpage


%%% Local Variables:
%%% mode: latex
%%% TeX-master: "../queueing_book"
%%% End:
