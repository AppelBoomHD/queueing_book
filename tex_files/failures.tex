%\section{Service Interruptions}
\section{Preemptive Interruptions, Server Failures}
\label{sec:preempt-interr-serv}

In~\cref{sec:setups-batch-proc,sec:non-preempt-interr} we assumed that servers are never interrupted while serving a job.
However, in many situations this assumption is not satisfied: a person might receive a short phone call while working on a job, a machine may fail in the midst of processing, and so on.
In this section, we develop a model to compute the influence on the mean waiting time of such \emph{preemptive outages}, i.e., interruptions that occur \emph{during} a service.
As in~\cref{sec:non-preempt-interr}, we can use Sakasegawa's formula to estimate expected queueing time for the $G/G/1$ queue.
For this, it suffices to find expressions for $\E S$ and $\V S$.

We will first present the main formulas and show how to apply them.
The exercises at the end of the section ask you to provide the derivations.
Specifically, here we compute the expectation and variance of a random number of failures, each with a random duration.
Similar problems occur in, for instance, insurance theory (or inventory theory) where we are interested in (the distribution of) the total claim (demand) size that occurs in a period.
Noting that the total claim (demand) is given by a random number of claims (demands), each of which is a random variable by itself, that problem is exactly the same as the problem here.

The techniques we develop here are of general interest, and they provide relations between the topics discussed in~\cref{sec:preliminaries}, \cref{sec:poisson-distribution},~\cref{sec:distr-queue-length}, and~\cref{sec:n-policies-mg1}.


\opt{solutionfiles}{
\subsection*{Theory and Exercises}
\Opensolutionfile{hint}
\Opensolutionfile{ans}
}

As in the previous sections, we derive expressions for the expectation and variance of the effective processing time $S$. 

Supposing that $N$ interruptions occur during the net service time $S_0$ of a job, and the repair times $\{R_i\}$ form an i.i.d.
sequence distributed as a common random variable $R$, write
\begin{equation}\label{eq:104}
S_N = \sum_{i=1}^N R_i
\end{equation}
for the total duration of the interruptions that occur during the service of a job. Then,  the  effective service time becomes
\begin{equation*}
  S= S_0 + S_N = S_0 + \sum_{i=1}^N R_i. 
\end{equation*}
It is important to realize that $N$ is a random number. 


A common assumption is that the time between two interruptions is memoryless; let the interruptions occur at a given \emph{failure rate} $\lambda_f$.
Then, if the net service time $S_0$ is a constant, the expected number of failures $\E N$ is $\lambda_f S_0$.
More generally, we show below that $\E N =\lambda_f \E{S_0}$.

%Consequently, the number of interruptions $N$ that occur during the net service time $S_0$ is Poisson distributed with mean $\E N = \lambda_f \E{S_0}$.

Define the \emph{availability} as
\begin{equation*}
 A=\frac{m_f}{m_f + m_r},
\end{equation*}
where $m_f$ is the mean time to fail and $m_r$ the mean time to repair.  From~\cref{ex:80} it follows that
\begin{equation}\label{eq:91}
A=\frac 1{1+\lambda_f \E R}.
\end{equation}
Then, from~\cref{ex:f-16}, 
\begin{equation}\label{eq:83}
 \E S = \frac{\E{S_0}}{A}. 
\end{equation}
Now  the load is easy to compute: 
\begin{equation*}
\rho = \lambda \E S = \lambda \frac{\E{S_0}}A.
\end{equation*}
Clearly, the server load increases due to failures. 


By assuming that repair times are exponentially distributed with mean $\E{R}$, we can find with the exercises below  that 
\begin{equation}\label{eq:87}
 C_s^2 = C_0^2 + 2 A(1-A) \frac{\E{R}}{\E{S_0}},
\end{equation}
where $C_0^2$ is the SCV of $S_0$, i.e., the service time without interruptions. 


Before deriving the above expressions, let us see how to apply them. 

\begin{exercise}\clabel{ex:l-157}
 Suppose we have a machine with memoryless failure behavior, with a mean-time-to-fail of $3$ hours. Regular service times are deterministic with an average of 10 minutes, jobs arrive as a Poisson process with rate of 4 per hour. Repair times are exponential with a mean duration of 30 minutes. What is the average sojourn time?
\begin{hint}
 Mind to work in a consistent set of units, e.g., hours. It is easy to make mistakes. 
\end{hint}
\begin{solution}
Let's first compute the load. If $\rho>1$ we are in trouble.
\begin{pyconsole}
labda = 4.
ES0 = 10./60 # in hours
labda_f = 1./3
ER = 30./60 # in hours
A = 1./(1+labda_f*ER)
A
ES = ES0/A
ES
rho = labda*ES
rho
\end{pyconsole}
As $\rho<1$, the system is not in overload. Now for the queueing time.
\begin{pyconsole}
Ca2 = 1.
C02 = 0. # deterministic service times
Ce2 = C02 + 2*A*(1-A)*ER/ES0
Ce2
EW = (Ca2+Ce2)/2 * rho/(1-rho) * ES
EW
EW + ES # sojourn time
\end{pyconsole}
\end{solution}
\end{exercise}

\begin{extra}
 Suppose we could buy another machine that never fails. What is the average sojourn time?
\begin{solution}
Now we don't need to take availability into account: the machine never fails so $A=1$. 
 \begin{pyconsole}
labda = 4.
ES0 = 10./60 # in hours
A = 1
ES = ES0/A
rho = labda*ES
rho
Ca2 = 1.
C02 = 0. # deterministic service times
EW = (Ca2+C02)/2 * rho/(1.-rho) * ES
EW
EW + ES # sojourn time
\end{pyconsole}
The average time in queue reduces from $\approx 0.6$ to $\approx 0.17$ hours, a reduction by about a factor 3. 
\end{solution}
\end{extra}


The rest of the section is concerned with the derivations of the formulas. 


\begin{extra}\clabel{ex:79}
 Provide an intuitive explanation for $\E S = \E{S_0}/A$.
\begin{solution}
An intuitive way to obtain this result is by noting that $A$ is the fraction of time the server is working. As the total service time of a job is $\E S$, the net work done is $ A\E S$. But this must be the time needed to do the real job, hence $A \E S = \E{S_0}$. 
\end{solution}
\end{extra}



\begin{exercise}\clabel{ex:80}
 Derive~\cref{eq:91} for our model of interruptions.
\begin{hint}
 Observe that $m_f = 1/\lambda_f$ and $m_r = \E R$.
\end{hint}
\begin{solution}
  The time to fail is the time in between two interruptions.
  By assumption, the failure times are $\Exp(\lambda_f)$, hence $m_f = 1/\lambda_f$. 
  The expected duration of an interruption is $\E R = m_r$.
  With this
\begin{equation*}
 A=\frac{m_f}{m_f + m_r}=\frac{1/\lambda_f }{1/\lambda_f + \E R}. 
\end{equation*}
\end{solution}
\end{exercise}


\begin{extra}\clabel{ex:84}
 Suppose that the number of failures $N=n$, show that $\E{S_n}=n\E R$, where $S_n$ is given by~\cref{eq:104}. 
\begin{hint}
 Is it relevant for the expectation of $S_n$ that $R_1,\ldots, R_n$ are mutually independent?
\end{hint}
\begin{solution}
  The expectation of the \emph{fixed} sum of random variables is the same as the sum of the expectations; independence is irrelevant.
  Hence,
\begin{equation*}
 \E{S_n } = \E{\sum_{i=1}^n R_i}= n \E R, 
\end{equation*}
since by assumption $\{R_i\}$ have the same distribution, hence $\E{R_i} = \E R$.
\end{solution}
\end{extra}


\begin{extra}\clabel{ex:16}
 Use~\cref{ex:84} to show that $\E{S_N}=\E R \E N$.
 This result is known as \recall{Wald's equation}.
\begin{hint}
 Use~\cref{eq:105}. 
\end{hint}
\begin{solution}
Using~\cref{eq:105} and the fact that $\E{S_n} = n \E R$, we get
\begin{align*}
 \E{S_N } 
&= \E{ \sum_{n=0}^\infty \1{N=n} S_n } 
= \E{ \sum_{n=0}^\infty \1{N=n} \left(\sum_{i=1}^n R_i \right)} \\
&= \sum_{n=0}^\infty \E{\1{N=n} n\E{R}} \\
&= \E{R} \sum_{n=0}^\infty n \E{\1{N=n}} = \E R \sum_{n=0}^\infty n p_n\\
&= \E R \E N,
\end{align*}
where we specifically use~\cref{eq:p-78} in the second to last equation.
\end{solution}
\end{extra}


\begin{exercise}\clabel{ex:f-3}
Show that $\E N = \lambda_f \E{S_0}$ and $\E{N^2} = \lambda_f^2 \E{S_0^2} + \lambda_f \E{S_0}$.
\begin{hint}
  Use that the joint distribution of $S_0=s, N=k$ is given by $g(s)\, e^{-\lambda s}(\lambda s)^k/k!$, where we assume for the sake of simplicity that $S_0$ has the density $g$.
  Then use~\cref{ex:p-1}.
\end{hint}
\begin{solution}
  If $S=s$, then , by~\eqref{eq:6}, the expected number of failures that arrive is $\E{N(s)} = \lambda s$.
  Therefore, $\E{N} = \E{\lambda_f S_0} = \lambda_f \E{S_0}$.

  In more detail, and with the hint, 
  \begin{equation*}
    \E{N} = \int_0^\infty \sum_{k=0}^\infty k e^{-\lambda_f s}\frac{(\lambda_f s)^k}{k!} g(s) \d s = \int_0^\infty \lambda_f s g(s) \d s = \lambda_f \E{S_0}.
  \end{equation*}
Next, ~\cref{ex:p-1},
  \begin{equation*}
    \E{N^2} = \int_0^\infty \sum_{k=0}^\infty k^2 e^{-\lambda_f s}\frac{(\lambda_f s)^k}{k!} g(s) \d s = \int_0^\infty (\lambda_f^2 s^2 + \lambda_f s) g(s) \d s.
  \end{equation*}
\end{solution}
\end{exercise}


\begin{exercise}\clabel{ex:f-16}
 Show that $\E{S} = \E{S_0} + \lambda_r \E{S_0} \E R$, and  conclude that $\E S = \E{S_0}/A$.
\begin{hint}
 Realize that $\E N = \lambda_f \E{S_0}$. Then use~\cref{ex:80}.
\end{hint}
\begin{solution}
 \begin{align*}
   \E S &= \E{S_0 + S_N} = \E{S_0} + \E{S_N} \\
    &= \E{S_0} + \E N \E R = \E{S_0} + \lambda_f \E{S_0} \E R= \E{S_0}(1+\lambda_f \E R).
 \end{align*}
\end{solution}
\end{exercise}


\begin{exercise}\clabel{ex:l-159}
 The derivation of $C_s^2$ is a bit more involved.
To understand why,  explain first that $\V{S} \neq \V{S_0} + \V{\sum_{i=0}^N R_i}$.
\begin{solution}
 Observe that $S_0$ and $N$ are not independent. In fact, when $S_0=s$, the number of failures $N$ is Poisson distributed with mean $\lambda_f s$. 
\end{solution}
\end{exercise}



\begin{exercise}\clabel{ex:81}
Show that 
\begin{equation*}
 \E{S^2} = \E{S_0^2} + 2\E{S_0 \sum_{i=1}^N R_i} + \E{\sum_{i=1}^N R_i^2} + \E{\sum_{i=1}^N \sum_{j\neq i} R_i R_j}.
\end{equation*}
\begin{solution}
 Just work out the square of $S_0+\sum_{i=1}^N R_i$ and take expectations. Realize that $(\sum_i R_i)^2 = \sum_i R_i^2 + \sum_i\sum_{j\neq i} R_i R_j$. 
\end{solution}
\end{exercise}


\begin{extra}
Show that  $\E{S_0 \sum_{i=1}^N R_i} = \lambda_f \E R \E{S_0^2}. $
 % To simplify the result of~\cref{ex:81}, we assume at first that $S_0$ is known, so that the number of failures that occur during a service time $S_0$ is Poisson distributed, i.e., $N\sim P(\lambda_f S_0)$.
 % Show that $\E{S_0 \sum_{i=1}^N R_i\given S_0} = \lambda_f S_0^2 \E{R}$.
\begin{hint}
  Use~\cref{eq:105} and~\cref{ex:f-3}.
\end{hint}
\begin{solution}
With the hint,
  \begin{align*}
    \E{S_0 \sum_{i=1}^N R_i}
    &=   \E{S_0 \sum_{n=0}^\infty \1{N=n} \sum_{i=1}^N R_i} \\
    &=   \E{S_0 \sum_{n=0}^\infty \1{N=n} \sum_{i=1}^n R_i} \\
    &=   \E R \E{S_0 \sum_{n=0}^\infty \1{N=n} n } \\
    &=   \E R \sum_{n=0}^\infty n \E{S_0 \1{N=n}}.
  \end{align*}
  Now observe that $S_0$ and  $\1{N=n}$ are not independent. Thus, we need to use the joint distribution of~\cref{ex:f-3}, and then,
  \begin{align*}
    \E{S_0 \1{N=n}}
    &=  \int_0^\infty s  g(s) e^{-\lambda_f s} \frac{(\lambda_f s)^n}{n!} \d s.
  \end{align*}
  Continuing with the previous relation:
  \begin{align*}
    \E R \sum_{n=0}^\infty n \E{S_0 \1{N=n}}
    & =  \E R \sum_{n=0}^\infty n \int_0^\infty s  g(s) e^{-\lambda_f s} \frac{(\lambda_f s)^n}{n!} \d s \\
    & =  \E R \int_0^\infty s  g(s) \sum_{n=0}^\infty n e^{-\lambda_f s} \frac{(\lambda_f s)^n}{n!} \d s \\
    & =  \E R \int_0^\infty s  g(s) \lambda_f s \d s \\
    & =  \lambda_f \E R \E{S_0^2}.
  \end{align*}
\end{solution}
\end{extra}

\begin{extra}\clabel{ex:f-82}
 Show that $\E{\sum_{i=1}^N R_i^2} = \lambda_f \E{S_0 }\E{R^2}$.
\begin{hint}
 Use Wald's equation, which we derived in~\cref{ex:16}.
\end{hint}
\begin{solution}
  With the hint,
 \begin{align*}
 \E{\sum_{i=1}^N R_i^2} = \E{R^2}\E{N} = \lambda_f \E{S_0} \E{R^2}.
 \end{align*}
\end{solution}
\end{extra}

\begin{extra}\clabel{ex:f-88}
Show that 
$\E{\sum_{i=1}^N \sum_{j\neq i} R_i R_j} = \lambda_f^2 \E{S_0^2} (\E{R})^2.$
\begin{solution}
Since the $\{R_i\}$ are i.i.d., 
 \begin{align*}
\E{\sum_{i=1}^N \sum_{j\neq i} R_i R_j}
&=\E{\sum_{n=0}^\infty \1{N=n} \sum_{i=1}^N \sum_{j\neq i} R_i R_j} \\
&=\E{\sum_{n=0}^\infty \1{N=n} \sum_{i=1}^n \sum_{j\neq i} R_i R_j} \\
&=\E{\sum_{n=0}^\infty \1{N=n} n(n-1)} (\E{R})^2\\ 
&=(\E{N^2} - \E{N}) (\E{R})^2\\ 
&=(\lambda_f^2 \E{S_0^2} + \lambda_f \E{S_0} - \lambda \E{S_0} ) (\E{R})^2.
 \end{align*}
where we use \cref{ex:f-3} in the last line. 
\end{solution}
\end{extra}

Principally we are done now. With  $\E{S^2}$ and $\E{S}$ we can compute the SCV via $(\E{S^2}-(\E{S})^2)/(\E{S})^2$, so that we have all components for Sakasegawa's formula.  The rest is polishing. 

\begin{extra}\clabel{ex:802}
 Combine the above to see that
 \begin{equation*}
 \E{S^2} = \frac{\E{S_0^2}}{A^2} + \lambda_f \E{R^2} \E{S_0}.
 \end{equation*}
\begin{solution}
It is just algebra based on the results above.  Observe that $(1/A) = 1+\lambda_f \E R$,
\begin{align*}
  \E{S^2}
  &= \E{S_0^2} + 2\E{S_0 \sum_{i=1}^N R_i} + \E{\sum_{i=1}^N R_i^2} + \E{\sum_{i=1}^N \sum_{j\neq i} R_i R_j} \\
  &= \E{S_0^2} + 2 \lambda_f \E R \E{S_0^2} + \lambda_f \E{S_0 }\E{R^2} +  \lambda_f^2 \E{S_0^2} (\E{R})^2 \\
  &= \E{S_0^2}/A  +  \lambda_f \E R \E{S_0^2} + \lambda_f \E{S_0 }\E{R^2} +  \lambda_f^2 \E{S_0^2} (\E{R})^2 \\
  &= \E{S_0^2}/A  +  \lambda_f \E R \E{S_0^2}(1+\lambda_f \E R)  + \lambda_f \E{S_0 }\E{R^2}\\
  &= \E{S_0^2}/A  +  \lambda_f \E R \E{S_0^2}/A + \lambda_f \E{S_0 }\E{R^2}\\
  &= (1  +  \lambda_f \E R) \E{S_0^2}/A  + \lambda_f \E{S_0 }\E{R^2}\\
  &= \E{S_0^2}/A^2  + \lambda_f \E{S_0 }\E{R^2}.
\end{align*}
\end{solution}
\end{extra}

\begin{extra}\clabel{ex:83}
Show that 
 \begin{equation*}
 \V{S} = \frac{\V{S_0}}{A^2} + \lambda_f \E{R^2} \E{S_0}.
 \end{equation*}
\begin{solution} 
 \begin{equation*}
 \V{S} = \E{S^2} - (\E S)^2 = 
\frac{\E{S_0^2}}{A^2} + \lambda_f \E{R^2} \E{S_0} -\frac{(\E{S_0})^2}{A^2}.
 \end{equation*}
\end{solution}
\end{extra}

\begin{exercise}\clabel{ex:l-160}
Show that
 \begin{equation*}
 C_s^2 = \frac{\V{S}}{(\E S)^2} = C_0^2 + \frac{\lambda_f \E{R^2} A^2}{\E{S_0}},
 \end{equation*}
\begin{hint} Just realize that $\E{S} = \E{S_0}/A$, and use the above.
\end{hint}
\begin{solution} Using~\cref{ex:81}--\cref{ex:83}
 \begin{align*}
C_s^2 &= \frac{\V{S}}{(\E S)^2} =\frac{V(S) A^2}{(\E{S_0})^2} \\
&=\frac{\E{S_0^2} + \lambda_f \E{R^2} \E{S_0}A^2 -(\E{S_0})^2}{(\E{S_0})^2} \\
&=\frac{\E{S_0^2} -(\E{S_0})^2}{(\E{S_0})^2} + \frac{\lambda_f \E{R^2} \E{S_0}A^2}{(\E{S_0})^2} \\
&=C_0^2 + \frac{\lambda_f \E{R^2}A^2}{\E{S_0}}.
 \end{align*}
\end{solution}
\end{exercise}


\begin{exercise}\clabel{ex:l-161}
With the above assumption that $R$ is exponentially distributed, show that
 \begin{equation*}
 C_s^2 = C_0^2 + 2 A(1-A) \frac{\E{R}}{\E{S_0}}.
 \end{equation*}
\begin{solution} 
When repair times are exponentially distributed with mean $\E{R}$ we have, by~\cref{ex:15}, that  $\E{R^2}=2(\E R)^2$. 
Since $A=1/(1+\lambda_f \E R)$, 
 \begin{equation*}
 \begin{split}
 \lambda_f \E{R^2} A^2 
&= 2\lambda_f (\E R)^2 A^2 \\ 
&= 2 \frac{\lambda_f \E R }{1+\lambda_f \E R} A \E R \\
&= 2 \left(1-\frac{1}{1+\lambda_f \E R}\right) A \E R = 2(1-A)A \E R.
 \end{split}
 \end{equation*}
\end{solution}
\end{exercise}



\opt{solutionfiles}{
\Closesolutionfile{hint}
\Closesolutionfile{ans}
\subsection*{Hints}
\input{hint}
\subsection*{Solutions}
\input{ans}
}


%\clearpage


%%% Local Variables:
%%% mode: latex
%%% TeX-master: "../companion"
%%% End:
