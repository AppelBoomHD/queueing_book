%\section{Service Interruptions}
\section{Preemptive Interruptions, Server Failures}
\label{sec:preempt-interr-serv}

\subsection*{Theory and Exercises}

\Opensolutionfile{hint}
\Opensolutionfile{ans}

%See section 1.11 of Zijm's book for the theory.

Up to now we assumed that servers are never interrupted while serving a job. In many situations this is not true, a person  might be receive short phone calls while working on a job, a machine may fail in the midst of processing, and so on. In this section we develop a model to account for such \emph{preemptive interruptions}, i.e., interruptions that occur \emph{during} a service, and compute the influence on the mean waiting time. 

Let us assume that a job's normal service, without interruptions, is given by $S_0$. The durations of the interruptions are given by the i.i.d. random variables $\{D_i\}$ and have common mean $\E D$ and variance $\V D$. If $N$ interruptions occur, the effective service time will then be
\begin{equation*}
S= S_0 + \sum_{n=1}^N D_i.
\end{equation*}

To compute the mean waiting time we can use the $G/G/1$ waiting time formula. From this formula it is clear that it suffices to find expressions for $\E S$ and $\V S$, since $C_s^2 = \V S /(\E S)^2$.  Thus, this will be our task for the rest of the section. We remark in passing that the results and the derivation are of general interest.

We first aim to find an expression for $\E S$.  Write  $S_N = \sum_{i=1}^N D_i$ for the total duration of the interruptions, so that the total job duration becomes $S = S_0 + S_N$.

\begin{exercise}
  Suppose that $N=n$, show that $\E{S_n}=n\E D$.

\begin{hint}
    Is it relevant that for the expectation of $S_n$ that $D_1,\ldots, D_n$ are mutually independent?
  \end{hint}
\begin{solution}
The expectation of the sum of random variables is the same as the sum of the expectations; independence is irrelevant. Hence,
\begin{equation*}
  \E{S_n } =  \E{\sum_{i=1}^n D_i}= \E{D_1} + \E{D_2} + \cdots \E{D_n} = n \E D,
\end{equation*}
where the last equation follows from the fact that the $D_i$ have the same distribution. 
\end{solution}
\end{exercise}

Let $p_n=\P{N=n}$; then it is reasonable that $\E{S_N}=\sum_{n=0}^\infty \E{S_n}p_n$. (Compare the definition of $\E{f(X)}=\sum_{n} f(n) p_n$.)

\begin{exercise}\label{ex:16}
Use this expression to show that $\E{S_N}=\E D \E N$.
\begin{hint}
  Use that $\{D_i\}$ is a sequence of i.i.d. r.v.s. Hence, $\E{\sum_{i=1}^n D_i} = n \E D$. 
\end{hint}
\begin{solution}
Since $\E{S_n} = n \E D$, 
\begin{align*}
  \E{\sum_{i=1}^N D_i} 
&=  \E{ \sum_{n=0}^\infty \1{N=n} \left(\sum_{i=1}^n D_i \right)} \\
&=  \sum_{n=0}^\infty \E{\1{N=n} n\E{D}} \\
&=  \E{D} \sum_{n=0}^\infty n \E{\1{N=n}} = \E D \sum_{n=0}^\infty p_n n \\
&= \E D \E N.
\end{align*}
This result in known as Wald's equation. 
\end{solution}
\end{exercise}

With the above, 
\begin{equation*}
  \E{S} = \E{S_0 + S_N} = \E{S_0} + \E D \E N.
\end{equation*}
To make further progress, we need some additional assumptions. A common assumption is that the time between two interruptions is memoryless, hence the time between failures is exponentially distributed, with rate $\lambda_f$ say.  Consequently, the number of interruptions $N$ that occur during the net service time $S_0$ must be Poisson distributed with mean $\E N = \lambda_f \E{S_0}$. 


Define the \emph{availability} as
\begin{equation*}
  A=\frac{m_f}{m_f + m_r},
\end{equation*}
where $m_f$ is the mean time to fail  and $m_r$ the mean to time to repair. 
\begin{exercise}
  Show that 
  \begin{equation*}
A=\frac 1{1+\lambda_f \E D}
  \end{equation*}
for our model of interruptions.
  \begin{hint}
    Observe that $m_f = 1/\lambda_f$ and $m_r = \E D$?
  \end{hint}
  \begin{solution}
The time to fail is the time in between to interruptions. We assumed that these times were exponential with mean $1/\lambda_f$. The duration of an interruption is $D$, which can be interpreted as the time to repair the server, hence $m_r = \E D$. With this
\begin{equation*}
  A=\frac{m_f}{m_r + m_f}=\frac{1/\lambda_f }{1/\lambda_f + \E D}. 
\end{equation*}
  \end{solution}
\end{exercise}


\begin{exercise}
  Show that 
  \begin{equation*}
\E{S} = \frac{\E{S_0}} A = \E{S_0} (1+\lambda_f \E D).
  \end{equation*}
  \begin{hint}
    Realize that $\E N = \lambda_f \E{S_0}$.
  \end{hint}
  \begin{solution}
    \begin{equation*}
      \E S = \E{S_0} + \E N \E D = \E{S_0} +  \lambda_f \E{S_0} \E D= \E{S_0}(1+\lambda_f \E D).
    \end{equation*}
  \end{solution}
\end{exercise}
An intuitive way to obtain this result is by noting that $A$ is the fraction of time the server is working. As the total service time of a job is $\E S$, the net work done is $ A\E S$. But this must be the time needed to do the real job, hence $A \E S = \E{S_0}$.  

It is important to realize that 
\begin{equation*}
\rho = \lambda \E S = \lambda \frac{\E{S_0}}A,
\end{equation*}
hence the load increases due to failures. 


We can similar ideas to derive an expression for the variance of $S$. The next exercise helps to  understand why this derivation is a bit more involved.
\begin{exercise}
  Why is $\V{S} \neq \V{S_0} + \V{\sum_{i=0}^N D_i}$?
  \begin{solution}
    Observe that $S_0$ and $N$ are not independent. In fact, when $S_0=s$, the number of failures $N$ is Poisson distributed with mean $\lambda_f s$. 
  \end{solution}
\end{exercise}

So let us first consider $\E{S^2}$; recall that $\V S = \E{S^2} - (\E S)^2$, and we already know that $\E S = \E{S_0}/A$. 

\begin{exercise}
Show that 
\begin{equation*}
  \E{S^2} = \E{S_0^2} + 2\E{S_0 \sum_{i=1}^N D_i} + \E{\sum_{i=1}^N D_i^2} + \E{\sum_{i=1}^N \sum_{j\neq i} D_i D_j}.
\end{equation*}
\begin{solution}
  Just work out the square of $S_0+\sum_{i=1}^N$ and take expectations. Realize that $(\sum_i D_i)^2 = \sum_i D_i^2 + \sum_i\sum_{j\neq i} D_i D_j$.  
\end{solution}
\end{exercise}

To simplify this, we  assume at first that $S_0$ is known, so that the number of failures that occur during a service time $S_0$ is Poisson distributed, i.e.,  $N\sim \text{Poisson}(\lambda_f S_0)$.  Hence $\E{N\given S_0} = \lambda_f S_0$ and $\E{N^2\given S_0}= \lambda_f S_0 + \lambda_f^2 S_0^2$. 

\begin{exercise}
  Show that $\E{S_0 \sum_{i=1}^N D_i\given S_0} = \lambda_f S_0^2 \E{D}$.
\begin{solution}
$\E{S_0 \sum_{i=1}^N D_i\given S_0} = 
S_0 \E{\sum_{i=1}^N D_i\given S_0} = S_0 \E D \E N = \lambda_f \E D S_0^2$.
\end{solution}
\end{exercise}

\begin{exercise}
Show that $\E{\sum_{i=1}^N D_i^2\given S_0} = \lambda_f S_0 \E{D^2}$.
\begin{hint}
  Use Wald's equation, which we derived in Exercise~\ref{ex:16}.
\end{hint}
\begin{solution}
  \begin{align*}
    \E{\sum_{i=1}^N D_i^2\given S_0} 
&= \E{D^2}\E{ \sum_{i=1}^n n \1{N=n}\given S_0}\\
&= \E{D^2} \E{N\given S_0} \\
&= \lambda_f S_0 \E{D^2}.
  \end{align*}
\end{solution}
\end{exercise}

\begin{exercise}
Show that 
$\E{\sum_{i=1}^N \sum_{j\neq i} D_i D_j\given S_0} = \lambda_f^2 S_0^2 (\E{D})^2.$
\begin{solution}
Since the $\{D_i\}$ are i.i.d., 
  \begin{equation*}
\E{\sum_{i=1}^N \sum_{j\neq i} D_i D_j\given S_0}
= \E{N(N-1)|S_0} (\E{D})^2 
= (\E{N^2|S_0}-\E{N\given S_0}) (\E{D})^2.
  \end{equation*}
Now $\E{N^2\given S_0}=\lambda_f^2 S_0^2 +\lambda_f S_0$ and $\E{N\given S_0} = \lambda_f S_0$.
\end{solution}
\end{exercise}

\begin{exercise}
  Combine the above to see that
    $\E{S^2\given S_0} = \frac{S_0^2}{A^2} + \lambda_f \E{D^2} S_0$. From this, 
  \begin{equation*}
    \E{S^2} = \frac{\E{S_0^2}}{A^2} + \lambda_f \E{D^2} \E{S_0}.
  \end{equation*}
  \begin{solution}
For the first equation,
\begin{equation*}
  \E{S^2\given S_0} = S_0^2 + 2\lambda_f \E D S_0^2 + \lambda_f \E{D^2} S_0 + \lambda_f^2 (\E D)^2 S_0^2.
\end{equation*}
Assemble all terms with $S_0^2$ and observe that $(1/A) = 1+\lambda_f \E D$. For the second, recall that we assumed at first that $S_0$ was fixed, which we indicated by the condition on $S_0$. When $S_0$ is a random variable, we can just take the expectation at the left and right, and obtain the second result. 
  \end{solution}
\end{exercise}

\begin{exercise}
Next, show that  
  \begin{equation*}
    \V{S} = \frac{\V{S_0}}{A^2} + \lambda_f \E{D^2} \E{S_0}.
  \end{equation*}
  \begin{solution}
    \begin{equation*}
    \V{S} = \E{S^2} - (\E S)^2 = 
\frac{\E{S_0^2}}{A^2} + \lambda_f \E{D^2} \E{S_0} -\frac{(\E{S_0})^2}{A^2}.
    \end{equation*}
  \end{solution}
\end{exercise}

\begin{exercise}
  Finally, show that
  \begin{equation*}
    C_s^2 = \frac{\V{S}}{(\E S)^2} = C_0^2 + \frac{\lambda_f \E{D^2} A^2}{\E{S_0}}.
  \end{equation*}
where $C_0^2$ is the SCV of $S_0$, i.e., the service time without interruptions. 
\begin{hint} Just realize that $\E{S} = \E{S_0}/A$, and use the above.
\end{hint}
\begin{solution}
  \begin{align*}
C_s^2 &= \frac{\V{S}}{(\E S)^2} =\frac{V(S) A^2}{(\E{S_0})^2} \\
&=\frac{\E{S_0^2} + \lambda_f \E{D^2} \E{S_0}A^2 -(\E{S_0})^2}{(\E{S_0})^2} \\
&=\frac{\E{S_0^2} -(\E{S_0})^2}{(\E{S_0})^2} + \frac{\lambda_f \E{D^2} \E{S_0}A^2}{(\E{S_0})^2} \\
&=C_0^2 + \frac{\lambda_f \E{D^2}A^2}{\E{S_0}}.
  \end{align*}
\end{solution}
\end{exercise}

If we assume that repair times are exponentially distributed with mean $\E{D}$, we can simplify this yet further.
\begin{exercise}
  Under this condition, show that $\E{D^2}=2(\E D)^2$. 
\end{exercise}
\begin{exercise}
With the above assumption on the distribution of $D$, show that
  \begin{equation*}
    C_s^2 = C_0^2 + 2 A(1-A) \frac{\E{D}}{\E{S_0}}.
  \end{equation*}
\begin{solution} 
Since $A=1/(1+\lambda_f \E D)$, 
  \begin{equation*}
    \begin{split}
    \lambda \E{D^2} A^2 
&= 2\lambda (\E D)^2 A^2 = 2 \lambda \E D A A \E D \\
&= 2 \frac{\lambda \E D }{1+\lambda \E D} A \E D \\
&= 2 \left(1-\frac{1}{1+\lambda \E D}\right) A \E D  = 2(1-A)A \E D.
    \end{split}
  \end{equation*}
\end{solution}
\end{exercise}

Again, we have all elements ready to use the $G/G/1$ waiting time formula. Let's illustrate this. 

\begin{exercise}
  Suppose we have a machine with memoryless failure behavior, with a mean-time-to-fail of $3$ hours, Regular service times are deterministic with an average of 10 minutes, jobs arrive as a Poisson process with rate of 4 per hour.  Repair times are exponential with a mean duration of 30 minutes. What is the average sojourn time?
  \begin{hint}
    Mind to work in a consistent set of units, e.g., hours. Its easy to make mistakes. 
  \end{hint}
  \begin{solution}
Let's first compute the load. If $\rho>1$ we are in trouble.
    \begin{pyconsole}
labda = 4.
ES0 = 10./60 # in hours
labda_f = 1./3
ED = 30./60 # in hours
A = 1./(1+labda_f*ED)
A
ES = ES0/A
ES
rho = labda*ES
rho
    \end{pyconsole}
As $\rho<1$, the system is not in overload. Now for the queueing time.
\begin{pyconsole}
Ca2 = 1.
C02 = 0. # deterministic service times
Ce2 = C02 + 2*A*(1-A)*ED/ES0
Ce2
EW = (Ca2+Ce2)/2 * rho/(1-rho) * ES
EW
EW + ES # sojourn time
\end{pyconsole}
  \end{solution}
\end{exercise}

\begin{exercise}
  Suppose we could buy another machine that never fails. What is the average sojourn time?
  \begin{solution}
Now we don't need to take availability into account: the machine never fails so $A=1$. 
    \begin{pyconsole}
labda = 4.
ES0 = 10./60 # in hours
A = 1
ES = ES0/A
rho = labda*ES
rho
Ca2 = 1.
C02 = 0. # deterministic service times
EW = (Ca2+C02)/2 * rho/(1.-rho) * ES
EW
EW + ES # sojourn time
\end{pyconsole}
The average time in queue reduces for $\approx 0.6$ to $\approx 0.17$ hours, a reduction by about a factor 3. 
\end{solution}
\end{exercise}

\begin{comment}
  
Consider the compound Poisson process such that $S(t) = \sum_{i=1}^{N(t)} D_i$, where $N(t)$ is the number of customers arrived up to time $t$, and $D_i$ is the service demand of the $i$th customer. The demands $\{D_i\}$ are i.i.d. with mean $\E D$ and variance $\V D$, and $N(t)$ is Poisson distributed with rate $\lambda$. 

\begin{exercise}
First assume that $N$ is some general random variable. Use the tools you learned above to show that
\begin{equation*}
\V{S(t)} = \E{N(t)} \V D + \V{N(t)} (\E D)^2.
\end{equation*}
\begin{solution}
The mean is $\E{S(t)}= \E{N(t)} \E D$.

It remains to compute $\E{S^2(t)}$.  If $N=n$, 
\begin{equation*}
  \begin{split}
  \E{ S_n^2 } 
&=  \E{\left(\sum_{i=1}^n D_i\right)^2}  
=  \E{\sum_i D_i^2 + \sum_i \sum_{j\neq i}^n D_i D_j}  \\
&= \sum_i \E{D^2_i} + \sum_i \sum_{j \neq i} \E{D_i} \E{D_j},
  \end{split}
\end{equation*}
since $\E{D_iD_j} = \E{D_i} \E{D_j}$ by independence. Therefore, using
that the $\{D_i\}$ are i.i.d. as $D$,
\begin{equation*}
  \E{ S_n^2} = n \E{D^2} + n(n-1) (\E D)^2. 
\end{equation*}

Now take $N$ as a random variable, with $p_n=\P{N=n}$,
\begin{equation*}
  \begin{split}
\E{ S_N^2}  &= 
\sum_n p_n(n\E{D^2} + n(n-1)(\E D)^2 \\
&=\E N \E{D^2} + (\E{N^2} - \E N)(\E D)^2 \\
%& = \E N \E{ D^2} + \E{N^2} (\E D)^2 - \E N (\E D)^2 \\
& = \E N (\E{ D^2}  - (\E D)^2) + \E{N^2} (\E D)^2 \\
& = \E N \V D + \E{N^2} (\E D)^2.
  \end{split}
\end{equation*}

Hence, 
\begin{equation}
  \begin{split}
  \V{S_N} 
&= \E{S_N^2}  - (\E{S_N})^2 \\
&= \E N \V D + \E{N^2} (\E D)^2   - (\E N)^2 (\E D)^2 \\
&= \E N \V D + \V N (\E D)^2.
  \end{split}
\end{equation}

\end{solution}
\end{exercise}

\begin{exercise}
  Now assume that $N(t)$ is Poisson with rate $\lambda$. Then, show that 
\begin{equation}
  \V{S_{N(t)}} = \lambda t \E{D^2}.
\end{equation}
\begin{solution}
\begin{equation}
  \begin{split}
  \V{S_{N(t)}}  = \E N \V D + \V N (\E D)^2 =\lambda t \V D + \lambda t (\E D)^2 = \lambda t \E{D^2}.
  \end{split}
\end{equation}
  
\end{solution}
\end{exercise}
\end{comment}

\begin{comment}
\begin{exercise}
Zijm.Ex.1.11.1
 \begin{solution}
Yes. The availability is $93\%$. Since $\E{S} = \E{S_0}/A$, and if $\E{S_0} $ is known, $\E{S_e}$ follows.
\end{solution}
\end{exercise}

\begin{exercise}
Zijm.Ex.1.11.2
 \begin{solution}
   No, from Zijm.Eq.1.51, the average repair time has to be known. The
   repair time here is of course the time it takes for a mechanic to
   show up at work again.
\end{solution}
\end{exercise}
\begin{exercise}
Zijm.Ex.1.11.3
 \begin{solution}
   It might, but perhaps a normal distribution would be better. It
   makes sense to make a histogram of the recover times to see whether
   some clear pattern is present. 

   Besides this, I don't know how sensitive the result for $C_s^2$ is
   on the distribution of the repair times. Perhaps it is not that
   sensitive, so in that case it would be ok to simply use the
   exponential distribution. 

The sensitivity study would be an interesting topic for simulation. 
\end{solution}
\end{exercise}
\begin{exercise}
Zijm.Ex.1.11.4 and 1.11.5
 \begin{solution}
   We need to make an assumption about the distribution of the repair
   times. Inferring from the text, the repair time is always $2$
   days. Lets also assume that all jobs accumulate in front of the
   broken machine, in other words, the broken machine is part of the
   job routing of each job. Then
   \begin{equation*}
     \P{N(2) > 20} = 
\sum_{n=21}^\infty e^{-5\cdot 2} \frac{10^n}{n!}.
1  - \sum_{n=0}^{20} e^{-5\cdot 2} \frac{10^n}{n!} = 1-0.9984,
   \end{equation*}
i.e., very small.  This is the code I used: 

\begin{pyconsole}
from math import exp, factorial
exp(-10) * sum((10)**n / factorial(n) for n in range(21))
\end{pyconsole}
\end{solution}
\end{exercise}

\begin{exercise}
Zijm.Ex.1.11.4

 \begin{solution}
If the shop already contains 10 jobs, somewhere upstream of the broken machine, then 
   \begin{equation*}
     \P{N(2) > 10} = 1- 0.58.
   \end{equation*}
where  I used
\begin{pyconsole}
exp(-10) * sum((10)**n / factorial(n) for n in range(11))
\end{pyconsole}
\end{solution}
\end{exercise}

\begin{exercise}
Zijm.Ex.1.11.6
 \begin{solution}
   Cleaning times will be pretty constant. Changing dies, or other
   machine parts, is also typically quite predictable, although it can
   take a lot of time, in particular in case a crane or other heavy
   machinery is needed to replace parts. If the machine require
   temporary adjustments, then the variation in setup times may be
   quite a bit higher.
\end{solution}
\end{exercise}

\begin{exercise}
Zijm.Ex.1.11.7
 \begin{solution}
   Then the effective service times, and in particular, $C_s^2$ will
   be quite a bit bigger. It is preferable to avoid such a situation. 

   Mathematically, it is only given that $N_s$ is a random
   variable. As, however, this does not state anything about its
   distribution, we cannot make any general claim. The intent of the
   problem is to have you check the relevant formulas and notice that
   the variance of $N_s$ appears in the formulas.
\end{solution}
\end{exercise}

\end{comment}

\begin{comment}
  
\begin{exercise}
  Suppose a single-machine workstation has to process $n$ different
  product families. Jobs of family $i$ arrive as a Poisson process
  with rate $\lambda_i$.  Service times are exponentially distributed
  with average $\mu_i^{-1}$ for family $i$. A setup of time $S$ is
  required if the machine switches from one family to another. If the
  machine works according to a cyclic schedule, i.e., produce family
  1, then 2, then 3, and so on until family $n$, can you find batch
  sizes $B_i$ for family $i$ such that an average waiting time can be
  guaranteed?
  \begin{solution}
    \TBD. 

    This problem can be generalized and has been widely studied as one
    of the core problems in stochastic machine scheduling. For
    instance, what to do if the setup times are family-dependent, or
    even more general, dependent on the sequence in which the families
    are planned? This is a very hard problem. 
  \end{solution}
\end{exercise}
\end{comment}





\Closesolutionfile{hint}
\Closesolutionfile{ans}

\opt{solutionfiles}{
\subsection*{Hints}
\input{hint}
\subsection*{Solutions}
\input{ans}
}


%\clearpage


%%% Local Variables:
%%% mode: latex
%%% TeX-master: "../queueing_book"
%%% End:
