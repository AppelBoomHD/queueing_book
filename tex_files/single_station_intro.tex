In this chapter we start with a discussion of the Poisson process.
We then construct queueing processes in discrete time and apply the Poisson process to model the number of arrivals in periods of fixed length.
In Section~\ref{sec:expon-distr} we relate the exponential distribution to the Poisson distribution.
The exponential distribution often serves as a good model for inter-arrival times of individual jobs.
As such this is a key component of the construction of queueing processes in continuous time.
As it turns out, both ways to construct queueing processes are easily implemented as computer programs, thereby allowing us to use simulation to analyze queueing systems.
In passing we develop a number of performance measures to provide insight into the (transient and average) behavior of queueing processes.

We assume that you  \emph{know all} results of Section~\ref{sec:preliminaries}. 




%%% Local Variables:
%%% mode: latex
%%% TeX-master: "../queueing_book"
%%% End:
