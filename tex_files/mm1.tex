\section
[$M/M/1$ queue]
{$\mathbf{M/M/1}$ queue}
\label{sec:mm1}


\subsection*{Theory and Exercises}

\Opensolutionfile{hint}
\Opensolutionfile{ans}

In the $M/M/1$ queue, one server serves jobs arriving with
exponentially distributed inter-arrival times and each job requires an
exponentially distributed processing time.  With Eq.~(\ref{eq:25}),
i.e., $\lambda(n)p(n)= \mu(n+1)p(n+1)$ we can derive a number of
important results for this queueing process.

Recall from Section~\ref{sec:queu-proc-as} that we can construct the
$M/M/1$ queue as a reflected random walk where the arrivals are
generated by a Poisson process $N_\lambda(t)$ and the departures
(provided the number $L(t)$ in the system is positive) are generated according to the
Poisson process $N_\mu(t)$. Since the rates of these processes do not
depend on the state of the random walk, or the queue for that matter,
$\lambda(n)=\lambda$ for all $n \geq 0$ and $\mu(n)=\mu$ for all $n \geq 1$. Thus, \eqref{eq:25}
reduces to
\begin{equation*}
  p(n+1) = \frac{\lambda(n)}{\mu(n+1)} p(n) = \frac{\lambda}{\mu} p(n) = \rho p(n),
\end{equation*}
where we use the definition of the load $\rho=\lambda/\mu$. Since this
holds for any $n\geq 0$, it follows with recursion that
\begin{equation*}
  p(n+1) = \rho^{n+1} p(0).
\end{equation*}
Then, from the normalization condition
\begin{equation*}
1=  \sum_{n=0}^\infty p(n) = p(0)\sum_{n=0}^\infty \rho^n = \frac{p(0)}{1-\rho},
\end{equation*}
it follows that
\begin{align}\label{eq:23}
p(0) &=1-\rho, &   p(n) &=  (1-\rho)\rho^{n}.
\end{align}

How can we use these equations? First, note that $p(0)$ must be the
fraction of time the server is idle. Hence, the fraction of time the
server is busy, i.e., the utilization, is
\begin{equation*}
  1-p(0) = \rho = \sum_{n=1}^\infty p(n).
\end{equation*}
Here the last equation has the interpretation of the fraction of time
the system contains at least 1 job. 


\begin{exercise}\label{ex:12}
Show that the average number of jobs in an  $M/M/1$ queue is given by
\begin{equation}\label{eq:el}
  \E L = \frac \rho{1-\rho}.
\end{equation}
\begin{hint}
There are various ways to get this result, one is with indicator functions, another with moment generating functions.   
\end{hint}
  \begin{solution}
With indicators, a bit long, but I spell out every step.
\begin{align*}
\E L &= \sum_{n=0}^\infty n p(n) \\
&= \sum_{n=0}^\infty \sum_{i=1}^n \1{i\leq n} p(n)  && n=\sum_{i=1}^n \1{i\leq n}\\
&= \sum_{n=0}^\infty \sum_{i=1}^\infty   \1{i\leq n} p(n)  && i>n\implies \1{i\leq n} = 0\\
&= \sum_{i=1}^\infty \sum_{n=0}^\infty  \1{i\leq n} p(n) &&\text{Fubini} \\
&= \sum_{i=1}^\infty \sum_{n=i}^\infty p(n) && n < i \implies \1{i\leq n}=0 \\
&= \sum_{i=1}^\infty \sum_{n=i}^\infty (1-\rho)\rho^n && p(n) = (1-\rho)\rho^n \\
&= \sum_{i=1}^\infty \sum_{n=0}^\infty (1-\rho)\rho^{n+i} && n\to n+i \\
&= \sum_{i=1}^\infty (1-\rho)\rho^i \sum_{n=0}^\infty \rho^n && \rho^{n+i}=\rho^i \rho^n\\
&= \sum_{i=1}^\infty (1-\rho)\rho^i \frac1{1-\rho}   \\
&= \sum_{i=1}^\infty \rho^i \\
&= \sum_{i=0}^\infty \rho^{i+1} && i\to i+1\\
&= \rho \sum_{i=0}^\infty \rho^i \\
&= \frac{\rho}{1-\rho}.
\end{align*}
Note that, since the summands are positive, we can use Fubini's theorem
to justify the interchange of the summations.

With moment generating functions it is a bit shorter. 
\begin{equation*}
  \begin{split}
  M_L(s) 
&= \E{e^{s L}} = \sum_{n=0}^\infty e^{s n}p(n) = (1-\rho) \sum_n e^{s n} \rho^n \\
&=\frac{1-\rho}{1-e^{s}\rho},
  \end{split}
\end{equation*}
where we assume that $s$ is such that $e^s \rho < 1$. Then, 
\begin{equation*}
  M_L'(s) = (1-\rho) \frac{1}{(1-e^s\rho)^2} e^s \rho.
\end{equation*}
Hence, $\E L = M_L'(0) = \rho/(1-\rho)$.
  \end{solution}
\end{exercise}

From the previous exercise, it follows that
\begin{equation*}
 \E L \sim \frac 1{1-\rho}, \quad \text{ as }\rho\to 1.
\end{equation*}
Let us interpret this expression. The fact
that $\E L \sim (1-\rho)^{-1}$ for $\rho\to 1$ implies that the
average waiting time increases very fast when $\rho\to1$.  If we want to avoid  long waiting times, this formula tells us that  situations with
$\rho\approx 1$ should be avoided. As a practical guideline, it is typically best to  keep $\rho$ quite a bit below 1, and accept that servers are not fully  utilized. 


\begin{exercise}
  Show that the excess probability, i.e., the probability that a long queue occurs, is 
$\P{L\geq n} = \rho^n$.
  \begin{hint}
$\P{L\geq n} = \sum_{k\geq n} p(k)$.
  \end{hint}
  \begin{solution}
    \begin{equation*}
      \begin{split}
 \P{L\geq n} 
 &= \sum_{k=n}^\infty p(k) = \sum_{k=n}^\infty p(0)\rho^k = (1-\rho)\sum_{k=n}^\infty \rho^k \\
 &= (1-\rho)\rho^n \sum_{k=0}^\infty\rho^k = (1-\rho) \rho^n \frac1{1-\rho} = \rho^n.
\end{split}
\end{equation*}
\end{solution}
\end{exercise}

Clearly,  the probability that the queue length exceeds some threshold decreases geometrically fast (for  $\rho<1$). If we make the simple assumption
that customers decide to leave (or rather, not join) the system when
the queue is longer than $9$ say, then $\P{L\geq 10} = \rho^{10}$ is
an estimator of the fraction of customers lost. 

% In the context of inventory theory these equations are particularly
% useful, see one of the questions below.


\paragraph{Relation between Inventory and Queueing Systems}

There exists an interesting relation between inventory and queueing systems, cf. Figure~\ref{fig:inv_queue}.  When a job arrives in the queueing system, the virtual workload~$V(t)$ increases by the service time of the job. In the inventory system, the inventory $I(t)$ decreases by the  demand size of the customer. Like this, the demand size of a customer at the inventory system converts into a production time (i.e., a service time) of a job at a queueing  system. In the figure, the demand size $D_1$  of the first customer corresponds to a production time of duration  $S_1=D_1$, and so on. Thus, even though in the inventory system, customers do not have to wait, their demands spawn production times at a server who has to replenish the consumed items. 

Assume now that  the inventory process is controlled by an order-up-to policy: produce (refill the inventory) as long as the inventory level is below $S$ and stop otherwise.  Then the figure shows that the inventory level $I(t)$ is equal to $S-V(t)$.   

In more general terms, in a queueing system or an inventory system, there is always `something' or `somebody'  waiting. Items in a supermarket are produced ahead of the moment they are `consumed', hence they `wait' for customers. In a queueing system, customers are waiting  while their product is `produced' by the server. When there are no customers in a queueing system, the server waits until a new customers comes along. Thus, queueing and inventory theory are both focused on the distribution of waiting times, either by customers, servers, or items, hence  both are related branches of (applied) probability theory.


\begin{figure}[th]
\begin{center}
\begin{tikzpicture}[yscale=0.5]
\draw[->] (0,0) -- coordinate (x axis mid) (8.5,0);
\draw[->] (0,0) -- coordinate (y axis mid) (0,10.5);
\node[below=0.2cm] at (x axis mid) {$t$};

\draw plot coordinates {(1,0) (1,2) (2,1) (2,4) (4,2) (4,4.2) (7.5,0)};
\node[left]  at (7,2.5) {$V(t)$};
\node[fill=white, rotate=90]  at (1,1) {$S_1$};
\node[fill=white, rotate=90]  at (2,2.5) {$S_2$};
\node[fill=white, rotate=90]  at (4,3.) {$S_3$};

\node  at (7,5) {$V(t)=S-I(t)$};

\draw[dotted] (0,10)--(8.5,10);
\node[left]  at (0,10) {$S$};
\node[left]  at (7,7.5) {$I(t)$};
\draw plot coordinates {(1,10) (1,8) (2,9) (2,6) (4,8) (4,6.0) (7.5,10)};
\node[fill=white, rotate=90]  at (1,9) {$D_1$};
\node[fill=white, rotate=90]  at (2,7.5) {$D_2$};
\node[fill=white, rotate=90]  at (4,7) {$D_3$};
\end{tikzpicture}
\end{center}
\caption{The relation between inventory and queueing systems. Here $I(t)$ models the evolution of the inventory level in an  inventory system, while $V(t)$ shows the virtual workload, and $S$ is the order up to level. When a customer requires $D_1$ items, say, it takes a server of a time $S_1=D_1$ to produce these items. Thus, demands at the inventory system convert into production times in a queueing system.  When the inventory is always replenished to level $S$, then the shortage of the inventory level relative to $S$, i.e., $S-I(t)$, becomes the workload $V(t)$ for the server in terms of amount of items or production time.} \label{fig:inv_queue}
\end{figure}


\begin{exercise}\label{ex:7}
Customers of fast-food restaurants prefer to be served from stock. For this reason such
restaurants often use a `produce-up-to' policy: When the on-hand inventory $I$ is equal or lower than some threshold $S-1$, the company produces items until the inventory level equals $S$ again. The level $S$ is known as the order-up-to level, and $S-1$ as the reorder level.

Suppose that customers arrive as a Poisson process with rate $\lambda$
and the production times of single items are i.i.d. and exponentially
distributed with parameter $\mu$. Assume also that customers who
cannot be served from on-hand stock are backlogged, that is, they wait
until their item has been produced. What are the average on-hand
inventory level and the average number of customer in backlog?
\begin{hint}
Use Figure~\ref{fig:inv_queue} to realize that the inventory level $I(t)$ (here measured in the number of items on stock) at time $t$ can be
  modeled as $I(t) = S+1-L(t)$, where $L$ is the number of jobs in an
  $M/M/1$ queue. Note that the number of jobs in queue corresponds to the number of items to be produced. A customer of an item, i.e., a demand of the inventory system, turns into  job at the queueing system, i.e., the demand becomes a job for the cook to replenish the item.
\end{hint}
  \begin{solution}
In the $M/M/1$ queue, $p(n)$ corresponds to the fraction of time there are $n$ jobs in the system. In the inventory system, it means that the cook has $n$ jobs to satisfy, hence the inventory is $n$ jobs short relative to the order up to level $S$.  Hence, the average on-hand inventory is $S$ minus the average number of jobs at the cook. i.e., $\E{I} = \sum_{i=0}^{S} (S-i) p(i)$. The average number of customers in backlog is the fraction of time there are more than $S$ replenishment jobs at the cook, i.e., $\E B = \sum_{i=S+1}^\infty (i-S) p(i)$. 
  \end{solution}
\end{exercise}


\paragraph{Supermarket Planning}

Let us consider the example of cashier planning of a supermarket to
demonstrate how to use the tools we developed up to now. Out of
necessity, our approach is a bit heavy-handed---Turning the example
into a practically useful scheme requires more sophisticated queueing
models and data assembly---but the present example contains the
essential analytic steps to solve the planning problem.

The \emph{service objective} is to determine the minimal service
capacity $c$ (i.e., the number of cashiers) such that the fraction of the time that more than 
10 people are in queue is less than 1\%. (If the supermarket has 3 cashiers open, 10 people in queue  means about 3 people per queue.)

The next step is to find the \emph{relevant data}: the arrival process and the service time distribution. For the arrival process it is reasonable to model it as a Poisson process. There are many potential customers, each choosing with small probability to go the supermarket on a certain moment in time. Thus, we only have to  characterize the arrival rate. Estimating this for a supermarket  is relatively easy: the cash registers track all customers
payments. Thus, we know the number of customers that left the shop,
hence entered the shop. (We neglect the time customers spend in the
shop.) Based on these data we make a \emph{demand profile}: the
average number of customers arriving per hour, cf. Figure~\ref{fig:loadprofile}. Then we model the arrival process as Poisson with an arrival rate that is constant during a certain hour and is specified by the demand profile. 

\begin{figure}[t]
  \centering
\begin{tikzpicture}[scale=.7]
 	%axis
	\draw[->] (0,0) -- coordinate (x axis mid) (13.5,0);
    	\draw[->] (0,0) -- coordinate (y axis mid) (0,5.5);
    	%ticks
    	\foreach \x in {0,...,13}
        \pgfmathsetmacro{\my}{int(\x+8)}
     		\draw (\x,1pt) -- (\x,-3pt)
			node[anchor=north] {$\my$};
    	\foreach \y in {0,...,5}
        \pgfmathsetmacro{\my}{int(\y*40)}
     		\draw (1pt,\y) -- (-3pt,\y) 
     			node[anchor=east] {\my}; 
%labels      
\node[below=0.6cm] at (x axis mid) {hour};
\node[rotate=90, left=1.2cm] at (y axis mid) {$\lambda$};

\draw (0,1)--(1,1);
\draw (1,2)--(2,2);
\draw (2,2.6)--(3,2.6);
\draw (3,2.8)--(4,2.8);
\draw (4,3.)--(5,3.);
\draw (5,3.1)--(6,3.1);
\draw (6,2.7)--(7,2.7);
\draw (7,1.9)--(8,1.9);
\draw (8,2.5)--(9,2.5);
\draw (9,3.3)--(10,3.3);
\draw (10,3.5)--(11,3.5);
\draw (11,2.3)--(12,2.3);
\draw (12,1.2)--(13,1.2);
\end{tikzpicture}
  \caption{A  demand profile of the arrival rate $\lambda$ modeled as constant over each hour.}
  \label{fig:loadprofile}
\end{figure}


It is also easy to find the service distribution from the cash
registers. The first item scanned after a payment determines the start
of a new service, and the payment closes the service. (As there is
always a bit of time between the payment and the start of a new
service we might add 15 seconds, say, to any service.)
To keep things simple here, we just model the service time distribution as
exponential with a mean of $1.5$ minutes. 

We also \emph{model} the behavior of all the cashiers together (a multi-server queue) as a single fast server. Thus, we neglect any differences between a station with, for instance, 3 cashiers and a
single server that works 3 times as fast as a normal cashier.  As yet
another simplification, we change the objective somewhat such that the
number of jobs in the system, rather than the number in queue, should not exceed 10. 

We now find a formula to convert the demand profile into the \emph{load profile}, which is the minimal number of servers per hour needed to meet the service objective. We already know for the $M/M/1$ that $\P{L>10}=\rho^{11}$.  Combining this with the objective $\P{L>10}\leq 1\%$, we get that $\rho^{11}\leq 0.01$, which  translates into $\rho \leq 0.67$. Using that $\rho = \lambda \E S/c$ and our estimate $\E{S}=1.5$ minutes,  we get  the following rough bound on $c$:
\begin{equation*}
c \geq \frac{\lambda \E S}{0.67} \approx \frac{3}2 \cdot \lambda \cdot 1.5  \approx 2.5 \lambda,
\end{equation*}
where $\lambda$ is the arrival rate (per minute, \emph{not} per hour).
For instance, for the hour from 12 to 13, we read in  the demand profile that $\lambda= 120$ customers per hour, hence $c=2.5 \cdot 120/60 = 5$. With this formula, the conversion of the demand profile to the load profile becomes trivial: divide the hourly arrival rate by $60$ and multiply by
$2.5$.

The last step is to \emph{cover the load profile with service shifts}. This
is typically not easy since shifts have to satisfy all kinds of rules,
such as: after 2 hours of work a cashier should take a break of at least
10 minutes; a shift length must be at least four hours, and not longer
than 9 hours including breaks; when the shift is longer than 4 hours
it needs to contain at least one break of 30 minutes; and so on. These
shifts also have different costs: shifts with hours after 18h
are more expensive per hour; when the supermarket covers traveling
costs, short shifts have higher marginal traveling costs; and so on.

The usual way to solve such  covering problems is by means of an integer
problem. First generate all (or a subset of the) allowed shift types
with associated starting times. For instance, suppose only 4 shift
plans are available
\begin{enumerate}
\item $++-++$
\item $+++-+$
\item $++-+++$
\item $+++-++$,
\end{enumerate}
where a $+$ indicate a working hour and $-$ a break of an hour. Then
generate shift types for each of these plans with starting times
$8$am, $9$am, and so on, until the end of the day. Thus, a shift type
is a shift plan that starts at a certain hour. Let $x_i$ be the number
of shifts of type $i$ and $c_i$ the cost of this type. Write $t\in s_i$ if
hour $t$ is covered by shift type $i$.  Then the problem is to solve
\begin{equation*}
  \min \sum_i c_i x_i,
\end{equation*}
such that 
\begin{equation*}
  \sum_i x_i \1{t \in s_i} \geq 2.5 \frac{\lambda_t}{60}
\end{equation*}
for all hours $t$ the shop is open and $\lambda_t$ is the demand for
hour $t$.




\Closesolutionfile{hint}
\Closesolutionfile{ans}

\opt{solutionfiles}{
\subsection*{Hints}
\input{hint}
\subsection*{Solutions}
\input{ans}
}



%%% Local Variables:
%%% mode: latex
%%% TeX-master: "../queueing_book"
%%% End:
