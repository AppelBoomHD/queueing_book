\opt{solutionfiles}{
\section{Old exam Questions}
\label{sec:expon-distr}

\Opensolutionfile{ans}
\subsection{Multiple-choice Questions}
}

\begin{extra}[201904, 1]
  One server serves two queues and has capacity $c$ per period available.
  %Each queue receives service capacity in period $k$ 
  We divide the capacity $c$ over the queues in proportion to the queue lengths $L^i_{k-1}$, $i=1,2$.
  The following implements this rule:
    \begin{align*}
      c_k^1 &= \left\lfloor \frac{L_{k-1}^1}{L_{k-1}^1 + L_{k-1}^2} c + \frac 1 2 \right\rfloor, & c_k^2 &= c - c_k^1,
    \end{align*}
    where $c_k^i$ be the capacity allocated to queue $i$ in period $k$ and we include the rounding to prevent the loss of capacity.

\begin{solution}
      Answer = A, \ref{ex:51}.
      We introduce rounding to prevent the service of `partial' customers.
      For instance, if the first queue contains 1 job, and the second 2, then without rounding we would server $2/3$ customer of the second type.

      Note that the case with empty queues does not lead to a problem.
      When there are no jobs, the service distribution is irrelevant.
\end{solution}
\end{extra}


\begin{extra}[201904, 1]
When  $X\sim\Exp(\lambda)$ its SCV is larger than 1..
\begin{solution} Answer = B, see \ref{ex:29}.
\end{solution}
\end{extra}

\begin{extra}[201904, 1]
  For the $G/G/1$ queue the waiting time satisfies the recursion
  \begin{equation*}
  W_{Q,k} = \max\{W_{Q,k-1} + S_{k}-X_k, 0\}.
  \end{equation*}
\begin{solution} Answer = B, \eqref{eq:56}.
\end{solution}
\end{extra}

\opt{solutionfiles}{
  \subsection{Open Questions}
}

\begin{extra}[201904, 2]
  A machine produces items, but a fraction $p$ of the items does not meet the quality requirements after the first pass at the server, but it requires a second pass.
  Assume that the repair of a faulty item requires half of the work of a new job, and that the faulty jobs are processed with priority over the new jobs.
  Also assume that faulty items do not need more than one repair (hence, faulty items that are repaired cannot be faulty anymore).
  Make a set of discrete-time recursions to analyze this case.
  \begin{solution}
See \ref{ex:52}
  \end{solution}
\end{extra}


\opt{solutionfiles}{
\Closesolutionfile{ans}
\subsection*{Solutions}
\input{ans}
}



%%% Local Variables:
%%% mode: latex
%%% TeX-master: "../companion"
%%% End:
