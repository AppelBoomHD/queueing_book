\section{Level-Crossing and Balance Equations}
\label{sec:level-cross-balance}


\subsection*{Theory and Exercises}

\Opensolutionfile{hint}
\Opensolutionfile{ans}


Consider a system at which customers arrive and depart in single
entities, such as customers in a shop or jobs at some machine.  If the
system starts empty, then we know that the number $L(t)$ is the system
at time $t$ is equal to $A(t) - D(t)$. To illustrate:

\begin{figure}[h]
  \centering
\begin{tikzpicture}[scale=1]
\draw[->] (0,0)--node[midway, fill=white] {$A(t)$}  (2,0); 
\draw (2,-0.25) rectangle node {$L(t)=A(t)-D(t)$} (5,0.25);
\draw[->] (5,0)--node[midway, fill=white] {$D(t)$}  (7,0); 
\end{tikzpicture}
\end{figure}

\noindent What goes in the box (i.e., $A(t)$) and has not yet left
  (i.e., $D(t)$) must still be in the box, hence $L(t)=A(t)-D(t)$). 



Let us denote an arrival as an `up-crossing' and a departure as a
`down-crossing'.  Then, clearly $L(t)$ is the number of up-crossings
up to time $t$ minus the number of down-crossings up to time $t$. If
$L(t)$ remains finite, or, more generally, $L(t)/t \to 0$ as $t\to\infty$, then it
must be that
\begin{equation*}
  \lambda =  \lim_{t \to \infty} \frac{A(t)}t  = \lim_{t \to \infty} \frac{D(t)+L(t)}t =  \lim_{t \to \infty} \frac{D(t)}t + \lim_{t \to \infty} \frac{L(t)}t 
  = \delta.  
\end{equation*}
Hence, when $L(t)/t\to0$, the \recall{up crossing rate}
$\lim_{t \to \infty} A(t)/t = \lambda$ is equal to the \emph{down-crossing rate}
$\lim_{t \to \infty} D(t)/t = \delta$.  We will generalize these notions of up- and
downcrossing in this section to derive the \recall{stationary}---also known as \emph{long-run time average} or \emph{steady-state},
distribution---$p(n)$ that the system contains $n$ jobs. 

\begin{exercise}
  If $L(t)/t \to 0$ as $t\to\infty$, can it still be true that $\E{L}>0$? 
  \begin{solution}
    \begin{equation*}
      \E{L} = \lim_{t\to\infty} \frac 1 t \int_0^t L(s) \d s \neq \lim_{t\to\infty} \frac{L(t)}t.
    \end{equation*}
If $L(t)=1$ for all $t$, $\E{L} =1 $, but $L(t)/t \to 0$. 
  \end{solution}
\end{exercise}


Let us say that the system is in \emph{state $n$} at time $t$ when it contains $n$
  jobs at that moment, i.e., when $L(t) = n$.  The system \emph{crosses}
level $n$ at time $t$ when its state changes from $n$ to $n+1$, either `from
below' due to an  arrival, or `from above' due to  a departure, cf. Figure~\ref{fig:A_n_t}.

\begin{figure}[th]
  \centering
\begin{tikzpicture}[scale=1,->,>=stealth',shorten >=1pt,auto,node distance=1.8cm,
                    semithick]
  \node[state] (0) {$p(0)$} ;
  \node[state] (1) [right of=0] {$p(1)$};
  \node[state] (2) [right of=1] {$p(2)$};
  \node[state] (3) [right of=2] {$p(3)$};
  \node[state] (4) [right of=3] {$\cdots$};

\path 
 (0) edge [bend left] node {$\lambda(0)$} (1)
 (1) edge [bend left] node {$\mu(1)$} (0)
 (1) edge [bend left] node {$\lambda(1)$} (2)
 (2) edge [bend left] node {$\mu(2)$} (1)
 (2) edge [bend left] node {$\lambda(2)$} (3)
 (3) edge [bend left] node {$\mu(3)$} (2)
 (3) edge [bend left] node {$\lambda(3)$} (4)
 (4) edge [bend left] node {$\mu(4)$} (3);

\draw[-, dotted, gray] (2.7,-2.)--(2.7,2.0) node[above, black] {level $1$};
\draw[->] (2,1.5)  node[left] {$A(1,t)$} -- (3.5,1.5);
\draw[<-] (2,-1.2) node[left] {$D(1,t)$} --(3.5,-1.2) ;

\end{tikzpicture}
\caption{ $A(1,t)$ counts the number of jobs up to time $t$ that saw 1
  job in the system upon arrival, and right after such arrivals the
  system contains 2 jobs.  Thus, each time $A(1,t)$ increases by
  one, level $1$ (the dotted line  separating states 1 and 2) is crossed from below.  Similarly, $D(1,t)$ counts the number of
  departures that leave 1 job behind, and just before such departures the system contains 2 jobs. Hence, level $1$ is crossed from above. 
It is evident that the number of times this
  level is crossed from below must be the same (plus or minus 1) the
  number of times it is crossed from above. (We introduce $\lambda(n)$, $\mu(n)$ and $p(n)$ below.) }
\label{fig:A_n_t}
\end{figure}


To establish the section's main result Eq.~\eqref{eq:12}  we need a few definitions
that are quite subtle and might seem a bit abstract, but below we will
provide intuitive interpretations in terms of system KPIs. Once we
have the proper definitions, the above result will follow
straightaway. Figure~\ref{fig:summaries} at the end of the section summarizes all concepts we  develop here.

\paragraph{Level crossing}




Define
\begin{subequations}\label{eq:rates_}
\begin{equation}\label{eq:19} 
  A(n,t) = \sum_{k=1}^\infty \1{A_k \leq t}\1{L(A_k-) = n}
\end{equation}
as the number of arrivals up to time $t$ that saw $n$ customers in the system at their arrival.

\begin{exercise}
  Why do we  take $L(A_k-)=n$ rather than $L(A_k)$ in the definition of $A(n,t)$?
  \begin{hint}
Recall that $L(t)$ is \emph{right-continuous}.
  \end{hint}
    \begin{solution}
    $L(t)$ is the number of customers in the system at time $t$. As
    such the function $t\to L(t)$ is \textsl{right-continuous}. The
    definition of $L(A_k-) = \lim_{t\uparrow A_k} L(t)$ is the
    limit from the left.  The customer therefore `sees' $L(A_k-)$ just
    before he/she arrives. 
\end{solution}  
\end{exercise}

\begin{exercise}
What is the difference between $A(n,t)$ and $A(t)$? 
\begin{solution}
  $A(t)$ counts all customers that arrive up to time $t$, i.e., during
  $[0,t]$. Note that this \textsl{includes} time $t$. $A(n,t)$ counts
  the jobs that see $n$ jobs in the system just before they arrive.
    \end{solution}
\end{exercise}

\begin{exercise}
 Show that $A(n,t) \leq A(t)$. 
\begin{solution}
       Observe that
      $\1{A_k\leq t}\1{L(A_k-) = n} \leq \1{A_k\leq t}$;
      the last inequality follows from the fact that
      $\1{L(A_k-) = n}\leq 1$. Therefore,
    \begin{equation*}
  A(n,t) = \sum_{k=1}^\infty \1{A_k \leq t} \1{L(A_k-) = n} 
\leq \sum_{k=1}^\infty \1{A_k \leq t} = A(t). 
    \end{equation*}
    For any `normal' queueing system, $A(t) > A(n,t)$, because the
    queue length fluctuates.
    \end{solution}
\end{exercise}



\begin{exercise}
  Why is $ \lim_{t\to\infty} \frac{A(n,t)}t = 0$ if
  $\lambda > \delta$, i.e., if the system is not rate-stable?
  \begin{solution}
 If $\lambda > \delta$, then $L(t)\to\infty$. But then there
      must be a last time, $s$ say, that $L(s) = n+1$, and $L(t) > n+1$
      for all $t>s$. Hence, after time $s$ no job will see the system
      with $n$ jobs. Thus $A(n,t) = A(n,s)$ for all $t>s$.  This is a
      finite number, while $t\to\infty$, so that $A(n,t)/t \to 0$.
  \end{solution}
 \end{exercise}


\begin{exercise}\label{ex:111}
  Consider the following (silly) queueing process. At times
  $0, 2,4, \ldots$ customers arrive, each customer requires $1$ unit
  of service, and there is one server. 
Find an expression for $A(n,t)$. (What acronym would describe this queueing situation?)
  \begin{hint}
For the acronym, observe that the service times and
    inter-arrival are deterministic and there is one server. For
    the computation of $Y(n,t)$, make a plot of $L(s)$ as a
    function of time for $n=1$.

Make a plot of $L(s)$ for $n=1$ as a function of time. 
  \end{hint}
    \begin{solution}
      It is the $D/D/1$ queue, since there is one server and
      the inter-arrival times and service times are constant, i.e.,
      deterministic.


$A_k = 2k$ as jobs arrive at $t=0, 2, 4, \ldots$, hence,  $A(t) \approx t/2$ when $t\gg 0$. We also know that $L(s)=1$ if $s\in [2i, 2i+1)$ and $L(s)=0$ for $s\in[2i-1, 2i)$ for $i=0, 1, 2, \ldots$. Thus, $L(A_k-) = L(2k-)=0$. Hence, $A(0,t) \approx t/2$ for $t\gg 0$, and $A(n,t)=0$ for $n\geq 1$. 
    \end{solution}

\end{exercise}


Next, let 
\begin{equation} \label{eq:17} 
   Y(n,t) = \int_0^t  \1{L(s) = n} \d s
\end{equation}
be  the total time the system contains $n$ jobs during $[0,t]$, and
\begin{equation} \label{eq:18}
   p(n,t) = \frac 1 t \int_0^t  \1{L(s) = n} \d s = \frac{Y(n,t)}t,
\end{equation}
\end{subequations}
be the fraction of time that $L(s) =n$ in $[0,t]$. Figure~\ref{fig:Y_1_t} illustrates the relation between $Y(n,t)$ and $A(n,t)$.
  


\begin{exercise}\label{ex:112} Continuation of Exercise~\ref{ex:111}. 
Find an expression for $Y(n,t)$. 
    \begin{solution}
      Next, to get $Y(n,t)$, observe that the system never contains
      more than 1 job. Hence, $Y(n,t)=0$ for all $n\geq 2$.  Then we see that
      $Y(1,t) = \int_0^t \1{L(s) = 1}\d s.$ Now observe that for our
      queueing system $L(s)=1$ for $s\in[0,1)$, $L(s)=0$ for
      $s\in[1,2)$, $L(s)=1$ for $s\in[2,3)$, and so on. Thus, when
      $t<1$, $Y(1,t)=\int_0^t \1{L(s)=1} \d s = \int_0^t 1\d s = t$.
      When $t\in[1,2)$, 
      \begin{equation*}
        L(t)=0 \implies \1{L(t)=0} \implies Y(1,t) \text{ does not change}.
      \end{equation*}
Continuing to $[2,3)$ and so on gives
    \begin{equation*}
      Y(1,t) =
      \begin{cases}
        t & t\in[0,1), \\
        1 & t\in[1,2), \\
        1+(t-2) & t\in[2,3), \\
        2 & t\in[3,4), \\
        2+(t-4) & t\in[4,5), \\
      \end{cases}
    \end{equation*}
    and so on.  Since $Y(n,t)=0$ for all $n\geq 2$, $L(s) = 1$ or
    $L(s)=0$ for all $s$, therefore, 
    \begin{equation*}
      Y(0,t) = t-Y(1,t).
    \end{equation*}
    \end{solution}
\end{exercise}


\begin{figure}[t]
  \centering
\begin{tikzpicture}[scale=1,
  open/.style={shape=circle, fill=white, inner sep=1pt, draw, node contents=},
  closed/.style={shape=circle, fill=black, inner sep=1pt, draw, node contents=},
]

%axis
\draw[->] (0,0) -- coordinate (x axis mid) (8.5,0);
\draw[->] (0,0) -- coordinate (y axis mid) (0,4.5);
%\node[below=0.2cm] at (x axis mid) {$t$};
\node[left=0.3cm, rotate=90] at (y axis mid) {$L(t)$};


\draw 
node (0) at (0,0) [closed] {}
node (c1) at (1,0) [open] {};
\draw[thick] (0)--(c1);
\node[below] at (1,0) {$A_1$};

\draw 
node (c2) at (1,1) [closed] {}
node (c3) at (2.5,1) [open] {};
\draw[thick] (c2)--(c3);
\node[below] at (2.5,0) {$D_1$};

\draw 
node (c2) at (2.5,0) [closed] {}
node (c3) at (3,0) [open] {};
\draw[thick] (c2)--(c3);
\node[below] at (3,0) {$A_2$};

\draw 
node (c2) at (3,1) [closed] {}
node (c3) at (4,1) [open] {};
\draw[thick] (c2)--(c3);

\draw 
node (c2) at (4,2) [closed] {}
node (c3) at (5,2) [open] {};
\draw[thick] (c2)--(c3);
\node[below] at (4,0) {$A_3$};

\draw 
node (c2) at (5,1) [closed] {}
node (c3) at (5.8,1) [open] {};
\draw[thick] (c2)--(c3);
\node[below] at (5.,0) {$D_2$};

\draw 
node (c2) at (5.8,2) [closed] {}
node (c3) at (6.3,2)  {};
\draw[thick] (c2)--(c3);
\node[below] at (5.8,0) {$A_4$};

\draw[dashed] (1,0)--(2.5, 1.5) -- (3, 1.5)--(4,2.5) -- 
(5, 2.5)--(5.8, 3.3)--(6.3, 3.3);
\node[right] at (6.3, 3.3) {$Y(1,t)$};

\draw[dotted] (4,0) -- (4,0.5)--(5.8, 0.5)--(5.8, 1.5)--(6.3, 1.5);
\node[right] at (6.3, 1.5) {$A(1,t)$};

\end{tikzpicture}
  \caption{Plots of $Y(1,t)$ and $A(1,t)$. (For visual clarity, we subtracted $\nicefrac12$ from $A(1,t)$, for otherwise its graph would partly overlap with the graph of $L$.)}
  \label{fig:Y_1_t}
\end{figure}



Define also the limits:
\begin{align}\label{eq:p(n)}
  \lambda(n) &= \lim_{t\to\infty} \frac{A(n,t)}{Y(n,t)}, &p(n) &=\lim_{t\to\infty} p(n,t),
\end{align}
as the \emph{arrival rate in state $n$} and the \emph{long-run fraction of
  time the system spends in state $n$}. To clarify the former
definition, observe that $A(n,t)$ counts the number of arrivals that
see $n$ jobs in the system upon arrival, while $Y(n,t)$ tracks the amount of time
the system contains $n$ jobs. Suppose that at time~$T$ a job arrives that
sees $n$ in the system. Then $A(n,T)=A(n, T-)+1$, and this job finishes
an interval that is tracked by $Y(n,t)$, precisely because this job
sees $n$ in the system just prior to its arrival. Thus, just as
$A(t)/t$ is the total number of arrivals during $[0,t]$ divided by~$t$, $A(n,t)/Y(n,t)$ is the number of arrivals that see $n$ divided by
the time the system contains $n$ jobs.

\begin{exercise}\label{ex:113} Continuation of Exercises~\ref{ex:111} and~\ref{ex:112}.  Compute $p(n)$  and $\lambda(n)$. 
  \begin{solution}
    From the other exercises:
    \begin{align*}
      \lambda(0) &\approx \frac{A(0,t)}{Y(0,t)} \approx \frac{t/2}{t/2} = 1, \\
      \lambda(1) &\approx \frac{A(1,t)}{Y(1,t)} \approx \frac{0}{t/2} = 0, \\
      p(0) &\approx \frac{Y(0,t)}{t} \approx \frac{t/2}{t} = \frac 1 2, \\
      p(1) &\approx \frac{Y(1,t)}{t} \approx \frac{t/2}{t} = \frac 1 2.
    \end{align*}
For the rest $\lambda(n) = 0$, and $p(n)=0$, for $n\geq 2$.
  \end{solution}
\end{exercise}

Similar as the definition for $A(n,t)$, denote by
\begin{equation*}
    D(n,t) = \sum_{k=1}^\infty \1{D_k \leq t} \1{L(D_k) = n}
  \end{equation*}
  the number of departures up to time $t$ that\emph{ leave $n$
    customers behind}. Then,  define
\begin{equation*}
  \mu(n+1) = \lim_{t\to\infty} \frac{D(n,t)}{Y(n+1,t)},
\end{equation*}
as \emph{ the departure rate from state $n+1$}. (It is easy to get
confused here: to leave $n$ jobs behind, the system must contain $n+1$
jobs just prior to the departure.) Figure~\ref{fig:A_n_t} shows how
$A(n,t)$ and~$\lambda(n)$ relate to $D(n+1,t)$ and $\mu(n)$.

\begin{exercise}
Should  we take $D(n-1,t)$ or  $D(n,t)$ in the definition of $\mu(n)$?
    \begin{solution}
      $D(n-1,t)$ counts the departures that leave $n-1$ behind. Thus,
      just before the customer leaves, the system contains $n$
      customers.
\end{solution}
\end{exercise}

\begin{exercise}\label{ex:4}

Continuation of Exercises~\ref{ex:111},~\ref{ex:112}, and~\ref{ex:113}.  Compute 
$D(n,t)$ and $\mu(n+1)$ for $n\geq 0$.
\begin{solution}
  $D(0,t) = \sum_{k=1}^\infty\1{D_k\leq t, L(D_k)=0}$. From the graph of $\{L(s)\}$ we see that all jobs leave an empty system behind. Thus, $D(0,t) \approx t/2$, and $D(n,t)=0$ for $n\geq 1$. With this, $D(0,t)/Y(1,t) \sim (t/2)/(t/2) = 1$, and so,
  \begin{equation*}
    \mu(1) = \lim_t \frac{D(0,t)}{Y(1, t)} = 1,
  \end{equation*}
and $\mu(n) = 0$ for $n\geq2$. 
\end{solution}
\end{exercise}

Observe that customers arrive and depart as single units. Thus, if
$\{T_k\}$ is the ordered set of arrival and departure times of the
customers, then $L(T_k) = L(T_k-) \pm 1$. But then we must also have
that $|A(n,t) - D(n,t)| \leq 1$ (Think about this.). From this
observation it follows immediately that
\begin{equation}\label{eq:15}
  \lim_{t\to\infty} \frac{A(n,t)}t = \lim_{t\to\infty} \frac{D(n,t)}t.
\end{equation}
With this equation we can obtain two nice and fundamental
identities. The first we develop now; the second follows in
Section~\ref{sec:poisson-arrivals-see}.

The rate of jobs that `see the system with $n$ jobs' can be defined as
$A(n,t)/t$. Taking limits we get
\begin{subequations}
\label{eq:21}
\begin{equation}\label{eq:63}
\lim_{t\to\infty}  \frac{A(n,t)}t =  \lim_{t\to\infty} \frac{A(n,t)}{Y(n,t)}\frac{Y(n,t)}t = \lambda(n) p(n),
\end{equation}
where we use the above definitions for $\lambda(n)$ and $p(n)$.
Similarly, the departure rate of jobs that leave $n$ jobs behind is
\begin{equation}\label{eq:22}
\lim_{t\to\infty}  \frac{D(n,t)}t =  \lim_{t\to\infty} \frac{D(n,t)}{Y(n+1,t)}\frac{Y(n+1,t)}t = \mu(n+1) p(n+1).
\end{equation}
\end{subequations}
Combining this with~\eqref{eq:15} we arrive at \recall{the
  level-crossing equations}
\begin{equation}\label{eq:12}
  \lambda(n) p(n) = \mu(n+1)p(n+1).
\end{equation}

\begin{exercise} Continuation of Exercises~\ref{ex:111},~\ref{ex:112}, and~\ref{ex:113}.  Compute 
$\lambda(n) p(n)$ for $n\geq 0$, and check $\lambda(n) p(n) = \mu(n+1) p(n+1)$. 
\begin{solution}
  $\lambda(0)p(0)=1\cdot 1/2 = 1/2$, $\lambda(n)p(n)= 0$ for $n>1$, as $\lambda(n)=0$ for $n>0$.

From Exercise~\ref{ex:4}, $\mu(1)=1$, hence $\mu(1) p(1) = 1\cdot 1/2 = 1/2$. Moreover, $\mu(n)=0$ for $n\geq 2$. 

Clearly, for all $n$ we have $\lambda(n)p(n)= \mu(n+1)p(n+1)$. 

\end{solution}
\end{exercise}

Result~\eqref{eq:12} turns out to be exceedingly useful, as will become evident from Section~\ref{sec:mm1} onward. More specifically, by specifying (i.e., modeling) $\lambda(n)$ and $\mu(n)$, we can compute the long-run fraction of
time $p(n)$ that the system contains $n$ jobs. To see this, rewrite
the above into
\begin{equation}\label{eq:25}
  p(n+1) = \frac{\lambda(n)}{\mu(n+1)}p(n). 
\end{equation}
Thus, this equation fixes the ratios between the probabilities. In other words, if we know $p(n)$ we can compute $p(n+1)$, and so on. Hence, if $p(0)$ is known, then $p(1)$ follows, from which $p(2)$
follows, and so on. A straightaway iteration then leads to
\begin{equation}\label{eq:38}
  p(n+1) = \frac{\lambda(n)\lambda(n-1)\cdots \lambda(0)}{\mu(n+1)\mu(n)\cdots \mu(1)}p(0).
\end{equation}

To determine $p(0)$ we can use the fact that the numbers $p(n)$ represent probabilities, which means that the sum of the probabilities should be one. Hence, from the requirement
\begin{align*}
1 
&= \sum_{n=0}^\infty p(n) \\
&= p(0) \left(1+\sum_{n=0}^\infty \frac{\lambda(n)\lambda(n-1)\cdots\lambda(0)}{\mu(n+1)\mu(n)\cdots \mu(1)}\right),
\end{align*}
we obtain  
\begin{equation*}
  p(0) = G^{-1},
\end{equation*}
where $G$ is the \recall{normalization constant}
\begin{equation}
  \label{eq:20}
G = 1+\sum_{n=0}^\infty \frac{\lambda(n)\lambda(n-1)\cdots\lambda(0)}{\mu(n+1)\mu(n)\cdots \mu(1)}.
\end{equation}
Now that we know $p(0)$,  $p(n)$ follows from Eq.~\eqref{eq:38}. 

Let us now express a few important performance measures in terms of
$p(n)$: the average number of items $\E L$ in the system and the
fraction of time $\P{L\geq n}$ the system contains at least $n$ jobs.
As $L(s)$ counts the number of jobs in the system at time $s$ (thus
$L(s)$ is an integer),
\begin{equation*}
  L(s) = \sum_{n=0}^\infty n\, \1{L(s) = n}.
\end{equation*}
With this we can write for the time-average number of jobs in the system
\begin{equation}\label{eq:1}
\frac 1 t \int_0^t L(s) \d s = \frac 1 t \int_0^t \left(\sum_{n=0}^{\infty} n\, \1{L(s) = n}\right) \d s
= \sum_{n=0}^{\infty} \frac n t \int_0^t   \1{L(s) = n} \d s,
\end{equation}
where we interchange the integral and the summation\footnote{This is
  allowed as the integrand is non-negative. More generally, the
  interested reader should check Fubini's theorem.}.  It then follows
  from Eq.~(\ref{eq:18}) that
\begin{equation*}
\frac 1 t \int_0^t L(s) \d s =  \sum_{n=0}^{\infty} n\, p(n,t).
\end{equation*}
Finally, assuming that the limit $p(n,t) \to p(n)$ exists as
$t\to\infty$ (and that the summation and limit can be interchanged in
the above), it follows that 
\begin{equation*}
\E L = \lim_{t\to\infty} \frac 1 t \int_0^t L(s) \d s = \sum_{n=0}^\infty n p(n) = \E L.
\end{equation*}
In a loose sense we can say that
$\E L$ is the average number in the system as perceived by the
\emph{server}. (Recall that this is not necessarily the same as what
\emph{arriving} jobs `see').  Similarly, the probability that the
system contains at least $n$ jobs is 
\begin{equation*}
  \P{ L \geq n} = \sum_{i=n}^\infty p(i).
\end{equation*}

From the above we conclude that with the probabilities $p(n)$ we can compute numerous performance measures.
In the next few sections we will make suitable choices for $\lambda(n)$ and $\mu(n)$ to model many different queueing situations so that, based on \eqref{eq:12}, we can analyze the performance of these models.




Finally, the following two exercises show that level-crossing arguments extend well beyond the queueing systems modeled by Figure~\ref{fig:A_n_t}.

\begin{exercise}
  Consider a single server that serves one queue and serves only in
  batches of 2 jobs at a time (so never 1 job or more than 2 jobs),
  i.e., the $M/M^2/1/3$ queue.  Single jobs arrive at rate $\lambda$
  and the inter-arrival times are exponentially distributed so that we can assume that $\lambda(n) = \lambda$. The batch service times are exponentially distributed with mean $1/\mu$. Then, by the memoryless property, $\mu(n) = \mu$. At most 3 jobs fit in the system.   Make a graph of the state-space and show, with arrows, the
  transitions that can occur.

  \begin{solution}
See the figure below.

\begin{tikzpicture}[scale=1,->,>=stealth',shorten >=1pt,auto,node distance=2.8cm,
                    semithick]
\node[state] (0) {$0$}; 
\node[state] (1) [right of=0] {$1$}; 
\node[state] (2) [right of=1] {$2$}; 
\node[state] (3) [right of=2] {$3$}; 

\path 
(0) edge [bend left] node[above] {$\lambda$} (1)
(1) edge [bend left] node[above] {$\lambda$} (2)
(2) edge [bend left] node[above] {$\lambda$} (3)
(3) edge [bend left] node[below] {$\mu$} (1)
(2) edge [bend left] node[below] {$\mu$} (0);

\draw[-, dotted, gray] (4,-2.)--(4,2.0) node[above, black] {level $1$};
\end{tikzpicture}
  \end{solution}
\end{exercise}

\begin{exercise}
  Use the graph of the previous question and a level-crossing argument
  to express the steady-state probabilities $p(n), n=0,\ldots, 3$ in
  terms of $\lambda$ and $\mu$.
  \begin{hint}
First balance the rates across the levels. Then solve for $\pi(k)$.
  \end{hint}
  \begin{solution}
With level crossing
  \begin{align*}
    \lambda p(0)  &= \mu p(2), \quad\text{the level between 0 and 1,}\\
    \lambda p(1)  &= \mu p(2) +\mu p(3), \quad\text{see level 1,}\\
    \lambda p(2)  &= \mu p(3), \quad\text{the level between 2 and 3.}\\
  \end{align*}
  Solving this in terms of $p(0)$ gives $p(2) = \rho p(0)$, $p(3) = \rho p(2) = \rho^2p(0)$, and
  \begin{equation*}
    \lambda p(1) = \mu(p(2) + p(3)) = \mu (\rho + \rho^2) p(0) = (\lambda + \lambda^2/\mu) p(0),
  \end{equation*}
hence $p(1) = p(0)(\mu + \lambda)/\mu$. 
  \end{solution}
\end{exercise}

\paragraph{Interpretation}

The definitions in~(\ref{eq:rates_}) may seem a bit abstract, but they
obtain an immediate interpretation when relating them to
applications. To see this, we discuss two examples.

Consider the sorting process of post parcels at a distribution center
of a post delivery company.  Each day tens of thousands of incoming
parcels have to be sorted to their final destination. In the first
stage of the process, parcels are sorted to a region in the
Netherlands. Incoming parcels are deposited on a conveyor belt. From
the belt they are carried to outlets (chutes), each chute
corresponding to a specific region. Employees take out the parcels
from the chutes and put the parcels in containers.  The arrival rate
of parcels for a certain chute may temporarily exceed the working
capacity of the employees, as such the chute serves as a queue.  When
the chute overflows, parcels are directed to an overflow container and
are sorted the next day. The target of the sorting center is to
deliver at least a certain percentage of the parcels within one
day. Thus, the fraction of parcels rejected at the chute should remain
small.

Suppose a chute can contain at most 20 parcels, say. Then, each parcel
on the belt that `sees' 20 parcels in its chute will be blocked. Let
$L(t)$ be the number of parcels in the chute at time $t$. Then,
$A(20,t)$ as defined in Eq.~(\ref{eq:19}) is the number of\emph{ blocked
  parcels} up to time $t$, and $A(20,t)/A(t)$ is the fraction of
rejected parcels. In fact, $A(20,t)$ and $A(t)$ are continuously
tracked by the sorting center and used to adapt employee capacity to
control the fraction of rejected parcels. Thus, in simulations, if one
wants to estimate loss fractions, $A(n,t)/A(t)$ is the most natural
concept to consider.

For the second example, suppose there is a cost associated with
keeping jobs in queue. Let $w$ be the cost per job in queue per unit
time so that the cost rate is $n w$ when $n$ jobs are in queue. But
then $ w n Y(n,t)$ is the total cost up to time $t$ to have $n$ jobs in
queue, hence the total cost up to time $t$ is
  \begin{equation*}
C(t) =     w \sum_{n=0}^\infty n Y(n,t),
  \end{equation*}
and the average cost is
\begin{equation*}
\frac{C(t)}t =    w \sum_{n=0}^\infty n \frac{Y(n,t)}t = w \sum_{n=0}^\infty n p(n,t).
\end{equation*}
All in all, the concepts developed above have natural interpretations
in practical queueing situations; they are useful in theory and in
simulation, as they relate the theoretical concepts to actual measurements.



\paragraph{Balance equations}

It is important to realize that  that the level-crossing argument cannot
always be used as we do here. The reason is that not always  line exists between two states such that the state space splits  into two disjoint
parts. For a more general approach, we focus on a single state and
count how often this state is entered and left,
cf. Figure~\ref{fig:balance}. Specifically, define
\begin{align*}
  I(n,t) &= A(n-1,t) + D(n,t),
%  O(n,t) &= A(n,t) + D(n-1,t),
\end{align*}
as the number of times the queueing process enters state $n$ either
due to an arrival from state $n-1$ or due to a departure leaving $n$
jobs behind. Similarly,
\begin{align*}
 O(n,t) &= A(n,t) + D(n-1,t),
\end{align*}
counts how often state $n$ is left either by an arrival (to state $n-1$) or a departure (to state $n-1$).

Of course, $|I(n,t)-O(n,t)|\leq 1$. Thus, from the fact that
\begin{equation*}
\lim_{t\to\infty}  \frac{I(n,t)}t = \lim_{t\to\infty} \frac{A(n-1,t)}t + \lim_{t\to\infty} \frac{D(n,t)}t = \lambda(n-1) p(n-1) + 
\mu(n+1) p(n+1)
\end{equation*}
and 
\begin{equation*}
\lim_{t\to\infty}   \frac{O(n,t)}t = \lim_{t\to\infty} \frac{A(n,t)}t + \lim_{t\to\infty} \frac{D(n-1,t)}t = \lambda(n) p(n) + 
\mu(n) p(n)
\end{equation*}
we get that
\begin{equation*}
  \lambda(n-1)p(n-1)+\mu(n+1)p(n+1) = (\lambda(n)+\mu(n))p(n).
\end{equation*}
These equations hold for any $n\geq 0$ and are known as the
\recall{balance equations}.  We will use these equations when studying
queueing systems in which level crossing cannot be used, for instance
for queueing networks.

\begin{figure}[t]
  \centering
\begin{tikzpicture}[->,>=stealth',shorten >=1pt,auto,node distance=1.8cm,
                    semithick]
  \node[state] (0) {$p(0)$} ;
  \node[state] (1) [right of=0] {$p(1)$};
  \node[state] (2) [right of=1] {$p(2)$};
  \node[state] (3) [right of=2] {$p(3)$};
  \node[state] (4) [right of=3] {$\cdots$};

\draw[dashed] (2.6,-1.2) rectangle (4.5,1.2);

\path 
 (0) edge [bend left] node {$\lambda(0)$} (1)
 (1) edge [bend left] node {$\mu(1)$} (0)
 (1) edge [bend left] node[fill=white] {$\lambda(1)$} (2)
 (2) edge [bend left] node[fill=white] {$\mu(2)$} (1)
 (2) edge [bend left] node[fill=white] {$\lambda(2)$} (3)
 (3) edge [bend left] node[fill=white] {$\mu(3)$} (2)
 (3) edge [bend left] node[above] {$\lambda(3)$} (4)
 (4) edge [bend left] node[below] {$\mu(4)$} (3);


\end{tikzpicture}
\caption{ For the balance equations we count how often a box around a
  state is crossed from inside and outside. On the long run the
  entering and leaving rates should be equal. For the example here,
  the rate out is $p(2)\lambda(2) + p(2) \mu(2)$ while the rate in is
  $p(1)\lambda(1)+p(3)\mu(3)$.}
\label{fig:balance}
\end{figure}


Again, just by using properties, i.e., counting differences, that hold
along any sensible sample path we obtain very useful statistical and
probabilistic results.




\begin{figure}[p]
  \centering

  \begin{tikzpicture}[node distance = 2.5cm]

\tikzset{
    %Define standard arrow tip
    >=stealth',
    %Define style for boxes
    % Define arrow style
    pil/.style={
           ->,
           thick,
           shorten <=2pt,
           shorten >=2pt,}
}
\tikzstyle{block} = [rectangle, draw,text centered, rounded corners, minimum height=3em]

    % nodes
    \node [block, text width=5.5cm, align=center] (level) {LEVEL CROSSING: Counting up- and downcrossings};
\node[block, below=1cm of level] (An) {$|A(n,t)-D(n,t)|\leq 1$}
edge[pil,<-] (level); 

\node[block, left=1.5cm of An] (A) {$A(t)-D(t)=L(t)$} 
edge[pil,<-] (level); 

\node[block, right=1.5cm of An] (Anm) {$|A(m,n,t)-D(n,t)|\leq 1$}
edge[pil,<-] (level); 

\node[block, below=1cm of A] (At) {$\frac{A(t)}t \approx \frac{D(t)}t$ if $\frac{L(t)}t \to 0$} 
edge[pil,<-] (A); 

\node[block, below=1.5cm of At] (lambda) {$\lambda=\delta$} 
edge[pil,<-] node[fill=white] {$t\to\infty$} (At); 

\node[block, below=1cm of An] (AnDn) {$\frac{A(n,t)}t\approx\frac{D(n,t)}t$}
edge[pil,<-] (An); 

\node[block, below=1.5cm of AnDn] (AnDn2) {$\frac{A(n,t)}{Y(n,t)}\frac{Y(n,t)}t\approx\frac{D(n,t)}{Y(n+1)}\frac{Y(n+1)}t$}
edge[pil,<-] (AnDn); 

\node[block, below=1.5cm of AnDn2, text width=4cm] (lp) {Recursion: \\
$\lambda(n)p(n) = \mu(n+1)p(n+1)$}
edge[pil,<-] node[fill=white] {$t\to\infty$} (AnDn2); 

\node[block, below=1cm of lp, text width=3cm] (poisson) {Poisson: \\
$\lambda=\lambda(n)$, \\
$\mu=\mu(n)$}
edge[pil,<-] (lp);

\node[block, below=1cm of poisson, text width=4cm] (mm1) {$M/M/1$, $M/M/c$, $M/M/c/k$, \ldots} edge[pil,<-] (poisson);
;

\node[block, right=0.6cm of lp, text width=5cm, align=center] (batch) {Recursion: \\ $\lambda\sum_{m=0}^nG(n-m)p(m) = \mu(n+1)p(n+1)$}
edge[pil,<-] node[fill=white] {$t\to\infty$} (Anm); 

\node[block, right=2.3cm of mm1] (batch2) {$M^X/M/1$}
edge[pil,<-]  (poisson)
edge[pil,<-]  (batch); 


\node[block, below=1cm of mm1, text width=4.5cm] (perf) {Performance
  measures:
$\E L= \sum_{n=0}^\infty n p(n)$, $\P{L\geq m}$, \ldots} 
edge[pil,<-] (mm1)
edge[pil,<-] (batch2);

\node[block, below=1.5cm of lambda] (pasta1) {$\frac{A(t)}t\frac{A(n,t)}{A(t)} = \frac{A(n,t)}{Y(n,t)}\frac{Y(n,t)}t$} 
edge[pil,<-] (AnDn2)
edge[pil,<-,bend left=20] (At.south west)
;

\node[block, below=1.5cm of pasta1] (pasta2) {$\lambda \pi(n) = \lambda(n)p(n)$} 
edge[pil,<-] node[fill=white] {$t\to\infty$} (pasta1);

\node[block, below=1.5cm of pasta2, text width=3cm] (pasta3) {PASTA: $\pi(n) = p(n)$} 
edge[pil,<-] (poisson)
edge[pil,<-] (pasta2)
edge[pil,->] (perf);

\end{tikzpicture}
  \caption{With level-crossing arguments we can derive a number of
    useful relations. This figure presents an overview of these
    relations that we derive in this and the next sections.}
\label{fig:summaries}
\end{figure}

\Closesolutionfile{hint}
\Closesolutionfile{ans}

\opt{solutionfiles}{
\subsection*{Hints}
\input{hint}
\subsection*{Solutions}
\input{ans}
}


%\clearpage


%%% Local Variables:
%%% mode: latex
%%% TeX-master: "../queueing_book"
%%% End:
