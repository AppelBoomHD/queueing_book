\section{Rate Stability and Utilization}
\label{sec:rate-stability}

\subsection*{Theory and Exercises}

\Opensolutionfile{hint}
\Opensolutionfile{ans}

In the analysis of any queueing process the first step should be to check the relations between the arrival, service and departure rates.
The concept of rate is crucial because it captures our intuition that when, on the long run, jobs arrive faster than they can leave, the system must `explode'.
Thus, the first performance measures we need to estimate when analyzing a queueing system are the arrival and departure rate, and then we need to check that the arrival rate is smaller than the departure rate.

We first formalize the \emph{arrival rate} and \emph{departure rate} in terms of the \emph{counting processes} $\{A(t)\}$ and $\{D(t)\}$.
The \recall{arrival rate} is the long-run average number of jobs that arrive per unit time, i.e.,
\begin{equation}
  \label{eq:3}
  \lambda = \lim_{t\to\infty} \frac{A(t)}t.
\end{equation}
We remark in passing that this limit does not necessarily exist if $A(t)$ is some pathological function.
If, however, the inter-arrival times $\{X_k\}$ are the basic data, and $\{X_k\}$ are \recall{independent and identically distributed (i.i.d.)}
and distributed as a generic random variable $X$ with finite mean $\E{X}$, we can construct $\{A_k\}$ and $\{A(t)\}$ as described in Section~\ref{sec:constr-gg1-queu}; the strong law of large numbers guarantees that the above limit exists.

\begin{exercise}[\faFlask]
  Can you make an arrival process such that $A(t)/t$ does not have a
  limit?  
  \begin{hint}
As a start, the function $\sin(t)$ does not have a limit as
    $t\to\infty$. However, the time-average $\sin(t)/t \to 0$.  Now
    you need to make some function whose time-average does not
    converge, hence it should grow fast, or fluctuate wilder and
    wilder.
  \end{hint}
  \begin{solution}
 If $A(t) = 3 t^2$, then clearly $A(t)/t = 3t$. This does not
    converge to a limit. 

  Another example, let the arrival rate $\lambda(t)$ be given as
    follows:
    \begin{equation*}
      \lambda(t) = 
    \begin{cases}
      1 & \text{if } 2^{2k} \leq t < 2^{2k+1} \\
      0 & \text{if } 2^{2k+1} \leq t < 2^{2(k+1)},
    \end{cases}
    \end{equation*}
    for $k=0,1,2,\ldots$. Let $A(t) = \lambda(t) t$. Then $A(t)/t$
    does not have limit. Of course, these examples are quite
    pathological, and are not representable for `real life cases'.
    (Although this is also quite vague. What, then, is a real life
    case?)

For the mathematically interested, we seek a
    function for which its Ces\`aro limit does not exist.
  \end{solution}
\end{exercise}



Observe that at time $t=A_n$, precisely $n$ arrivals occurred. Thus,
by applying the definition of $A(t)$ at the epochs $A_n$, we see that
$A(A_n) = n$. Thus,
\begin{equation*}
  \frac{1}n\sum_{k=1}^n X_k = \frac{A_n}n = \frac{A_n}{A(A_n)}. 
\end{equation*}
But since $A_n\to\infty$ if $n\to\infty$, it follows from
Eq.~(\ref{eq:3}) that the average inter-arrival time between two
consecutive jobs is
\begin{equation}\label{eq:54}
  \E X = \lim_{n\to\infty}  \frac{1}n\sum_{k=1}^n X_k = \lim_{n\to \infty} \frac{A_n}{A(A_n)} = \lim_{t\to\infty} \frac t{A(t)} = \frac 1 \lambda,
\end{equation}
where we take $t=A_n$ in the limit for $t\to\infty$.  In words, the
above states that the arrival rate $\lambda$ is the inverse of the
expected inter-arrival time.

\begin{exercise}[\faRocket]
   In  Eq. (\ref{eq:54}) we replaced the limit with respect to $n$ by a
    limit with respect to $t$.  Use
    the notation $A_{A(t)}$ to show that  this is allowed.
  Show next that the function $t\to A(t)$ as defined by Eqs.~(\ref{eq:2})
  is right-continuous. 
    \begin{hint}
 Use that $A_{A(t)} \leq t < A_{A(t)+1}$. Divide by $A(t)$
      and take suitable limits. BTW, such type of proof is used quite
      often to show that the existence of one limit implies, and is
      implied by, the existence of another type of limit.  
    \end{hint}
 \begin{solution}
This exercise is meant to provide some insight into what needs
to be done to put everything on solid ground.


 Observing that $A_{A(t)}$ is the arrival time of the last job
    before time $t$ and that $A_{A(t)+1}$ is the arrival time of the
    first job after time $t$: 
  \begin{equation*}
    A_{A(t)}  \leq t  < A_{A(t)+1} \Leftrightarrow 
    \frac{A_{A(t)}} {A(t)}  \leq \frac{t}{A(t)}  <\frac{A_{A(t)+1}}{A(t)} = \frac{A_{A(t)+1}}{A(t)+1}\frac{A(t)+1}{A(t)}.
  \end{equation*}
  Now $A(t)$ is a counting process such that $A(t)\to\infty$ as
  $t\to\infty$. Therefore, $\lim_t A_{A(t)}/A(t) = \lim_n
  A_n/n$.
  Moreover, it is evident that
  $\lim_t A_{A(t)+1}/(A(t)+1) = \lim_t A_{A(t)}/A(t)$, and that
  $(A(t)+1)/A(t)\to 1$ as $t\to\infty$. Thus it follows from the above
  inequalities that $\lim_n A_n/n = \lim_t t/A(t)$.
     



  For the right-continuity of $A(t)$, define $f(t) = \1{A_1 \leq t}$.
  Observe first that $f(t)$ is increasing, and $f(t)\in\{0,1\}$.
  Thus, if $f(t)=1$ then $f(u)=1$ for all $u\geq t$, and if $f(t)=0$ then $f(u) = 0$ for all $u\leq t$.

    You may skip the rest of the prove below, but the above is
    essential to memorize; make a plot of $f(t)$, in particular the
    behavior around $A_1$ is important.

    We need to prove, for right-continuity, that $f(u)\to f(t) $ as
    $u\downarrow t$. When $f(t)=1$, $f(u)=1$ for any $u>1$, by the
    definition of $f(x)$. When $f(t)=0$ we have to do a bit more
    work. Formally, we have to prove that, for fixed $t$ and for all
    $\epsilon > 0$, there is a $\delta>0$ such that
    $u\in(t, t+\delta) \Rightarrow |f(u) -f(t)| < \epsilon$. (Note the
    differences with the regular definition of continuity.) Since, by
    assumption, $t$ is such that $f(t)=0$, and $f\in\{0,1\}$ we need
    to show that $f(u)=0$ for $u\in(t, t+\delta)$. Now, clearly, 
    $f(t)=0$ only if $t < A_1$.  But, then for any $u\in(t, A_1)$, we
    have that $f(u) = 0$. Thus, taking $\delta = A_1 - t$ suffices.

    The next step is to observe that $A(t)$ is a sum of
    right-continuous functions whose steps do not overlap since by
    assumption $0<A_1 < A_2 < \cdots$. As $A$ is (almost surely)
    a finite sum of bounded, increasing and right-continuous functions,
    it is also right-continuous.

    If you like, you can try to prove this last step too. 

  Hopefully this problem, and its solution, clarifies that even such
  small details require attention. If we want to make some progress
  with respect to developing some queueing theory, we have to skip
  most of the proofs and mathematical problems; we simply don't have
  enough time in this course  to be concerned with all theorems
  and proofs.

    % For finite sums it is simple. Suppose that $f$ and $g$ are
    % right-continuous, then
    % \begin{equation*}
    %   |f(u) + g(u) - f(t) - g(t)| \leq |f(u)-f(t)|+|g(u)-g(t)|.
    % \end{equation*}
    % Since both terms at the right-hand side can be made arbitrarily
    % small, the left-hand side can also be made as small as we
    % like. With this we can see that the function
    % $F_N(t) = \sum_{n=1}^N f_n(x)$ is right-continuous if all $f_n$
    % are right-continuous, and also that $F_{N+1} = F_N + f_{N+1}$ is
    % right-continuous. As this applies for all $N$, it follows from
    % induction that $\lim_N F_N$ is right-continuous, provided this
    % limit exists. When $f_n$ are all increasing, which is the case for
    % our situation by taking $f_n(t) = 1\{A_n \leq t\}$, then this
    % limit certainly exists.
 \end{solution}
\end{exercise}

The development of the departure times $\{D_k\}$ is entirely analogous
to that of the arrival times; we leave it to the reader to provide the
details. As a result we can define the \recall{departure rate} as
\begin{equation}\label{eq:28}
  \delta = \lim_{t\to\infty} \frac{D(t)}t.
\end{equation}


\begin{exercise}[\faFlask]
  Define the departure time $D_{k}$ of the $k$th job in terms of
  $\{D(t)\}$. 
  \begin{hint}
Use the analogy with Eq.~\eqref{eq:27}.
  \end{hint}
\begin{solution}
  \begin{equation*}
 D_{k} = \inf\{t: D(t) \geq k\}.
  \end{equation*}
\end{solution}
\end{exercise}

Assume now that there is a single server. Let $S_k$ be the required
service time of the $k$th job to be served, and define
\begin{equation*}
U_n = \sum_{k=1}^n S_k
\end{equation*}
as the total service time required by the first $n$ jobs. With this,
let 
\begin{equation*}
  U(t) = \max\{n: U_n \leq t\}
\end{equation*}
and  define the \recall{service rate} or \recall{processing rate} as
\begin{equation*}
  \mu = \lim_{t\to\infty} \frac{U(t)}t.
\end{equation*}
In the same way as we derived that $\E X= 1/\lambda$, we obtain for the expected (or average service time required by an individual job
\begin{equation*}
  \E S = \lim_{n\to\infty} \frac 1 n \sum_{k=1}^n S_k = \lim_{n\to\infty} \frac{U_n}{n} = \lim_{n\to\infty} \frac{U_n}{U(U_n)} = \lim_{t\to\infty} \frac t{U(t)} = \frac 1 \mu.
\end{equation*}

Now observe that, if the system is empty at time $0$, it must be that
at any time the number of departures must be smaller than or equal to the number
of arrivals, i.e., $D(t) \leq A(t)$ for all $t$. Therefore,
\begin{equation}\label{eq:26}
\delta =   \lim_t \frac{D(t)}t \leq \lim_t \frac{A(t)}t = \lambda.
\end{equation}
We call a system \recall{rate stable} if
\begin{equation*}
  \lambda = \delta,
\end{equation*}
in other words, the system is stable if, on the long run, jobs leave
the system just as fast as they arrive. Of course, if
$\lambda > \delta$, the system length process $L(t) \to \infty$ as
$t\to \infty$.

It is also evident that jobs cannot depart faster than they can be
served, hence, $D(t) \leq U(t)$ for all~$t$. Combining this with the
fact that $\delta \leq \lambda$, we get
\begin{equation*}
  \delta \leq \min\{\lambda, \mu\}.
\end{equation*}
When $\mu \geq \lambda$ the above inequality reduces to $\delta = \lambda$ for rate-stable systems.
(It is interesting to prove this.)
As it turns out, when $\mu = \lambda$ and the variance of the service time $\V{S} > 0$ or $\V{X} >0$ the queue length process behaves in a very peculiar way.
For this reason we henceforth (and implicitly) require that $\mu > \lambda$.



\begin{exercise} [\faFlask]
Define the random variables $\{\tilde X_k,k=1,\ldots\}$ as $\tilde X_k = S_{k-1}-X_k$.
  For stability of the queueing process it is essential that
  $\tilde X_k$ has negative expectation, i.e.,
  $\E{\tilde X_k} = \E{S_{k-1}-X_k} < 0$.  What is the conceptual
  meaning of this inequality?
  \begin{solution}
 That the average time customers spend in service is smaller
      than the average time between the arrival of two subsequent
      jobs. 
  \end{solution}
\end{exercise}

\begin{exercise} [\faFlask]
Define $\tilde X_k = S_{k-1}-X_k$.
 Show that $\E{\tilde X_k} <0$ implies that $\lambda<\mu$. 
 \begin{hint}
Remember that $\{X_k\}$ and $\{S_k\}$ are sequences of i.i.d. random variables. What are the implications for the expectations?
 \end{hint}
  \begin{solution}
  $0> \E{\tilde X_k} = \E {S_{k-1}-X_k} =  \E{ S_{k-1}}- \E {X_k} = \E S - \E X$, where we use the fact that the $\{S_k\}$ and $\{X_k\}$ are i.i.d. sequences. Hence, 
  \begin{equation*}
    \E X > \E S \iff \frac 1{\E S} > \frac1{\E X} \iff \mu > \lambda.
  \end{equation*}

  \end{solution}
\end{exercise}


\begin{exercise}[\faPhoto]
  Consider a paint factory which contains a paint mixing machine that
  serves two classes of jobs, A and B. The processing times of jobs of
  types A and B are constant and require $t_A$ and $t_B$ hours. The
  job arrival rate is $\lambda_A$ for type A and $\lambda_B$ for type
  $B$ jobs. It takes a setup time of $S$ hours to clean the mixing
  station when changing from paint type A to type B, and there is no
  time required to change from type B to A.

  To keep the system (rate) stable, it is necessary to produce the
  jobs in batches, for otherwise the server, i.e., the mixing machine,
  spends a too large fraction of time on setups, so that
  $\mu < \lambda$. Thus, it is necessary to identify minimal batch
  sizes to ensure that $\mu > \lambda$.  Motivate that the following linear
  program  can be used to determine the minimal batch sizes:
\begin{equation*}
  \text{minimize }  T
\end{equation*}
such that $ T=  k_A t_A + S + k_B t_B$, $\lambda_A T < k_A$ and $\lambda_B T < k_B$.
\begin{hint}
Here are some questions to help you interpret this formulation.
\begin{enumerate}
\item   What are the decision variables for this problem? In other words, what are the `things' we can control/change?
\item What are the interpretations of $k_A t_A$, and $S+k_B t_B$?
\item What is the meaning of the first constraint?  Realize that $T$
  represents one production cycle. After the completion of one such
  cycle, we start another cycle. Hence, the start of every cycle can
  be seen as a restart of the entire system.
\item   What is the meaning of the other two constraints?
\item Why do we minimize the cycle time $T$?
\item Solve for $k_A$ and $k_B$ in terms of $S$,  $\lambda_A, \lambda_B$ and $t_A, t_B$. 
\item Generalize this to $m$ job classes and such that the cleaning
  time between jobs of class $i$ and $j$ is given by $S_{i j}$. (Thus,
  the setup times are sequence dependent.) 
\end{enumerate}
\end{hint}

  \begin{solution}
    Realize that the machine works in cycles. A cycle starts with
    processing $k_A$ jobs of type A, then does a setup, and processes
    $k_B$ jobs of type B, and then a new cycle starts again.  The time
    it takes to complete one such cycle is $T=k_A t_A + S + k_B t_B$.
    The number of jobs of type A processed during one such cycle is,
    of course, $k_A$. Observe next that the average number of jobs
    that arrive during one cycle is $\lambda_A T$. We of course want
    that $\lambda_A T< k_A$, i.e., less jobs of type A arrive on
    average per cycle than what we can process.
  \end{solution}
\end{exercise}



\Closesolutionfile{hint}
\Closesolutionfile{ans}

\opt{solutionfiles}{
\subsection*{Hints}
\input{hint}
\subsection*{Solutions}
\input{ans}
}
%\clearpage

%%% Local Variables:
%%% mode: latex
%%% TeX-master: "../queueing_book"
%%% End:
