\section{Little's Law}
\label{sec:littles-law}

There is an important relation between the average time $\E W$ a job
spends in the system and the long-run time-average number $\E L$ of jobs
that is contained in the system, which is called \emph{Little's law}:
\begin{equation}\label{eq:53}
 \E L = \lambda \E W.
\end{equation}
Part of the usefulness of Little's law is that it applies under very general conditions to all input-output systems, whether the system is a queueing system or an inventory system or some much more general system.
Hence, we will apply Little's law often in the forthcoming sections.
The aim of this section is to prove this law.

\opt{solutionfiles}{
\subsection*{Theory and Exercises}
\Opensolutionfile{hint}
\Opensolutionfile{ans}
}


%\Cref{ex:62} provides a proof of this under some simple conditions.

We start by defining a few intuitively useful concepts. From~\cref{eq:14}, we see that
\begin{equation*}
\frac 1 t\int_0^t L(s)\, \d s = \frac 1 t\int_0^t (A(s)-D(s)) \, \d s
\end{equation*}
is the time-average of the number of jobs in the system during
$[0,t]$.
% Observe once again from the second equation that
% $\int_0^t L(s)\,\d s$ is the area enclosed between the graphs of $A(s)$
% and $D(s)$.
Next, the waiting time of the $k$th job is the time between the moment the
job arrives and departs, that is, 
\begin{equation*}
 W_k = \int_0^\infty \1{A_k \leq s < D_k}\,\d s.
\end{equation*}
\cref{fig:atltdt}  relates $W_k$ to $L(t)$.

Consider a departure time $T$ at which the system is empty so that $A(T) = D(T)$.
%as at time $T$ all jobs that arrived up to $T$ also have left.
%As for all jobs $k\leq A(T)$ we have that $D_k \leq T$, we can replace the integration bounds in the above expression for $W_k$ by
Then, for $k\leq A(T)$, 
\begin{equation*}
 W_k = \int_0^T \1{A_k \leq s < D_k}\,\d s,
\end{equation*}
and for $s\leq T$,
\begin{equation*}
L(s) = \sum_{k=1}^\infty \1{A_k \leq s < D_k} = \sum_{k=1}^{A(T)}\1{A_k \leq s < D_k}.
\end{equation*}



\begin{extra}\clabel{ex:59}
 Show that 
\begin{equation*}
 \int_0^T L(s)\, \d s = \sum_{k=1}^{A(T)} W_k.
\end{equation*}
\begin{hint}
 Substitute the definition of $L(s)$ in the left-hand side, then reverse the integral and summation.
\end{hint}
\begin{solution}
\begin{equation*}
 \begin{split}
 \int_0^T L(s)\, \d s & = \int_0^T \sum_{k=1}^{A(T)} \1{A_k \leq s < D_k}\, \d s \\
& = \sum_{k=1}^{A(T)}\int_0^T \1{A_k \leq s < D_k}\, \d s = \sum_{k=1}^{A(T)} W_k.
 \end{split}
\end{equation*}
\end{solution}
\end{extra}


\begin{extra}
Observe that the area between the graphs of $A(s)$ and $D(s)$ must
be equal to the total waiting time spent by all jobs in the system
until $T$. Use this to provide a graphical interpretation of the proof of Little's law.
\begin{hint}
 Make a drawing of $A(t)$ and $D(t)$ until time $T$, i.e., the
 first time the system is empty. Observe that $A(t)-D(t)$ is the number of jobs in the system. Take some level $k$, and compute $A_k = A^{-1}(k)$ and $D_k = D^{-1}(k)$. Observe that $D_k - A_k = D^{-1}(k) - A^{-1}(k)$ is the waiting time of job $k$.
\end{hint}
\begin{solution}
  The area enclosed between the graphs of $A(t)$ and $D(t)$ until $T$ can be `chopped up' in two ways: in the horizontal and in the vertical direction.
  (Please make the drawing as you go along\ldots) A horizontal line between $A(t)$ and $D(t)$ corresponds to the waiting time of a job, while a vertical line corresponds to the number of jobs in the system at time $t$.
  Now adding all horizontal lines (by integrating along the $y$-axis) makes up the total amount of waiting done by all the jobs until time $T$.
  On the other hand, adding the vertical lines (by integrating along the $x$-axis) is equal to the summation of all jobs in the system.
  Since the area is the same no matter whether you sum it in the horizontal or vertical direction:
 \begin{equation*}
 \sum_{k=1}^{A(T)} W_k = \text{enclosed area} = \int_0^T (A(t)-D(t))\,dt. 
 \end{equation*}
 Dividing both sides by $A(T)$ gives
 \begin{equation*}
\frac{1}{A(T)} \sum_{k=1}^{A(T)} W_k =\frac{1}{A(T)} \int_0^T (A(t)-D(t))\,dt. 
 \end{equation*}

 Finally, observe that this equality holds between any two times
 $T_i, T_{i+1}$, where times $\{T_i\}$ are such that
 $A(T_i)=D(T_i)$. Then, as $T_i\to \infty$, which we assumed from
 the on-set, $\frac{1}{A(T_i)} \sum_{k=1}^{A(T_i)} W_k\to \E W$,
 and
 \begin{equation*}
\frac{T_i}{A(T_i)}\frac{1}{T_i} \int_0^{T_i} (A(t)-D(t))\,dt \to \lambda^{-1} \E L.
 \end{equation*}
Hence, Little's law follows.
\end{solution}
\end{extra}

\begin{exercise}\clabel{ex:62}
 Prove Little's law under the assumptions that $A(T_i) = D(T_i)$ for an infinite number of times $\{T_i\}$ such $T_i\to\infty$ and that all limits exist. 
\begin{hint}
In  the result of~\cref{ex:59},  divide both sides by $T$. At the right-hand side use that $1/T = A(T)/T \cdot 1/A(T)$. Take limits.
\end{hint}
\begin{solution}
  First solve~\cref{ex:59}. Then,
\begin{equation*}
 \frac 1 T \int_0^T L(s)\, \d s = \frac{A(T)} T \frac{1}{A(T)} \sum_{k=1}^{A(T)} W_k.
\end{equation*}
Assuming there are an infinite number of times
$0\leq T_i<T_{i+1}<\cdots$, $T_i\to\infty$, at which $A(T_i) = D(T_i)$
and the following limits exist
\begin{align*}
\frac 1 T \int_0^T L(s)\, \d s &\to \E L,&
\frac{A(T_i)}{T_i} &\to \lambda, &
\frac{1}{A(T_i)} \sum_{k=1}^{A(T_i)} W_k &\to \E W,
\end{align*}
we obtain Little's law.

\end{solution}
\end{exercise}


\begin{extra}\clabel{ex:42}
  Use the (physical) dimensions of the components of Little's law to check that $\E{W} \neq \lambda \E{L}$.
  (With this check, you can prevent making an often-made mistake.)
\begin{hint}
Checking the dimensions in the formula prevents painful mistakes.
\end{hint}
\begin{solution}
 Sometimes (often?) students memorize Little's law in the wrong
 way. Thus, as an easy check, use the dimensions of the concepts:
 $\E L$ is an average \emph{number}, $\lambda$ is a \emph{rate},
 i.e., \emph{numbers per unit time}, and $\E W$ is waiting
 \emph{time}. 
\end{solution}
\end{extra}



\begin{extra}\clabel{ex:37}
 Consider the server of the $G/G/1$ queue as a system by itself.
 The time jobs stay in this system is $\E S$, and jobs arrive at rate $\lambda$.
 Use  Little's law to conclude that  $\lambda \E S = \rho := \lim_{t\to\infty} t^{-1}\int_0^t L_S(s)\d s$.
\begin{solution}
 The arrival rate at the server must be $\lambda$ and the time a job remains at the server is $\E S$.
 The fraction of time the server is busy is precisely the fraction of time there is a job present at the server.
 Thus, applying Little's law to the server itself, we see that $\rho = \E{L_S} = \lambda \E S$.
\end{solution}
\end{extra}


\begin{extra}\clabel{ex:43}
 For a given single-server queueing system the average number of customers in the system is $\E L = 10$, customers arrive at rate $\lambda=5$ per hour and are served at rate $\mu=6$ per hour.
 What is the average time customers spend in the system?
\begin{hint}
Start with checking the units when applying Little's law.
\end{hint}
\begin{solution}
 \begin{equation*}
 \E W = \E L/\lambda = 10/\lambda = 10/5 = 2.
 \end{equation*}
\end{solution}
\end{extra}

\begin{exercise}\clabel{ex:44}
 For a given single-server queueing system the average number of customers in the system is $\E L = 10$, customers arrive at rate $\lambda=5$ per hour and are served at rate $\mu=6$ per hour.
 Suppose that at the moment you join the system, the number of customers in the system is 10.
 What is your expected time in the system?
\begin{solution}
If you arrive at a queueing system, you first have to wait until the job in service is finished. Then you need to wait until the 9 jobs in queue are finished. This takes, in expectation, $9/\mu$. (Recall, 1 job is in service at the moment you arrive, so 9 are in queue.) Assuming that service times are exponential, so that, by the memoryless property, the remaining service time of the job in service is still $\E S$ when you arrive, you spend $10/\mu + 1/\mu = 11/6 \neq 2$. (To account for the last $+1/\mu$, observe that yourself also have to be served to compute the time you spend in the system.)


Now in this question, it is \emph{given} that the system
 length is 10 at the moment of arrival. However, $L$ as `seen' upon arrival by this
 given customer is in general not the same as the time-average $\E{L}$.

Thus, Little's law need not hold at all moments in time; it is a statement about \emph{averages}.
\end{solution}

\end{exercise}

With the PASTA property and Little's law it becomes quite easy to derive expressions for the average queue length and waiting times for the $M/M/1$ queue.
The average waiting time $\E W$ in the entire system is the expected time in queue plus the expected time in service, i.e.,
\begin{equation}\label{eq:78}
 \E W = \E{W_Q}+ \E S.
\end{equation}
By the PASTA property we have for the $M/M/1$ queue that
\begin{equation}\label{eq:wqes}
 \E{W_Q} = \E L \E S.
\end{equation}


\begin{exercise}\clabel{ex:l-215}
Use Little's law to show for the $M/M/1$ queue that 
 \begin{align*}
 \E W &= \frac{\E S}{1-\rho}, & \E L &= \frac\rho{1-\rho}, \\
 \E{L_Q} &= \frac{\rho^2}{1-\rho}, & \E{L_s} &= \rho.
 \end{align*}
\begin{hint}
 Combine~\cref{eq:78,eq:wqes} and apply Little's law. 
\end{hint}
\begin{solution}
\begin{align*}
 \E W &= \E L \E S + \E S = \lambda \E W \E S + \E S= \rho \E W + \E S, \\
 \E L &= \lambda \E W = \frac{\lambda \E S}{1-\rho} = \frac\rho{1-\rho}, \\
 \E{W_Q} &= \E W - \E S = \frac{\E S}{1-\rho} - \E S = \frac{\rho}{1-\rho} \E S,\\
 \E{L_Q} &= \lambda \E{W_Q} = \frac{\rho^2}{1-\rho}, \\
 \E{L_s} &= \E L - \E{L_Q} = \frac{\rho}{1-\rho} - \frac{\rho^2}{1-\rho} = \rho, 
\end{align*}
\end{solution}
\end{exercise}

\begin{exercise}\clabel{ex:l-216}
Why is~\cref{eq:wqes} \emph{not} true in general for the $M/G/1$ queue? 
\begin{solution}
 By the memoryless property of the (exponential) distributed service times of the $M/M/1$ queue, the duration of a job in service, if any, is $\Exp(\mu)$ also at an arrival moment.
 Therefore, at an arrival moment, all jobs in the system (whether in service or not) have the same expected duration.
 Hence, the expected time to spend in queue is the expected number of jobs in the system times the expected service time of each job, i.e., $\E{W_q} = \E L \E S$.
 Note that we use PASTA to see that the expected number of jobs in the system at an arrival is $\E{L}$.
 For the $M/G/1$ queue, the job in service (if any) does not have the same distribution as a job in queue.
 Hence, the expected time in queue is not $\E L \E S$.
\end{solution}
\end{exercise}




In the above proof of Little's law we assumed that there is a sequence of moments $\{T_k, k=0,1,\ldots\}$ at which the system is empty and such that $T_k < T_{k+1}$ and $T_k \to \infty$.
However, in many practical queueing situations the system is never empty.
Thus, to be able to apply Little's law to such more general situations we should slacken the assumption that such a sequence exists.
The aim of this set of questions is to find an educated guess for a more general assumption under which Little's law can hold.


\begin{exercise}\clabel{ex:l-220}
 Motivate in words (or with a derivation, if you prefer this) why the following is true:
 \begin{equation*}
 \sum_{k=1}^{A(t)} W_k \geq \int_0^t L(s) \d s \geq \sum_{k=1}^{D(t)} W_k.
 \end{equation*}
\begin{solution}
 Intuitively, the left term is all the work that arrived up to time $t$, the middle term is all the work that has been processed, and the right term all the work that left.
 Any job that that is half way its service counts for full at the left, for half in the middle expression, and not in the right.

 More formally, for any job $k$ and time $t$, we have $W_k \1{A_k \leq t} \geq \int_0^t \1{A_k \leq s < D_k } \d s \geq W_k \1{D_k \leq t}$. (To see this, fix $k$, and check the three cases $t < A_k, A_k \leq t < D_k, D_k < t$.) Then,
 \begin{equation*}
 \sum_{k=1}^\infty W_k \1{A_k \leq t} \geq \int_0^t \sum_{k=1}^\infty \1{A_k \leq t < D_k} \d s \geq \sum_{k=1}^\infty W_k \1{D_k \leq t}. 
 \end{equation*}
 Finally, note that $ \sum_{k=1}^\infty W_k \1{A_k \leq t} = \sum_{k=1}^{A(t)} W_k$ and $ \sum_{k=1}^\infty W_k \1{D_k \leq t} = \sum_{k=1}^{D(t)} W_k$, and use the definition of $L(s)$.
\end{solution}
\end{exercise}

\begin{exercise}\clabel{ex:l-221}
Take suitable limits (and assume all these limits exist) to show that 
 \begin{equation*}
\lambda \E W \geq \E L \geq \delta \E W.
 \end{equation*}
 Make explicit all the points where you use the strong law of large numbers.
\begin{solution}
 \begin{equation*}
 \lim_{t\to\infty} \frac{A(t)}{t}\frac 1{A(t)}\sum_{k=1}^{A(t)} W_k \geq \lim_{t\to \infty} \frac 1 t \int_0^t L(s) \d s \geq \lim_{t\to\infty} \frac{D(t)}{t}\frac 1{D(t)} \sum_{k=1}^{D(t)} W_k. 
 \end{equation*}
 We use the strong law of large numbers to conclude that limits converges to $n^{-1} \sum_{k=1}^n W_k \to \E W$, and we assume that $\{W_k, k\geq N\}$ forms a sequence of i.i.d.
 random variables for $N$ sufficiently large.
\end{solution}
\end{exercise}


\begin{exercise}\clabel{ex:l-222}
 Suppose that $A(t) = \lambda t$ and $D(t)= [A(t) - 10]^+$.
 Explain that for this system the above assumption on $\{T_k\}$ is violated. Show that Little's law is still true.
\begin{solution}
 When $t>10$, $L(t) = 10$.
 Hence the system is never empty.
 Since still $\delta=\lim_{t\to\infty} D(t)/t = \lambda$, and $\lim_{n\to \infty} n^{-1}\sum_{k=1}^n W_k$ exists, Little's law follows.
\end{solution}
\end{exercise}

\begin{exercise}\clabel{ex:l-223}
 Based on the above formulate an educated guess for more general conditions under which Little's law holds.
 (You don't have to prove Little's law under your condition; postpone that to after the exam.)
\begin{solution}
We seem to need that $\lambda = \delta$ and that $\lim_{n\to \infty} n^{-1}\sum_{k=1}^n W_k$ exists. 
\end{solution}
\end{exercise}




\opt{solutionfiles}{
\Closesolutionfile{hint}
\Closesolutionfile{ans}
\subsection*{Hints}
\input{hint}
\subsection*{Solutions}
\input{ans}
}

%\clearpage



%%% Local Variables:
%%% mode: latex
%%% TeX-master: t
%%% End:
