\section{(Limits of) Empirical Performance Measures}
\label{sec:limits-of-emperical}


\subsection*{Theory and Exercises}

\Opensolutionfile{hint}
\Opensolutionfile{ans}

If the arrival and service processes are such that the queueing system
is rate-stable, we can sensibly define other performance measures such
as the average waiting time. In this section we define the second most
important performance measures; the most important being the
utilization $\rho$. We provide an overview of the relations between
these performance measures in Figure~\ref{fig:constructiongg1}.


With the construction of queueing processes in
Section~\ref{sec:constr-gg1-queu} we can compute the waiting time as
observed by the first $n$, say, jobs. Thus, the average waiting time
of the first $n$ arrivals is given by $n^{-1}\sum_{k=1}^n W_k$. We
therefore define the \recall{expected waiting time} as
\begin{equation}\label{eq:49}
  \E W = \lim_{n\to\infty} \frac 1n\sum_{k=1}^n W_k,
\end{equation}
and the expected time in queue as
\begin{equation}\label{eq:50}
  \E{W_Q} = \lim_{n\to\infty} \frac 1 n\sum_{k=1}^n W_{Q,k}.
\end{equation}
Note that these performance measures are limits of \emph{empirical}
measures.  Note also that these statistics are as \emph{observed by
  arriving jobs}: the first job has a waiting time $W_1$ at its
arrival epoch, the second a waiting time $W_2$, and so on. For this
reason we colloquially say that $\E W$ is the average waiting time as
`seen by arrivals'.  The \emph{distribution of the waiting times at
  arrival times} can be found by counting:
\begin{equation}\label{eq:48}
  \P{W \leq x}  = \lim_{n\to\infty} \frac 1n\sum_{k=1}^n \1{W_k\leq x}.
\end{equation}
Finally, the (sample) \recall{average number of jobs} in the system as seen by
arrivals is given by
\begin{equation}\label{eq:EQ}
\E L =  \lim_{n\to\infty}\frac 1 n \sum_{k=1}^n L(A_k-),
\end{equation}
where $L(A_k-)$ is the number of jobs in the system at the arrival epoch
of the $k$th job.  The \recall{distribution of $\{L(t)\}$ as seen by
  customers upon arrival}, is
\begin{equation}\label{eq:Qm}
\P{L\leq m} = \lim_{n\to\infty} \frac 1 n \sum_{k=1}^n \1{L(A_k-) \leq m}.
\end{equation}


\begin{exercise}
 Assume that $X_k = 10$ minutes and $S_k = 11$ minutes for all
    $k$, i.e., $X_k$ and $S_k$ are deterministic and constant. What
    are $\lambda$ and $\mu$?  Compute $A_k$, $W_k$ and $D_k$. 
    How do these numbers behave as a function of $k$?
  \begin{solution}
 $\lambda=6$ per hour, and $\mu=60/11$ per hour. Note that
      $\mu < \lambda$. $A_0 = 0$, $A_1=10$, $A_2=20$, etc., hence
      $A_k = 10k$. $W_{Q,0} = 0$, $W_{Q,1} = \max\{0 + 0-10,0\} = 0$.
      $W_{Q,2} = \max\{0+11-10,0\} =1$.
      $W_{Q,3} = \max\{1+11-10,0\} =2$. Hence, $W_{Q,k} = k-1$ for
      $k\geq1$. Thus, $W_k = k-1+11 = k + 10$ for $k\geq1$, and
      $D_k = 10k + k+10 = 11k+10$. Note that $W_k$ increases linearly
      as a function of $k$.  You can infer that, for the queue length
      to remain bounded, it is necessary that the service rate
      $\mu > \lambda$. 
  \end{solution}
\end{exercise}


\begin{exercise}
 Yet another simple case is to take $X_k=10$ minutes and
    $S_k=9$ minutes for all $k$. Answer the same questions as in the
    previous exercise.
  \begin{solution}
 Trivial.
  \end{solution}
\end{exercise}


A related set of performance measures follows by tracking the system's
behavior over time and taking the \emph{time-average}, rather than the
average at sampling (observation) moments. Thus, if we simulate the
queueing system up to time $t$, the \recall{time-average number of
jobs in the system} is given by
\begin{equation}\label{eq:11}
\frac 1 t\int_0^t L(s)\d s =  \frac 1 t\int_0^t (A(s)-D(s)) \d s,
\end{equation}
where we use that $L(s)=A(s) - D(s) + L(0)$ is the total number of jobs in
the system at time $s$ and $L(0)=0$, cf. Figure~\ref{fig:atltdt}.  Observe from the second equation that $\int_0^t L(s)\d s$ is the area enclosed between the graphs of $\{A(t)\}$
and $\{D(t)\}$. Assuming the limit exists for $t\to\infty$, we define
\begin{equation}
  \label{eq:46}
  \E L = \lim_{t\to\infty} \frac 1 t\int_0^t L(s) \d s. 
\end{equation}
Observe that, notwithstanding that the symbols are the same, this
expectation need not be the same as~\eqref{eq:EQ}: in general,
\begin{equation*}
  \lim_{t\to\infty} \frac 1 t\int_0^t L(s)\, \d s \neq   \lim_{n\to\infty}\frac 1n  \sum_{k=1}^n L(A_k-).
\end{equation*}

\begin{exercise}
Design a queueing system to show that average number of jobs in the system as seen by the server is very different from what the customers see.
  \begin{hint}
Consider a queueing system with constant service and inter-arrival times.
  \end{hint}
\begin{solution}
  Take $X_k = 10$ and $S_k = 10-\epsilon$ for some tiny
  $\epsilon$. Then $L(t) = 1$ nearly all of the time. In fact,
  $\E L = 1-\epsilon/10$. However, $L(A_k-)=0$ for all $k$.
\end{solution}
\end{exercise}

Next, define the following probability as the
\emph{time-average fraction of time the system contains at most $m$
  jobs}:
\begin{equation}
  \label{eq:47}
  \P{L\leq m} =\lim_{t\to\infty} \frac 1 t\int_0^t \1{L(s)\leq m} \d s.
\end{equation}
Again, this probability need not be the same as what customers see
upon arrival.  

\begin{exercise}
Formulate a definition  for the  time-average of the waiting time.
\begin{solution}
  \begin{equation*}
    \E{W} = \lim_{t\to\infty} \frac1t \int_0^t W(s) \d s.
  \end{equation*}
\end{solution}
\end{exercise}



\begin{figure}[hp]
  \centering
  \begin{tikzpicture}[node distance = 2.5cm]
\tikzset{
    %Define standard arrow tip
    >=stealth',
    %Define style for boxes
    % Define arrow style
    pil/.style={
           ->,
           thick,
           shorten <=2pt,
           shorten >=2pt,}
}

% Define block styles
\tikzstyle{block} = [rectangle, draw,text centered, rounded corners, minimum height=3em]

    % nodes
    \node [block, fill=red!50] (X_k) {$\{X_k\}$};
    \node [block, right=2.5cm of  X_k,fill=red!50] (A_k) {$\{A_k\}$}
    edge[pil,bend left=45] node[below] {$X_k := A_k - A_{k-1}$} (X_k)
    edge[pil,<-, bend right=45] node[above] {$A_k := A_{k-1} + X_{k-1}$} (X_k); 
    \node [block, right=2.5cm of  A_k ] (A_t) {$\{A(t)\}$}
    edge[pil,bend left=45] node[below] {$A_k := \inf\{t: A(t)\geq k\}$} (A_k)
    edge[pil,<-, bend right=45] node[above] {$A(t) := \max\{k: A_k\leq t\}$} (A_k); 
    \node [block, below=2cm of  X_k ] (EX) {$\frac 1n \sum_{k=1}^n X_k \to \E X$}
    edge[pil, <-] (X_k);
    \node [block, below=2cm of  A_t ] (lambda) {$\frac{A(t)}t  \to \lambda$}
    edge[pil, <-] (A_t);
    \node [block, below=2cm of  A_k ] {$\E X = \lambda^{-1}$}
    edge[pil, <-] (EX)
    edge[pil, <-] (lambda);

    \node[below=1cm of lambda] (dummy) {}; 

    \node [block, below=1cm of EX, fill=red!50] (S_k) {$\{S_k\}$};
    \node [block, right=2.5cm of  S_k ] (U_k) {$\{U_k\}$}
    edge[pil,<->] (S_k);
    \node [block, right=2.5cm of  U_k ] (U_t) {$\{U(t)\}$}
    edge[pil,<->]  (U_k);
    \node [block, below=1cm of  S_k ] (ES) {$\frac 1n \sum_{k=1}^n S_k \to \E S$}
    edge[pil, <-] (S_k);
    \node [block, below=1cm of  U_t ] (mu) {$\frac{U(t)}t  \to \mu$}
    edge[pil, <-] (U_t);
    \node [block, below=1cm of  U_k ] {$\E S = \mu^{-1}$}
    edge[pil, <-] (ES)
    edge[pil, <-] (mu);

    \node [block, right=of dummy] {Stability:\newline $\lambda < \mu$}
edge[pil,bend right=25, <-] (lambda.east)
edge[pil,bend left=25, <-] (mu.east);

    \node[block, below=1cm of  ES, fill=red!50 ] (W_k) {$W_{k}=\max\{W_{k-1} - X_k,0\} +S_k$};
    %edge[pil, bend left=45,<-]  (X_k)
    %edge[pil, bend right=55,<-]  (S_k.west);
    \draw[->] (S_k.west) [out=180, in=110] to  (W_k.north west);
    \draw[->] (X_k.west) [out=230, in=110] to  (W_k.north west);
    \node[block, right=1cm of  W_k, fill=red!50 ] (D_k) {$D_{k}=A_k + W_{k}$} 
    edge[pil,bend right=25, <-]  (A_k)
    edge[pil,<-]  (W_k);
    \node[block, right=1cm of  D_k ] (D_t) {$D(t)=\max\{k: D_k\leq t\}$} 
    edge[pil, <-]  (D_k);

    \node[block, below=1cm of  W_k, fill=blue!40] (W) {$\frac1n\sum_{k=1}^n W_k \to \E W$} 
    edge[pil, <-]  (W_k);
    \node[block, below=1cm of  D_t] (Q_t) {$L(t) := A(t) - D(t)$} 
    edge[pil, <-]  (D_t);
    \draw[->] (A_t.east) [out=20, in = 40] to  (Q_t);
    %edge[pil, bend right=95, <-]  (A_t.north);

    \node[block, right=1cm of  D_t] (delta) {$\frac{D(t)}t \to \delta$} 
    edge[pil, <-]  (D_t);


    \node[block, below=1cm of  D_k, fill=blue!40] (L) {$\frac 1 t \int_0^t L(s)\,\d s \to \E L$} 
    edge[pil, <-]  (Q_t);
    % \node[block, below=1cm of  L] (Little) {$\E L = \lambda \E W$} 
    % edge[pil, <-]  (L)
    % edge[pil, <-]  (W);

    \node[block, right=1cm of  Q_t] (hoi) {$\delta \leq \lambda$} 
    edge[pil, <-]  (delta)
    edge[pil, bend right=10, <-]  (lambda)
    edge[pil, bend left = 30, <-]  node[below] {$L(t)>0$} (Q_t);

    \node[block, below=1cm of W, fill=blue!40] (PW) {$\frac 1n \sum_{k=1}^n \1{W_k \leq w} \to \P{W\leq w}$}
    edge[pil, <-] (W);

    \node[block, below=1cm of Q_t, fill=blue!40] (PL) {$\frac 1t \int_{0}^t \1{L(s) \leq l} \to \P{L\leq l}$}
    edge[pil, <-] (Q_t);

% \node[block, below=1cm of PW, fill=gray!40] (PM) {Performance measures};
% \node[block, below=1cm of PM, fill=gray!40] {$G/G/1$ Construction \& simulation};

    % \node[block, below=1cm of L, text width=2cm, fill=gray!40] (perf) {Performance measures}
    % edge[pil, ->] (L)
    % edge[pil, ->] (PL)
    % edge[pil, ->] (PW)
    % edge[pil, ->] (W);

  \end{tikzpicture}  

  \caption{Here we sketch the relations between the construction of
    the $G/G/1$ queue from the primary data, i.e., the inter-arrival
    times $\{X_k; k\geq 0\}$ and the service times $\{S_k; k\geq 0\}$,
    and different performance measures. 
% The performance measures are
%     shown in \protect\tikz \protect\node[fill=blue!40] {blue};, the
%     essential components for the construction of the $G/G/1$ are shown
%     in \protect\tikz \protect\node[fill=red!40] {red}.
} 
    \label{fig:constructiongg1}

\end{figure}



\begin{exercise}
  Consider a discrete-time model of a queueing system, as we developed
  in Section~\ref{sec:constr-discr-time}.  The average number of customers
  that \emph{see upon arrival} more than~$m$ customers in the system
  cannot be defined as~\eqref{eq:Qm}. Provide a better definition. 
  \begin{hint}
Why is \eqref{eq:Qm} not the same as the number of
  batches that see a queue length less than~$m$?
  \end{hint}
  \begin{solution} Since we deal with a system in discrete time, $L_k$
    is the queue length at the end of period~$k$. Thus,
    $\sum_{k=1}^n \1{L_k > m}$ counts the number of \emph{periods}
    that the queue is larger than $m$. This is of course not the same
    as the number of \emph{jobs} that see a queue larger than $m$;
    only when $a_k>0$ the customers in a batch would see a queue
    $L_k>m$. Thus,
    \begin{equation*}
      \sum_{k=1}^n \1{L_k > m} \1{a_k > 0},
    \end{equation*}
    counts the number of batches. 

    Next, by assumption, $a_k$ customers arrive during period $k$. The
    first of these customers sees a queue length of $L_{k-1} - d_k$,
    the second $L_{k-1}-d_k + 1$, and so on until the last customer
    who sees a queue length of $L_{k-1} - d_k + a_k -1 = L_k
    -1$.
    Thus, of all jobs the last sees the largest queue. Hence, if
    $L_k \leq m$, all customers of the batch see a queue less than
    $m$. If, however, $L_k > m$, then $L_k -m$ customers saw $m$ or
    more jobs in the system. Therefore, the fraction of arrivals that
    see a queue with $m$ or more jobs is equal to
\begin{equation*}
  \frac 1 {A(n)} \sum_{k=1}^n (L_k - m) \1{L_k > m} .
\end{equation*}
We could also define this a bit differently. Suppose that we don't
want to distinguish between jobs in a batch, but simply want to say
that if one job sees a long queue, all see a long queue. In that case,
\begin{equation*}
\frac 1{A(n)}\sum_{k=1}^n a_k \1{L_k > m}.
\end{equation*}
Thus, when jobs arrive in batches, the definition of loss fraction
requires some care; not all definitions need to measure the same.
  \end{solution}
\end{exercise}



\Closesolutionfile{hint}
\Closesolutionfile{ans}

\opt{solutionfiles}{
\subsection*{Hints}
\input{hint}
\subsection*{Solutions}
\input{ans}
}


%\clearpage
  



%%% Local Variables:
%%% mode: latex
%%% TeX-master: "../queueing_book"
%%% End:
