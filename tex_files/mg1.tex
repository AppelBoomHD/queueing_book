\section{$M/G/1$ Queue: Expected Waiting Time}
%{$\mathbf{M/G/1}$ Queue: Expected Waiting Time}
\label{sec:mg1}



\opt{solutionfiles}{
\subsection*{Theory and Exercises}
\Opensolutionfile{hint}
\Opensolutionfile{ans}
}

In many practical single-server queueing systems the service times are not really well approximated by the exponential distribution.
The $M/G/1$ queue then becomes a better model than the $M/M/1$ queue.
In this section we first present a formula to compute the average waiting time in queue for the $M/G/1$ queue, and then we derive it by means of sample path arguments.
The derivation is also of general interest as it develops some general results of renewal theory.


The fundamentally important \recall{Pollaczek-Khinchine formula}, or \recall{PK formula}, for the average waiting time in queue for the $M/G/1$ queue has the form
\begin{equation} \label{eq:710}
% \E{W_Q} = \frac{\E{S_r}}{1-\rho} =\frac 1 2 \frac{\lambda \E{S^2}}{1-\rho} =\frac{1 + C_s^2}2 \frac{\rho}{1-\rho} \E S.
 \E{W_Q} = \frac{1 + C_s^2}2 \frac{\rho}{1-\rho} \E S.
\end{equation}
Before deriving this formula, let us apply it.

\begin{extra}
 Show that when services are exponential, the expected waiting time $\E{W_Q(M/G/1)}$ reduces to $\E{W_Q(M/M/1)}$.
\begin{solution}
 Since the SCV for the exponential distribution is $1$, hence
 $C_s^2=1$, we get 
 \begin{equation*}
\E{W_Q} = \frac{1+1}2\frac{\rho}{1-\rho} \E S = \frac{\rho}{1-\rho} \E S,
 \end{equation*}
which we also found in a previous section.
\end{solution}
\end{extra}


\begin{extra}
 Compute $\E{W_Q}$ and $\E L$ for the $M/D/1$ queue.
\begin{hint}
Use that $\V S=0$ for the $M/D/1$ queue.    
\end{hint}
\begin{solution}
 For the $M/D/1$ the service time is deterministic.
 Thus, the service time $S=T$ always.
 Hence, $\E S = T$ and $\E{S^2}=T^2$, hence $\V S = \E{S^2} - (\E S)^2 = 0$, hence $C_s^2 = 0$.
 From the Pollaczek-Khinchine formula, the first term is $1+C_s^2=1$ for the $M/D/1$ queue, and $1+C_s^2=2$ for the $M/M/1$ queue.
 Hence,
 \begin{equation*}
\E{W_Q(M/D/1)} = \frac{\E{W_Q(M/M/1)}}2.
\end{equation*}
Therefore,
\begin{align*}
 \E{L(M/D/1)} &= \E{L_Q(M/D/1)} +\rho = \frac{\E{L_Q(M/M/1)}}2 + \rho \\
&= \frac{\rho^2}{2(1-\rho)} + \rho=\frac{\rho(2-\rho)} {2(1-\rho)}.
\end{align*}
\end{solution}
\end{extra}

\begin{extra}
 Compute $\E L$ for the $M/G/1$ queue with $S\sim U[0,\alpha]$.
\begin{hint}
Integrate $\alpha^{-1} \int_0^\alpha x \d x$, and likewise for the second moment.
\end{hint}
\begin{solution}
 \begin{align*}
\E S &= \alpha/2,\\
\E{S^2} &= \int_0^\alpha x^2 \d x/\alpha = \alpha^2/3,\\
\V S &= \alpha^2/3 - \alpha^2/4= \alpha^2/12,\\
C_s^2 &= (\alpha^2/12)/(\alpha^2/4) = 1/3,\\
\rho &= \lambda \alpha/2,\\
\E{W_Q} &= \frac{1+C_s^2}2 \frac{\lambda \alpha/2}{1-\lambda \alpha/2}\frac \alpha2, \\
\E W &= \E{W_Q} + \frac \alpha2,\\
\E L &= \lambda \E W.
 \end{align*}
\end{solution}
\end{extra}

\begin{exercise}\clabel{ex:l-240}
 A queueing system receives Poisson arrivals at the rate of 5 per hour.
 The single server has a uniform service time distribution, with a range of 4 minutes to 6 minutes.
 Determine $\E{L_Q}$, $\E L$, $\E{W_Q}$, $\E W$.
\begin{solution}
First the load.
\begin{pyconsole}
labda = 5./60 # arrivals per minute
a = 4.
b = 6.
ES = (a+b)/2. # service time in minutes
rho = labda*ES
rho
\end{pyconsole}

Next, the variance and SCV. With this the waiting times follow right away.
\begin{pyconsole}
Var = (b-a)*(b-a)/12.
SCV = Var/(ES**2)


Wq = (1+SCV)/2.*rho/(1.-rho)*ES
Wq # in minutes
Wq/60. # in hour


W = Wq + ES
W
Lq = labda*Wq
Lq
L = labda*W
L
\end{pyconsole}
\end{solution}
\end{exercise}



\begin{exercise}\clabel{ex:l-241}
 Consider a workstation with just one machine.
 We model the job arrival process as a Poisson process with rate $\lambda=3$ per day.
 The average service time $\E S = 2$ hours, $C^2_s = 1/2$, and the shop is open for 8 hours.
 What is $\E{W_Q}$?

 Suppose the expected waiting time has to be reduced to 1h.
 How to achieve this?
\begin{solution}
 $\E{W_Q} = 4.5$ h. $\rho = \lambda \E S = (3/8)\cdot 2 = 3/4$.

 One way to increase the capacity/reduce the average service time is to choose $\E S =1$ hour and reduce $C^2_s$ to $1/4$.
 Of course, there are many more ways.
 Reducing $C^2_S$ to zero is (nearly) impossible or very costly.
 Hence, $\rho$ must go down, i.e., capacity up or service time down.
 Another possibility is to plan the arrival of jobs, i.e., reduce the variability in the arrival process.
 However, typically this is not possible.
 For instance, would you accept this as a customer?
\end{solution}
\end{exercise}


\begin{exercise}[Hall 5.16] \clabel{ex:46}
The manager of a small firm would like to determine which of two people to hire.
 One employee is fast, on average, but somewhat inconsistent.
 The other is a bit slower, but very consistent.
 The first has a mean service time of $2$ minutes, with a standard deviation of 1 minute.
 The second has a mean service time of $2.1$ minutes, with a standard deviation of $0.1$ minutes.
 If the arrival rate is Poisson with rate 20 per hour, which employee would minimize $\E{L_Q}$?
 Which would minimize $\E L$?
\begin{solution}
 The arrival process is assumed to be Poisson. There is also
 just one server. Hence, we can use the PK formula to compute the average queue length.

\begin{pyconsole}
labda = 20./60 # per minute
ES = 2. # minutes
sigma = 1.
SCV = sigma*sigma/(ES*ES)
rho = labda*ES
rho

Wq = (1+SCV)/2 * rho/(1-rho) * ES
Wq
Lq = labda * Wq # Little's law
Lq
W = Wq + ES
W
L = labda * W
L
\end{pyconsole}


\begin{pyconsole}
ES = 2.1
SD = 0.1
SCV = SD**2/ES**2
rho = labda*ES
rho

Lq = rho**2/(1.-rho)*(1.+SCV)/2.
Lq
L = rho + Lq
L
\end{pyconsole}

\end{solution}
\end{exercise}


\begin{extra}
 Show that for the $M/G/1$ queue, the expected idle time is
 $\E I = 1/\lambda$. 
\begin{hint}
What is the average time between two
 arrivals? Observe that the inter-arrivals are memoryless, hence the
 average time until the next arrival after the server becomes idle
 is also $1/\lambda$.
\end{hint}
\begin{solution}
 Since the inter-arrival times are memoryless, the expected time to the first arrival after the system becomes empty is also $\E X=1/\lambda$.
\end{solution}
\end{extra}

\begin{extra}
 Express the utilization of the $M/G/1/1$ queue in terms of $\lambda$ and $\E S$.
\begin{solution}
 The system can contain at most 1 job.
 Necessarily, if the system contains a job, this job must be in service.
 All jobs that arrive while the server is busy are rejected.
 Just after a departure, the average time until the next arrival is $1/\lambda$, and then a new service starts with an average duration of $\E S$.
 After this departure, a new cycle starts.
 Thus, the utilization is $\E S/(1/\lambda + \E S) = \lambda \E S/ (1 + \lambda \E S)$.
 Since not all jobs are accepted, the utilization $\rho$ cannot be equal to $\lambda \E S$.
\end{solution}
\end{extra}

\begin{extra}
 For the $M/G/1/1$ queue what is the fraction of jobs rejected?
\begin{hint}
 The rate of accepted jobs is $\lambda \pi(0)$.
 What is the load of these jobs?
 Equate this to $1-\pi(0)$ as this must also be the load.
 Then solve for $\pi(0)$.
\end{hint}
\begin{solution}
 Let $p(0)$ be the fraction of time the server is idle. By PASTA,
 $\pi(0)=p(0)$. Thus, the rate of accepted jobs is
 $\lambda\pi(0)$. Therefore, the departure rate
 $\delta=\lambda\pi(0)$. The loss rate is
 $\lambda-\delta = \lambda (1-\pi(0))$.

 Since $\lambda\pi(0)$ is the rate at which jobs enter the system,
 the load must be $\lambda\pi(0)\E S$. Since the load is also
 $1-\pi(0)$, it follows from equating that
 \begin{equation*}
 \lambda\pi(0) \E S = 1 - \pi(0) \iff \pi(0)=\frac1{1+\lambda\E S} 
\iff 1-\pi(0) = \frac{\lambda \E S}{1+\lambda \E S},
 \end{equation*}
which is the same as in the previous problem.

Note that $\delta < \lambda$, as it should be the case. 
\end{solution}
\end{extra}


\begin{extra}
 Why is the fraction of lost jobs at a $M/G/1/1$ queue not necessarily the same as for a $G/G/1/1$ queue with the same load?
\begin{hint}
Provide an example.
\end{hint}
\begin{solution}
 Typically, in the $G/G/1$ queue, the arrivals do not see time-averages. Consequently, the fraction of arrivals that are blocked is not necessarily equal to the utilization~$\rho$.

 Again, take jobs with a duration 59 minutes and inter-arrival times of
 1 hour. The load is $59/60$, but no job is lost, also not in the
 $G/G/1/1$ case. Thus, $\delta=\lambda$ in this case.
\end{solution}
\end{extra}



To derive the PK-formula, suppose at first that we know the expected \recall{remaining service time} $\E{S_r}$, i.e., the expected time it takes to complete the job in service, if present, at the time a job arrives.




\begin{extra}\clabel{ex:32}
Show for the $M/G/1$ queue that the expected time in queue is
\begin{equation}\label{eq:24}
 \E{W_Q} = \E{S_r} + \E{L_Q} \E S.
\end{equation}
\begin{hint}
 What happens when you enter a queue? First you have to wait until the job in service (if there is any) completes, and then you have to wait for the queue to clear.
\end{hint}
\begin{solution}
It is evident that the expected waiting time for an arriving customer is the expected
remaining service time plus the expected time in queue. The expected
time in queue must be equal to the expected number of
customers in queue at an arrival epoch times the expected service time
per customer, assuming that service times are i.i.d. If the arrival
process is Poisson, it follows from PASTA that the average number of
jobs in queue perceived by arriving customers is also the
\emph{time-average} number of jobs in queue~$\E{L_Q}$. 
\end{solution}
\end{extra}


\begin{exercise}\clabel{ex:l-242}
Show that, given $\E{S_r}$, 
\begin{equation}\label{eq:35}
 \E{W_Q} = \frac{\E{S_r}}{1-\rho}.% = \frac{\rho}{1-\rho} \E{S_r\given S_r>0}.
\end{equation}
\begin{hint}
 Use~\cref{ex:32}. Then use Little's law to see that $\E{L_Q} = \lambda \E{W_Q}$. Substitute this in~\cref{eq:24} and simplify.
\end{hint}
\begin{solution}
With Little's law $\E{L_Q} = \lambda \E{W_Q}$. Using this,
\begin{equation*}
 \E{W_Q} = \E{S_r} + \lambda \E{W_Q} \E S =\E{S_r} + \rho \E{W_Q},
\end{equation*}
since $\rho=\lambda \E S$. But this gives for the $M/G/1$ queue that
\begin{equation*}
 \E{W_Q} = \frac{\E{S_r}}{1-\rho}.
\end{equation*}
\end{solution}
\end{exercise}



It remains to compute the average remaining service time $\E{S_r}$ for generally distributed service times.
Just like in~\cref{sec:mxm1-queue:-expected} we use the renewal reward theorem.
Consider the $k$th job of some sample path of the $M/G/1$ queueing process.
Let its service time start at time $\tilde A_k$ so that it departs at time $D_k=\tilde A_k + S_k$.

\begin{exercise}\clabel{ex:13}
 Use~\cref{fig:mg1remainingservicetime} to explain that the remaining service time of job $k$ at time $s$ is given by
$(D_k-s)\1{\tilde A_k \leq s < D_k}$.
With this, explain that
 \begin{equation*}
 Y(t) = \int_0^t (D_{D(s)+1}-s)\1{L(s) > 0} \d s
 \end{equation*}
is the total remaining service time as seen by the server up to $t$. 
\begin{solution}
 Observe that when $s\in [\tilde A_k, D_k)$, the remaining service time until job $k$ departs is $D_k-s $, while if $s\not \in [\tilde A_k, D_k)$, job $k$ is not in service so it cannot have any remaining service.

 At time $s$, the number of departures is $D(s)$.
 Thus, $D(s)+1$ is the first job to depart after time $s$.
 The departure time of this job is $D_{D(t)+1}$, hence the remaining service time at time $s$ is $D_{D(s)+1}-s$, provided this job is in service.
\end{solution}
\end{exercise}


\begin{figure}[htb]
 \centering
\begin{tikzpicture}[scale=1,
 open/.style={shape=circle, fill=white, inner sep=1pt, draw, node contents=},
 closed/.style={shape=circle, fill=black, inner sep=1pt, draw, node contents=}]
 \draw (-1,0) -- (4,0); 
 %x\draw (1,0) -- (3,0) node[midway, fill=white] {$s$};
 \draw node (c1) at (0,3) [closed, label={}]
 node (c2) at (3,0)[open, label={}]
 (c1) to (c2);
 \draw[dotted] (0,0) -- (0,3) node[midway, rotate=90, fill=white] {$S_k$};
 \node[below] at (0,0) {$\tilde A_k$};
 \node[below] at (3,0) {$D_k$};

 \draw[<->, dotted] (0.75,0) -- (0.75,2.25) node[midway,fill=white,rotate=90] {$D_k-s$};
 \node[below] at (0.75,0) {$s$};
 \end{tikzpicture}

 \caption{Remaining service time.}
 \label{fig:mg1remainingservicetime}
\end{figure}

\begin{exercise}\clabel{ex:l-243}
Apply the renewal reward theorem to the result of~\cref{ex:13} to prove that 
\begin{equation}\label{eq:34}
\E{S_r} = \frac{\lambda}2 \E{S^2}. 
\end{equation}
Then simplify to get~\cref{eq:710}. 
\begin{hint}
 Choose $T_k=D_k$ as epochs in the renewal reward theorem.
\end{hint}
\begin{solution}
 It is clear that the time-average $Y(t)/t \to \E{S_r}$.
 Moreover, $X_k = Y(D_k) - Y(D_{k-1})$ is the area under the triangle in~\cref{fig:mg1remainingservicetime}.
 Thus, $X=\lim_{n\to\infty} \sum_{k=1}^n S_k^2/ 2 = \E{S^2}/2$.
 Finally, $\delta = \lambda$ by rate-stability.

 The rest is plain algebra, see~\cref{ex:65}.
\end{solution}
\end{exercise}

\begin{extra}\clabel{ex:65}
Use $C_s^2 = \V S / (\E S)^2$ to show that $\lambda \E{S^2} = (1+C_s^2)\rho\E S$.
\begin{solution}
Use that 
\begin{equation*}
 \frac{\E{S^2}}{(\E S)^2} = 
 \frac{(\E{S^2}-(\E S)^2) + (\E S)^2}{(\E S)^2} =
 \frac{\V S + (\E S)^2}{(\E S)^2} =
 C_s^2 + 1.
\end{equation*}
Then
\begin{equation*}
 \lambda\E{S^2} = \frac{\E{S^2}}{(\E S)^2} \lambda(\E S)^2=
 \frac{\E{S^2}}{(\E S)^2} \rho \E S = (1+C_s^2) \rho \E{S}.
\end{equation*}
\end{solution}
\end{extra}



\begin{extra}
Use the PASTA property to show that
\begin{equation}\label{eq:37}
\E{S_r} = \rho \E{S_r\given S_r>0}.
\end{equation}
\begin{hint}
 Use~\cref{ex:28}.
\end{hint}
\begin{solution}
By the PASTA property we know that $\rho$ is the probability to find the server busy upon arrival; hence $1- \rho$ is the probability that the server is idle upon arrival. Then,
\begin{equation*}
\E{S_r} = \rho \E{S_r\given S_r >0} + (1-\rho) \E{S_r\given S_r = 0} = \rho \E{S_r\given S_r>0},
\end{equation*}
since, evidently, $\E{S_r\given S_r=0}=0$. 
\end{solution}
\end{extra}

\begin{extra}
 It is an easy mistake to think that $\E{S_r} = \E S$ when service
 times are exponential. Why is this wrong?
\begin{hint}
Realize again that $\E{S_r}$ includes the jobs that arrive at an empty system.
\end{hint}
\begin{solution}

 $\E{S_r \given S_r>0} = \E S$ for the $M/M/1$ queue, and
 $\E{S_r} = \rho \E{S_r \given S_r>0}$ for the $M/G/1$ queue, it
 follows that
 \begin{equation*}
 \E{S_r} = \rho \E{S_r\given S_r>0} = \rho \E S.
 \end{equation*}
\end{solution}
\end{extra}




\begin{exercise} \clabel{ex:9}
Show from~\cref{eq:34} that 
\begin{equation}\label{eq:55}
\E{S_r\given S_r>0} = \frac{\E{S^2}}{2 \E S}.
\end{equation}
\begin{solution}
 From~\cref{eq:37} and the sentences above this equation,
 we have that
 \begin{equation*}
 \E{S_r} = \rho \E{S_r\given S_r>0}=\lambda \E S \E{S_r\given S_r>0}.
 \end{equation*}
Hence,
 \begin{equation*}
 \lambda \E{S_r\given S_r>0} = \frac{\E{S_r}}{\E S} = \lambda \frac{\E{S^2}}{2 \E S}.
 \end{equation*}
\end{solution}
\end{exercise}


\begin{extra}\clabel{ex:45}
When $\V{S} = 0$, is it true that
\begin{equation*}
\E{S_r\given S_r>0} = \frac{\E{S^2}}{2 \E S} \implies \E {S_r\given S_r>0} = \frac{\E S} 2?
\end{equation*}
\begin{hint}
 $\V S = 0$ implies that $S$ is deterministic.
\end{hint}
\begin{solution}
 When $S$ is deterministic $\E{S^2} = (\E{S})^2$. Now use~\cref{eq:55} to see that it is true.
\end{solution}
\end{extra}

\begin{extra}
 What is $\E{S_r\given S_r>0}$ for the $M/D/1$ queue? 
\begin{solution}
By~\cref{ex:45}, it is half the service time. 
\end{solution}
\end{extra}

\begin{extra}
Show that $\E{S_r\given S_r>0} = \alpha/3$ when $S \sim U[0, \alpha]$.
\begin{solution}
 When $S$ is uniform on $[0, \alpha]$,
 \begin{align*}
 \E S &= \alpha^{-1}\int_0^\alpha x \, \d x =\frac{\alpha^2}{2 \alpha} = \frac{\alpha}2, 
&
\E{S^2} &= \alpha^{-1}\int_0^\alpha x^2 \, \d x = \frac{\alpha^2}3. 
 \end{align*}
 Thus, with~\cref{eq:55}, $\E{S_r\given S_r>0} = (\alpha^2/3)/(2 \alpha/2) = \alpha/3$. 
\end{solution}
\end{extra}

\begin{extra}
 Show that $\E{S_r\given S_r>0} = \mu^{-1}$ when $S\sim \Exp(\mu)$.
\begin{solution}
 \begin{align*}
\E S &= \mu \int_0^\infty x e^{-\mu x}\,\d x = 1/\mu, \\
\E{S^2} 
&= \mu \int_0^\infty x^2 e^{- \mu x} \,\d x = - \left. x^2 e^{-\mu x} \right|_0^\infty + 2 \int_0^\infty x e^{-\mu x} \, \d x \\
&= -\left. 2\frac x \mu e^{-\mu x} \right|_0^\infty + \frac 2 \mu\int_0^\infty e^{-\mu x} \, \d x =\frac2{\mu^2}.
 \end{align*}
Now use~\cref{eq:55}.
\end{solution}
\end{extra}



\opt{solutionfiles}{
\Closesolutionfile{hint}
\Closesolutionfile{ans}
\subsection*{Hints}
\input{hint}
\subsection*{Solutions}
\input{ans}
}

%\clearpage




%%% Local Variables:
%%% mode: latex
%%% TeX-master: "../companion"
%%% End:
