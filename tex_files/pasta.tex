\section{Poisson Arrivals See Time Averages}
\label{sec:poisson-arrivals-see}


\opt{solutionfiles}{
\subsection*{Theory and Exercises}
\Opensolutionfile{hint}
\Opensolutionfile{ans}
}

Suppose the following limit exists:
\begin{equation}\label{eq:jaap}
  \pi(n) 
= \lim_{m\to\infty} 
\frac1m\sum_{k=1}^m \1{L(A_k-) = n},
\end{equation}
then $\pi(n)$ is the long-run fraction of jobs that observe $n$ customers in the system at the moment an arbitrary job arrives.
It is natural to ask whether $\pi(n)$ and $p(n)$, as defined by~\eqref{eq:p(n)}, are related, that is, whether what customers see upon arrival is related to the time-average behavior of the system.
In this section we will derive the famous \recall{Poisson arrivals see time averages} (\recall{PASTA}) condition that ensures that $\pi(n)=p(n)$ if jobs arrive in accordance with a Poisson process.


%We can make some progress by rewriting $\pi(n)$ in the following way.
Since $A(t)\to \infty$ as $t\to\infty$, it is reasonable that (see Exercise~\ref{ex:18} for a proof)
\begin{equation}\label{eq:132}
  \begin{split}
  \pi(n) &= \lim_{t\to\infty} \frac1{A(t)}\sum_{k=1}^{A(t)} \1{L(A_k-) = n} 
= \lim_{t\to\infty} \frac1{A(t)}\sum_{k=1}^\infty \1{A_k \leq t, L(A_k-) = n} \\
  &= \lim_{t\to\infty} \frac{A(n,t)}{A(t)},
  \end{split}
\end{equation}
where we use~\eqref{eq:19} in the last row. But,  with~\eqref{eq:3}, 
\begin{equation}\label{eq:1333}
 \frac{A(n,t)}{t} 
= \frac{A(t)}t \frac{A(n,t)}{A(t)}
\to \lambda  \pi(n), \quad\text{as } t \to \infty, 
\end{equation}
while by ~\eqref{eq:21}, 
\begin{equation*}
\frac{A(n,t)}t = \frac{A(n,t)}{Y(n,t)}\frac{Y(n,t)}t \to \lambda(n) p(n), \quad\text{as } t \to \infty.
\end{equation*}
Thus
\begin{equation}\label{eq:13}
\lambda  \pi(n) = \lambda(n) p(n).
\end{equation}
This leads to our final result:
\begin{equation*}
  \lambda(n) = \lambda \iff \pi(n) = p(n).
\end{equation*}
This means that if the arrival rate does not depend on the state of the system, i.e., $\lambda(n)=\lambda$, the sample average is equal to the time-average.
In other words, the customer perception at arrival moments is the same as the server perception.

As the next exercises show, this property is not satisfied in general.
However, when the arrival process is Poisson we have that $\lambda(n)=\lambda$.
This fact is typically called PASTA: Poisson Arrivals See Time Averages.
Thus, for the $M/M/1$ queue in particular,
\begin{equation*}
  \pi(n) = p(n) = (1-\rho)\rho^n.
\end{equation*}

\begin{exercise}\label{ex:8} Show that for the case of Exercise~\ref{ex:112} $\pi(0)=1$ and $\pi(n)=0$, for $n>0$.
\begin{solution}
  All arrivals see an empty system. Hence $A(0,t)/A(t) \approx (t/2)/(t/2) = 1$, and $A(n,t)=0$ for $n>0$. Thus, $\pi(0) = \lim_t A(0,t)/A(t) = 1$ and $\pi(n)=0$ for $n>0$. Recall from the other exercises that $p(0)=1/2$. Hence, time average statistics are not the same as statistics at arrival moments. 
\end{solution}

\end{exercise}

\begin{exercise}
  Check that~(\ref{eq:13})  holds for the system of Exercise~\ref{ex:8}.
  \begin{solution}
From the relevant previous exercises, $\lambda = \lim_t A(t)/t = 1/2$. $\lambda(0)=1$, $p(0)=1/2$, and $\pi(0)=1$. Hence,
\begin{equation*}
  \lambda \pi(0) = \lambda(0) p(0) \implies  \frac 1 2 \times 1 = 1\times \frac 1 2.
\end{equation*}
For $n>0$ it's easy, everything is 0.
  \end{solution}
\end{exercise}





With the above reasoning, we can also establish a relation between $\pi(n)$ and the statistics of the system as obtained by the departures.
Define, analogous to \eqref{eq:132}, 
\begin{equation}
  \label{eq:33}
  \delta(n) = \lim_{t\to\infty} \frac{D(n,t)}{D(t)}
\end{equation}
as the long-run fraction of jobs that leave $n$ jobs \emph{behind}.
From Eq.~\eqref{eq:15}
%, i.e., $|A(n,t) - D(n,t)| \leq 1$.

%To obtain Eq.~\eqref{eq:12} we divided both sides of this equation by the time the system spends in a certain state.

%We can also use another form:
\begin{equation*}
\frac{A(t)}t \frac{A(n,t)}{A(t)} = \frac{A(n,t)}t \approx \frac{D(n,t)}t 
= \frac{D(t)}t \frac{D(n,t)}{D(t)}.
\end{equation*}
Taking limits at the left and right, and using~\eqref{eq:28}, we obtain for (queueing) systems in which customers arrive and leave as single units that
\begin{equation}
  \label{eq:36}
  \lambda \pi(n) = \delta \delta(n).
\end{equation}
Thus, if the system is rate-stable, statistics obtained  by arrivals is the same as statistics obtained by departures, i.e., 
\begin{equation}
  \label{eq:39}
\lambda = \delta \iff  \pi(n) = \delta(n).
\end{equation}


\begin{exercise}\label{ex:26}
  When $\lambda\neq \delta$, is $\pi(n)\geq \delta(n)$ or $\pi(n) < \delta(n)$ true? 
  \begin{hint}
    Use that    $\lambda \geq \delta$ always holds. Thus, when $\lambda \neq \delta$, it must be that $\lambda > \delta$. What are the consequences of this inequality; how does the queue length behave as a function of time?
  \end{hint}
  \begin{solution}
    The assumptions lead us to conclude that $\lambda > \delta$. As a consequence, the queue length must increase on the long run (jobs come in faster than they leave). Therefore, $A(n,t)/t \to 0$ for all $n$, and also $D(n,t)/t\to 0$. Consequenly, $\pi(n) = \delta(n) = 0$, which is the only sensible reconciliation with~\eqref{eq:36}. 
  \end{solution}
\end{exercise}

\begin{extra}
Show that 
\begin{equation*}
\lambda  \pi(n) = \lambda(n) p(n) = \mu(n+1) p(n+1) = \delta \delta(n).
\end{equation*}
What is the important condition for this to be true?
\begin{hint}
Check all definitions of $Y(n,t)/t$ and so on.
\end{hint}
\begin{solution}
  The important condition is that transitions occur as single
  steps. In other words, the relation is true for processes with
  \recall{one-step transitions}, i.e., when $|A(n,t) - D(n,t)|\leq 1$.
  In  that case, 
\begin{align*}
  \frac{A(n,t)}{t} &=   \frac{A(n,t)}{A(t)} \frac{A(t)}{t} \to \pi(n) \lambda\\
  \frac{A(n,t)}{t} &=   \frac{A(n,t)}{Y(n,t)} \frac{Y(n,t)}{t} \to \lambda(n)p(n)\\
  \frac{D(n,t)}{t} &=   \frac{D(n,t)}{Y(n+1,t)} \frac{Y(n+1,t)}{t} \to \mu(n+1)p(n+1)\\
  \frac{D(n,t)}{t} &=   \frac{D(n,t)}{D(t)} \frac{D(t)}{t} \to \delta(n)\delta. \\
\end{align*}
\end{solution}
\end{extra}



\begin{exercise}
  Why is $\mu(n) = \mu$ for the $M/M/1$ queue?  
  \begin{hint}
Think about the
    construction of the $M/M/1$ queue as a random walk, see
    Section~\ref{sec:queu-proc-as}.
  \end{hint}
 \begin{solution}
   The $M/M/1$-queue can be constructed as a reflection of the random
   walk $Z(t) = Z(0) + N_\lambda(t) - N_\mu(t)$. Clearly,
   down crossings can only occur when $N_\mu$ fires. The rate at which
   the transitions of $N_\mu$ occur is constant, and, in particular,
   independent of the history of $Z$. 

   More specifically, for the interested, define
   $\sigma\{X(t) : t\in I\}$ as the $\sigma$-algebra generated by the
   stochastic processes $\{X(t), t\in I\}$ on the index set $I$. Then,
   by construction of $\{N_\lambda(t)\}$ and $\{Z(t)\}$, we have that
   $\sigma\{Z(s) : s\in[0,t]\}$  and $\sigma\{N_\lambda(u) : u > t\}$ are independent.
 \end{solution}
\end{exercise}


\begin{exercise}
  Use PASTA and the balance equations of the $M/M/1$ queue to derive that $(\lambda + \mu) \pi(n) = \lambda \pi(n-1) + \mu \pi(n+1)$.
  \begin{hint}
    Consider some state $n$ (not a level) and count all transitions that `go in and out of' this state.
    Specifically, $A(n,t) + D(n-1,t)$ counts all transitions out of state $n$: $A(n,t)$ counts the number of arrivals that see $n$ in the system upon arrival, hence immediately after such arrivals the system contains $n+1$ jobs; likewise, $D(n-1,t)$ counts all jobs that leave $n-1$ jobs behind, hence immediately before such jobs depart the system contains $n$ jobs.
    In a similar way, $A(n-1,t) + D(n,t)$ counts all transitions into state $n$ (Recall once again, $D(n,t)$ counts the jobs that leave $n$ behind.
    Hence, when such departures occur, state $n$ is entered).
    Now use that `what goes in must go out'.
  \end{hint}
  \begin{solution}
By  the hint,  the difference between the `out
    transitions' and the `in transitions' is at most 1 for all $t$. Thus,  we can write
    \begin{align*}
\text{transitions out } &\approx \text{transitions in } \iff \\
      A(n,t) + D(n-1,t) &\approx A(n-1,t) + D(n,t)  \iff \\
      \frac{A(n,t) + D(n-1,t)}t &\approx \frac{A(n-1,t) + D(n, t)}t \iff \\
      \frac{A(n,t)}t + \frac{D(n-1,t)}t &\approx \frac{A(n-1,t)}t + \frac{D(n,t)}t.
    \end{align*}
Using the ideas of Section~\ref{sec:level-cross-balance} this becomes for $t\to\infty$, 
\begin{equation*}
  (\lambda(n) +\mu(n))p(n) = \lambda(n-1)p(n-1) + \mu(n+1)p(n+1).
\end{equation*}
Since we are concerned here with the $M/M/1$ queue we have that
$ \lambda(n) = \lambda$ and $\mu(n) = \mu$, and using PASTA we have
that $p(n) = \pi(n)$. We are done.
  \end{solution}
\end{exercise}





\begin{exercise}\label{ex:18}
  There is a subtle problem in the transition from~\eqref{eq:jaap} to~\eqref{eq:132} and the derivation of~\eqref{eq:1333}: $\pi(n)$ is defined as a limit over arrival epochs while in $A(n,t)/t$ we take the limit over time.
  Now the observant reader might ask why these limits should relate at all.
  Use the renewal reward theorem to show that~(\ref{eq:132}) is valid.
\begin{hint}
Check that the conditions of the renewal reward theorem are satisfied in the above proof~\eqref{eq:1333}. Then define  
\begin{align*}
  Y(t) &:= A(n,t) = \sum_{k=1}^{A(t)} \1{L(A_k-) = n} \\
X_k &:= Y(A_k) - Y(A_{k-1}) = A(n, A_k) - A(n, A_{k-1}) = \1{L(A_k-)=n}.
\end{align*}

\end{hint}
\begin{solution}
First we check the conditions.  The counting process here is $\{A(t)\}$ and the epochs at which
    $A(t)$ increases are $\{A_k\}$. By assumption, $A_k\to\infty$,
    hence $A(t)\to\infty$ as $t\to\infty$. Moreover, by assumption
    $A(t)/t \to \lambda$. Also $A(n,t)$ is evidently non-decreasing and
    $A(n,t)\to\infty$ as $t\to\infty$.


From the definitions in the hint,   
\begin{equation*}
X= \lim_{m\to\infty} \frac 1 m \sum_{k=1}^m X_k =\lim_{m\to\infty} \frac 1 m \sum_{k=1}^m \1{L(A_k-)=n} = \pi(n).
\end{equation*}
Since $Y=\lim_t Y(t)/t = \lim_t A(n,t)/t$ it follows from the renewal reward theorem that
\begin{equation*}
  Y=\lambda X \implies \lim_t \frac{A(n,t)} t = \lambda X = \lambda \pi(n).
\end{equation*}
Thus, Eq.~\eqref{eq:1333} follows from the renewal reward theorem.
\end{solution}
\end{exercise}

With the PASTA property we can determine the distribution of the inter-departure times of the $M/M/1$ queue.
Observing that in a network of queues the departures from one queueing station form the arrivals at another station, we can use this result to analyze networks of queues

\begin{exercise}\label{ex:burke}
  Try to prove \recall{Burke's law} which states that the departure process of the $M/M/1$ queue is a Poisson process with rate $\lambda$.
\begin{solution}
  The list of extra exercises, see the companion file, show that inter-departures times have the same density, i.e., $\lambda e^{-\lambda t}$.
  It can also be shown that the inter-departure times are independent.

Thus, the inter-departures times form a set of i.i.d.
exponentially distributed random variables with mean $1/\lambda$.
Consequently, the departures times form a Poisson process with rate $\lambda$.

\end{solution}
\end{exercise}



\begin{extra}\label{ex:dep}
Why is the output rate of the (stable) $M/M/1$ queue equal to~$\lambda$ and not~$\mu$?
\begin{solution}
Jobs arrive at rate $\lambda$. For a stable queue, $\mu>\lambda$. Moreover,  jobs can never leave faster than they arrive.
\end{solution}
\end{extra}


\begin{extra}
  Why is $\mu e^{-\mu t}$ not a reasonable density for the inter-departure times?
  In fact, the simplest guess for the inter-departure density might be $\lambda e^{-\lambda t}$; so this is what we will try to prove below.
We will focus on departure moments and use~\eqref{eq:39}, in particular that departures `see' what arrivals `see', i.e., $\delta(n)= \pi(n)$, and PASTA.
\begin{solution}
         Because jobs do not leave at rate $\mu$. 
    \end{solution}
\end{extra}




\begin{extra}\label{ex:28}
Show that the probability that a job leaves behind a busy station is $\rho$, hence $1-\rho$ is the probability to leave an idle server behind.
    \begin{solution}
Observe that $\rho$ is the fraction of time the server is busy. Then, from  PASTA, the fraction of jobs that see a busy server is also $\rho$.  This fraction of jobs is $\sum_{n=1}^\infty \pi(n)$. Finally, $\delta(n) = \pi(n)$ , a fraction $\rho$ of the departures leaves a busy system behind.

    \end{solution}
\end{extra}


\begin{extra}\label{ex:17}
 If job $n-1$, say, leaves behind an empty system, show that the expected time until the next departure is $\E{D_n - D_{n-1}} = 1/\lambda + 1/\mu$. 
    \begin{hint}
      After job $n-1$ left, job $n$ has to arrive, so we need to wait first for this inter-arrival time. Then job $n$ must be served. This adds up to $1/\lambda + 1/\mu$. 
    \end{hint}
    \begin{solution}
With the hint, we first have to wait for an inter-arrival
    time $X_n$. Then, since job $n$'s service starts right away, it
    leaves when $D_n = D_{n-1}+X_n + S_n$. Now observe that, due to the memoryless property of the inter-arrival times, $\E{X_n} = \E{A_n - D_{n-1}} = 1/\lambda$. Thus, the expected duration is $\E{X_n + S_n}=1/\lambda + 1/\mu$. 
    \end{solution}
\end{extra}

\begin{extra}
Show that the density of $D_{n} - D_{n-1}$ is
    \begin{equation*}
    f_{X+S}(t) = \frac{\lambda \mu}{\lambda - \mu} (e^{-\mu t} - e^{-\lambda t})
    \end{equation*}
if the server is idle after $D_{n-1}$.
    \begin{solution}
      By the previous point, the density of $D_{n} - D_{n-1}$ is the
      same as the density of $X_n + S_n$.  Since $\{X_n\}$ and $\{S_n\}$ are both i.i.d. sequences, the problem becomes to find the density of $X+S$.  We will use two ways of computing this. 

Since $X\sim \Exp(\lambda)$ and $S\sim\Exp(\mu)$, and $X$ and $S$ are independent, their joint density is $f_{X,S}(x,y) = \lambda \mu e^{-\lambda x - \mu y}$. With this,
  \begin{align*}
\P{X+S\leq t } 
&= \lambda \mu \int_0^\infty \int_0^\infty e^{-\lambda x - \mu y} \1{x+y\leq t} \d x \d y \\
&= \lambda \mu \int_0^t \int_0^{t-x} e^{-\lambda x - \mu y} \d y \d x \\
&= \lambda \mu \int_0^t e^{-\lambda x} \int_0^{t-x} e^{- \mu y} \d y \d x \\
&= \lambda \int_0^t e^{-\lambda x} (1-e^{- \mu (t-x)} ) \d x  \\
&= \lambda \int_0^t e^{-\lambda x}  \d x - \lambda e^{-\mu t} \int_0^t e^{(\mu-\lambda) x} \d x \\
&= 1- e^{-\lambda t} - \frac{\lambda}{\mu-\lambda} e^{-\mu t} ( e^{(\mu-\lambda) t} -1) \\
&= 1- e^{-\lambda t} - \frac{\lambda}{\mu-\lambda} e^{-\lambda t} + \frac{\lambda}{\mu-\lambda} e^{-\mu t} \\ 
&= 1 - \frac{\mu}{\mu-\lambda} e^{-\lambda t} + \frac{\lambda}{\mu-\lambda} e^{-\mu t}. \\
  \end{align*}
The density $f_{X+S}(t)$ is the derivative of this expression with respect to~$t$, hence,
\begin{align*}
  f_{X+S}(t) 
&= \frac{\lambda\mu}{\mu-\lambda} e^{-\lambda t}  - \frac{\mu \lambda}{\mu-\lambda} e^{-\mu t} \\
&= \frac{\lambda\mu}{\lambda -\mu}(e^{-\mu t} - e^{-\lambda t}). \\
\end{align*}

Conditioning is much faster, but requires the concept of conditional density. You can skip the rest if you are not interested. 
    \begin{align*}
    f_{X+S}(t) 
&= \P{X+S\in \d{t}} \\
&= \int \P{S+x\in \d{t}}\P{X\in \d{x}} \\
&=\int_0^t f_S(t-x) f_X(x) \d{x} \\
     &= \int_0^t \mu e^{-\mu(t-x)} \lambda e^{-\lambda x} \d{x} \\
     &= \lambda \mu e^{-\mu t} \int_0^t  e^{x(\mu-\lambda)} \d{x} \\
&= \frac{\lambda \mu}{\lambda - \mu}\left(e^{-\mu t} - e^{-\lambda t}\right).
    \end{align*}
    \end{solution}
\end{extra}


\begin{extra}
Show  that when the queue is not empty at a departure time, the density of the next inter-departure time is $f_D(t) = \mu e^{-\mu t}$.
    \begin{solution}
After the departure, the server can start right away with the job at the head of the queue. The inter-departure time of this job is $\Exp(\mu)$.
    \end{solution}
\end{extra}

\begin{extra}
Use conditioning on the server being idle or busy at a departure to show that  the density of  the inter-departure time is $\lambda e^{-\lambda t}$.
  \begin{hint}
Conditioning leads to 
\begin{equation*}
    f_D(t) = f_{X+S}(t) \P{\text{server is idle}} + f_S(t) \P{\text{ server is busy }}= (1-\rho) f_{X+S}(t) +
    \rho \mu e^{-\mu t}.
\end{equation*}
    Now use the above exercises to simplify.
  \end{hint}
  \begin{solution}
       \begin{align*}
    f_D(t) 
&= (1-\rho) f_{X+S}(t) +    \rho \mu e^{-\mu t} \\
&= (1-\rho) \frac{\mu\lambda}{\lambda-\mu} \left(e^{-\mu t}-e^{-\lambda t}\right) +    \rho \mu e^{-\mu t} \\
&= \left(1-\frac{\lambda}\mu\right) \frac{\mu\lambda}{\lambda-\mu}\left(e^{-\mu t}-e^{-\lambda t}\right)  +    \rho \mu e^{-\mu t} \\
&= \frac{\mu-\lambda}\mu \frac{\mu\lambda}{\lambda-\mu}\left(e^{-\mu t}-e^{-\lambda t}\right)  +    \frac\lambda \mu \mu e^{-\mu t} \\
% &= \frac{\mu-\lambda}\mu \frac{\mu\lambda}{\lambda-\mu}\left(e^{-\mu t}-e^{-\lambda t}\right)  +    \lambda e^{-\mu t} \\
&= - \lambda\left(e^{-\mu t}-e^{-\lambda t}\right)  +    \lambda e^{-\mu t} \\
&=  \lambda e^{-\lambda t}.
      \end{align*}
  \end{solution}
\end{extra}




\opt{solutionfiles}{
\Closesolutionfile{hint}
\Closesolutionfile{ans}
\subsection*{Hints}
\input{hint}
\subsection*{Solutions}
\input{ans}
}
%\clearpage


%%% Local Variables:
%%% mode: latex
%%% TeX-master: "../queueing_book"
%%% End:
