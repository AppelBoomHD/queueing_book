\section{Old exam Questions}

\Opensolutionfile{ans}
\subsection{Multiple-choice Questions}

\begin{exercise}[201807]
Sakasegawa's approximation for the waiting time in a $G/G/c$ queue is 
\begin{equation*}
  \E{W_Q} = \frac{C_a^2+C_s^2}2 \frac{\rho^{\sqrt{2(c+1)}-1}}{c(1-\rho)} \E S.
\end{equation*}
We claim that it is exact for the $M/M/1$ queue.
\begin{solution}
Answer = A.
\end{solution}
\end{exercise}

\begin{exercise}[201807]
  A production system consists of 2 stations in tandem. The first station has one machine, the second has two identical machines. Machines never fail and service times are deterministic. Jobs arrive at rate 1 per hour. The machine at first station has a service time of 45 minutes per job, a machine at the second station has a service time of 80 minutes. We claim that the second  station is the bottleneck. 
\begin{solution}
Answer = B. The second station has a utilization of $80/(2*60) = 8/12 = 2/3$, while  the first has a utilization of $45/60 = 3/4$, which is higher.
\end{solution}
\end{exercise}


\begin{exercise}[201904]
  A job's normal service time, without interruptions, is given by $S_0$.
  The durations of interruptions are given by the i.i.d.
  random variables $\{R_i\}$ and have common mean $\E R$ and variance $\V R$.
  If $N$ interruptions occur, the effective service time will then be
\begin{equation*}
S= S_0 + \sum_{i=1}^N R_i.
\end{equation*}
Then all steps in the computation below are correct: 
\begin{align*}
  \E{\sum_{i=1}^N R_i} 
&=  \E{ \sum_{n=0}^\infty \1{N=n}} \E{ \sum_{i=1}^n R_i } = \E N \E R
\end{align*}
\begin{solution} Answer = B,~\cref{ex:16}
\end{solution}
\end{exercise}


\begin{exercise}[201907] 
When a bus reaches the end of its line, it undergoes a series of inspections.
  The entire inspection takes 5 minutes on average, with a standard deviation of 2 minutes.
  Buses arrive with inter-arrival times uniformly distributed on $[3,9]$ minutes.
  We model the number of buses waiting to be inspected as an $G/G/1$ queue.
  The code below correctly computes the waiting time in queue.
\begin{pyconsole}
a = 3.
b = 9. 
EX = (b+a)/2. # expected inter-arrival time
EX
labda = 1./EX # per minute
VA = (b-a)*(b-a)/12.
CA2 = VA/(EX*EX)
ES = 5.
sigma = 2
VS = sigma*sigma
CS2 = VS/(ES*ES)
rho = labda*ES
Wq = (CA2+CS2)/2. * rho/(1.-rho) * ES
\end{pyconsole}

\begin{solution}
Answer = A, see~\cref{ex:91}.
\end{solution}
\end{exercise}

\begin{exercise}[201907]
A station contains 3 identical machines in parallel. Jobs arrive as a Poisson process with rate $3$ per hour. Service times are exponential with a mean duration of 1/2 hour. 
This queueing system cannot be modeled with level crossing arguments, hence we need  Sakasegawa's formula to compute the average time in queue.  
\begin{solution}
Answer = B, see~\cref{ex:92}.
\end{solution}
\end{exercise}


\subsection{Open Questions}



\begin{exercise}[201706]
  We have a machine that fails regularly, but we can control the way it fails.
  Option 1: Clean and tune the machine for precisely 15 minutes at the start of every hour.
  Option 2: Don't clean the machine, but wait until it fails and then do a repair.
  In the latter situation, the time between two failures is exponentially distributed with a mean of 6 hours.
  Repair times are i.i.d.
  and exponentially distributed with a mean of 1 hour.
  To simplify the analysis, you are allowed for Option 1 to model the time between the cleaning actions and the cleaning actions themselves as exponentially distributed.

Jobs arrive as a Poisson process with rate $2$ per hour, regular service times are exponential with a mean of $10$ minutes. Which of the two options would you prefer?
\begin{solution}
Note, the time between two failures is not the same as the time to failure.


Option 1. The availability is $A=45/60=3/4$. Hence, 
\begin{equation*}
\E{S_e} = 10\cdot 4/3 = 40/3 \approx 13\text{ minutes}
\end{equation*}
We also have
\begin{equation*}
   C_e^2 = C_0^2 + 2A(1-A)\frac{m_r}{\E{S_0}} = 1 + 2\frac{3}{4}\frac1{4}\frac{15}{10}.
 \end{equation*}
Now,
\begin{equation*}
  \rho = \lambda \E{S_e} = \frac{2}{60}\frac{40}3.
\end{equation*}
Now we can fill in 
\begin{equation*}
  \E{W_Q} = \frac{1 + C_e^2}2 \frac{\rho}{1-\rho} \E{S_e}.
\end{equation*}
Finally, $E W = \E{W_Q} + \E{S_e}$.


Option 2. The availability is $A=50/60=5/6$. Hence, 
\begin{equation*}
\E{S_e} = 10\cdot 6/5 = 12\text{ minutes}
\end{equation*}
We also have
\begin{equation*}
   C_e^2 = C_0^2 + 2A(1-A)\frac{m_r}{\E{S_0}} = 1 + 2\frac{5}{6}\frac1{6}\frac{60}{10}.
 \end{equation*}
and
\begin{equation*}
  \rho = \lambda \E{S_e} = \frac{2}{60}12
\end{equation*}

It is essential that you realize that both $\E S$ and $C_e^2$ are affected by failures. If you forgot to compensate in either of the two, I subtracted one point. 
\end{solution}
\end{exercise}


A station contains 3 identical machines in parallel. Jobs arrive as a Poisson process with rate $3$ per hour. Service times are exponential with a mean duration of 1/2 hour. 

\begin{exercise}[201807]\label{ex:92}
  Is there a model by which we can obtain an exact answer for the average time the system contains $n$ jobs? 
\begin{solution}
    Yes, use the $M/M(n)/1$ queueing model, or the $M/M/3$ model (which is a subset of the $M/M(n)/1$ queue). 

Note the $M/M/1$ queue with a fast server is not the same as the $M/M/3$ server. The service process is different. 
\end{solution}
\end{exercise}


\begin{exercise}[201807]
Use Sakasegawa's approximation to compute the average time a job spends in queue. (You do not have to use your calculator to compute the final answer;. I just want to see the numerical values of each component of the formula, you don't have to compute the end result.)
\begin{solution}
$\rho = \lambda \E{S}/c = 3 *0.5/3=0.5$.
\begin{align*}
  \E{W_Q} 
&= \frac{C_a^2+C_s^2}2 \frac{\rho^{\sqrt{2(c+1)}-1}}{c(1-\rho)} \E S \\
&= \frac{1+1}2 \frac{0.5^{\sqrt{7}}}{3*0.5} *0.5 \\
&=\frac{0.5^{\sqrt{7}}}{3}. 
\end{align*}
\end{solution}
\end{exercise}


\begin{exercise}[201807]
  What is the average time a job spends in the system?
\begin{solution}
    This is $\E{W_Q} + \E S = \E{W_Q} + 0.5$
\end{solution}

Some students add three times the average service time, but servers in parallel are not servers in tandem\ldots
\end{exercise}


\begin{exercise}[201807]
Suppose now that each machine can fail between any two jobs with constant probability $p=1/3$. The repair time is constant, and takes 1 hour.  What is the average time a job spends in queue? (Again, it suffices if you show the numerical answer for each component in the formula.)
\begin{solution}
  The expected service time of a job now becomes 
  \begin{equation*}
    \E S = \E{S_0} + \E{T}/B = 0.5 + 1/3 = 5/6
  \end{equation*}
\pyc{v = 0.5*2 + 1/3.*2/9}
Next, with the formula sheet: 
\begin{equation*}
\V{S} = 0.5*2 + \frac 1 3 + \frac{3-1}{3^2} 1 = \py{v}.
\end{equation*}
With this, 
\begin{equation*}
  C_s^ = \frac{\V{S}}{(\E S)^2} = \V{S}\frac{36}{25} = \py{v*32/25}.
\end{equation*}
Finally, $\rho = 3*\E{S} / 3 = 5/6$, and fill in Sakasegawa's formula with these values.

Mind that failures affect $\E S$ \emph{and} $C_s^2$. If you forgot to compute $C_s^2$: -1/2.

Some students use the wrong failure model, the one with preemptive failures: -1/2.
\end{solution}
\end{exercise}


\subsection{simulation}
\label{sec:simulation}


In this section we will deal with the code below.

\begin{pyverbatim}[stdout][numbers=left]
from heapq import heappop, heappush
import numpy as np
from scipy.stats import expon

np.random.seed(3)

ARRIVAL = 0
DEPARTURE = 1

stack = []  # this is the event stack
queue = []
served_jobs = []  # used for statistics

num_jobs = 10
labda = 2.0
mu = 3.0
rho = labda / mu
F = expon(scale=1.0 / labda)  # interarrival time distributon
G = expon(scale=1.0 / mu)  # service time distributon


class Server:
    def __init__(self):
        self.busy = False


server = Server()


class Job:
    def __init__(self):
        self.arrival_time = 0
        self.service_time = 0
        self.departure_time = 0
        self.queue_length_at_arrival = 0

    def sojourn_time(self):
       pass # write your code below.
    

    def __repr__(self):
        return f"{self.arrival_time}, {self.service_time}, {self.departure_time}\n"


def start_service(time, job):
    server.busy = True
    job.departure_time = time + job.service_time
    heappush(stack, (job.departure_time, job, DEPARTURE))


def handle_arrival(time, job):
    job.queue_length_at_arrival = len(queue)
    if server.busy:
        heappush(queue, (job.arrival_time, job))
    else:
        start_service(time, job)


def handle_departure(time, job):
    server.busy = False
    if queue:  # queue is not empty
        time, next_job = heappop(queue)
        start_service(time, next_job)


time = 0
for i in range(num_jobs):
    job = Job()
    time += F.rvs()
    job.arrival_time = time
    job.service_time = G.rvs()
    heappush(stack, (job.arrival_time, job, ARRIVAL))


while stack:
    time, job, typ = heappop(stack)
    if typ == ARRIVAL:
        handle_arrival(time, job)
    else:
        handle_departure(time, job)
        served_jobs.append(job)

\end{pyverbatim}

\begin{exercise}[201907]\label{ex:2019_2_4}
The above code simulates the $G/G/1$ queue. Complete the code to compute the sojourn time. (You can write your code at the correct place.)

\begin{solution}
  This is the code:
  \begin{pyverbatim}
    def sojourn_time(self):
       return self.departure_time - self.arrival_time

  \end{pyverbatim}
  
\end{solution}
\end{exercise}

\begin{exercise}[201907]
 Explain the code from lines 45 to 48.

\begin{solution}
  First we set a flag to indicate that the server is occupied.
  Then we compute the departure time.
  Finally, we push the job on the event stack such that it will popped by the simulator at its departure time.
\end{solution}
\end{exercise}              

\begin{exercise}[201907] 
Suppose we want to implement the LIFO queue. Which line of the above code has to change, and what should the change be? (It is ok if your answer is not perfect python code; however, your answer should show your correct understanding.) 

\begin{solution}
\pyv{heappush(queue, (-job.arrival_time, job))}
\end{solution}

\end{exercise}              

\begin{exercise}[201907] 
  Suppose we want to implement the shortest-processing-time-first scheduling rule.
  Which line of the above code has to change, and what should the change be?

\begin{solution}
\pyv{heappush(queue, (job.service_time, job))}
\end{solution}
\end{exercise}              


\begin{exercise}[201907]
What is the aim of test-driven coding? 

\begin{solution}
  Write some small function.
  Test it on some known data.
  Once the function works, build a new function, test this, and so on.
  Then combine the tested functions to make something more difficult, test, etc.
\end{solution}
\end{exercise}

\begin{exercise}[201907]
  Why is an event stack an essential concept for the discrete-time simulation of complicated systems?

\begin{solution}
  The event stack keeps all events in the simulator ordered in time.
\end{solution}  
\end{exercise}

\Closesolutionfile{ans}
\subsection*{Solutions}
\input{ans}



%%% Local Variables:
%%% mode: latex
%%% TeX-master: "../companion"
%%% End:
